% Options for packages loaded elsewhere
\PassOptionsToPackage{unicode}{hyperref}
\PassOptionsToPackage{hyphens}{url}
%
\documentclass[
]{article}
\usepackage{lmodern}
\usepackage{amssymb,amsmath}
\usepackage{ifxetex,ifluatex}
\ifnum 0\ifxetex 1\fi\ifluatex 1\fi=0 % if pdftex
  \usepackage[T1]{fontenc}
  \usepackage[utf8]{inputenc}
  \usepackage{textcomp} % provide euro and other symbols
\else % if luatex or xetex
  \usepackage{unicode-math}
  \defaultfontfeatures{Scale=MatchLowercase}
  \defaultfontfeatures[\rmfamily]{Ligatures=TeX,Scale=1}
\fi
% Use upquote if available, for straight quotes in verbatim environments
\IfFileExists{upquote.sty}{\usepackage{upquote}}{}
\IfFileExists{microtype.sty}{% use microtype if available
  \usepackage[]{microtype}
  \UseMicrotypeSet[protrusion]{basicmath} % disable protrusion for tt fonts
}{}
\makeatletter
\@ifundefined{KOMAClassName}{% if non-KOMA class
  \IfFileExists{parskip.sty}{%
    \usepackage{parskip}
  }{% else
    \setlength{\parindent}{0pt}
    \setlength{\parskip}{6pt plus 2pt minus 1pt}}
}{% if KOMA class
  \KOMAoptions{parskip=half}}
\makeatother
\usepackage{xcolor}
\IfFileExists{xurl.sty}{\usepackage{xurl}}{} % add URL line breaks if available
\IfFileExists{bookmark.sty}{\usepackage{bookmark}}{\usepackage{hyperref}}
\hypersetup{
  pdftitle={Contact Experiment Data Analysis},
  hidelinks,
  pdfcreator={LaTeX via pandoc}}
\urlstyle{same} % disable monospaced font for URLs
\usepackage[margin=1in]{geometry}
\usepackage{color}
\usepackage{fancyvrb}
\newcommand{\VerbBar}{|}
\newcommand{\VERB}{\Verb[commandchars=\\\{\}]}
\DefineVerbatimEnvironment{Highlighting}{Verbatim}{commandchars=\\\{\}}
% Add ',fontsize=\small' for more characters per line
\usepackage{framed}
\definecolor{shadecolor}{RGB}{248,248,248}
\newenvironment{Shaded}{\begin{snugshade}}{\end{snugshade}}
\newcommand{\AlertTok}[1]{\textcolor[rgb]{0.94,0.16,0.16}{#1}}
\newcommand{\AnnotationTok}[1]{\textcolor[rgb]{0.56,0.35,0.01}{\textbf{\textit{#1}}}}
\newcommand{\AttributeTok}[1]{\textcolor[rgb]{0.77,0.63,0.00}{#1}}
\newcommand{\BaseNTok}[1]{\textcolor[rgb]{0.00,0.00,0.81}{#1}}
\newcommand{\BuiltInTok}[1]{#1}
\newcommand{\CharTok}[1]{\textcolor[rgb]{0.31,0.60,0.02}{#1}}
\newcommand{\CommentTok}[1]{\textcolor[rgb]{0.56,0.35,0.01}{\textit{#1}}}
\newcommand{\CommentVarTok}[1]{\textcolor[rgb]{0.56,0.35,0.01}{\textbf{\textit{#1}}}}
\newcommand{\ConstantTok}[1]{\textcolor[rgb]{0.00,0.00,0.00}{#1}}
\newcommand{\ControlFlowTok}[1]{\textcolor[rgb]{0.13,0.29,0.53}{\textbf{#1}}}
\newcommand{\DataTypeTok}[1]{\textcolor[rgb]{0.13,0.29,0.53}{#1}}
\newcommand{\DecValTok}[1]{\textcolor[rgb]{0.00,0.00,0.81}{#1}}
\newcommand{\DocumentationTok}[1]{\textcolor[rgb]{0.56,0.35,0.01}{\textbf{\textit{#1}}}}
\newcommand{\ErrorTok}[1]{\textcolor[rgb]{0.64,0.00,0.00}{\textbf{#1}}}
\newcommand{\ExtensionTok}[1]{#1}
\newcommand{\FloatTok}[1]{\textcolor[rgb]{0.00,0.00,0.81}{#1}}
\newcommand{\FunctionTok}[1]{\textcolor[rgb]{0.00,0.00,0.00}{#1}}
\newcommand{\ImportTok}[1]{#1}
\newcommand{\InformationTok}[1]{\textcolor[rgb]{0.56,0.35,0.01}{\textbf{\textit{#1}}}}
\newcommand{\KeywordTok}[1]{\textcolor[rgb]{0.13,0.29,0.53}{\textbf{#1}}}
\newcommand{\NormalTok}[1]{#1}
\newcommand{\OperatorTok}[1]{\textcolor[rgb]{0.81,0.36,0.00}{\textbf{#1}}}
\newcommand{\OtherTok}[1]{\textcolor[rgb]{0.56,0.35,0.01}{#1}}
\newcommand{\PreprocessorTok}[1]{\textcolor[rgb]{0.56,0.35,0.01}{\textit{#1}}}
\newcommand{\RegionMarkerTok}[1]{#1}
\newcommand{\SpecialCharTok}[1]{\textcolor[rgb]{0.00,0.00,0.00}{#1}}
\newcommand{\SpecialStringTok}[1]{\textcolor[rgb]{0.31,0.60,0.02}{#1}}
\newcommand{\StringTok}[1]{\textcolor[rgb]{0.31,0.60,0.02}{#1}}
\newcommand{\VariableTok}[1]{\textcolor[rgb]{0.00,0.00,0.00}{#1}}
\newcommand{\VerbatimStringTok}[1]{\textcolor[rgb]{0.31,0.60,0.02}{#1}}
\newcommand{\WarningTok}[1]{\textcolor[rgb]{0.56,0.35,0.01}{\textbf{\textit{#1}}}}
\usepackage{longtable,booktabs}
% Correct order of tables after \paragraph or \subparagraph
\usepackage{etoolbox}
\makeatletter
\patchcmd\longtable{\par}{\if@noskipsec\mbox{}\fi\par}{}{}
\makeatother
% Allow footnotes in longtable head/foot
\IfFileExists{footnotehyper.sty}{\usepackage{footnotehyper}}{\usepackage{footnote}}
\makesavenoteenv{longtable}
\usepackage{graphicx,grffile}
\makeatletter
\def\maxwidth{\ifdim\Gin@nat@width>\linewidth\linewidth\else\Gin@nat@width\fi}
\def\maxheight{\ifdim\Gin@nat@height>\textheight\textheight\else\Gin@nat@height\fi}
\makeatother
% Scale images if necessary, so that they will not overflow the page
% margins by default, and it is still possible to overwrite the defaults
% using explicit options in \includegraphics[width, height, ...]{}
\setkeys{Gin}{width=\maxwidth,height=\maxheight,keepaspectratio}
% Set default figure placement to htbp
\makeatletter
\def\fps@figure{htbp}
\makeatother
\setlength{\emergencystretch}{3em} % prevent overfull lines
\providecommand{\tightlist}{%
  \setlength{\itemsep}{0pt}\setlength{\parskip}{0pt}}
\setcounter{secnumdepth}{-\maxdimen} % remove section numbering

\title{Contact Experiment Data Analysis}
\author{}
\date{\vspace{-2.5em}}

\begin{document}
\maketitle

First Step is to load the necessary package, If you dont have them just
install them. For jjstatsplot you need to install it remotely. Just
remove the dash and press enter. Then Press 3 (none package to be
updated). ?group\_by()

\begin{Shaded}
\begin{Highlighting}[]
\CommentTok{#remotes::install_github("sbalci/jjstatsplot") #Press 3 !!!!! i.e., installing/Updating none package! }
\KeywordTok{library}\NormalTok{(jmv)}
\KeywordTok{library}\NormalTok{(datasets)}
\KeywordTok{library}\NormalTok{(plyr)}
\KeywordTok{library}\NormalTok{(readr)}
\KeywordTok{library}\NormalTok{(dataframes2xls)}
\KeywordTok{library}\NormalTok{(data.table)}
\KeywordTok{library}\NormalTok{(plyr)}
\KeywordTok{library}\NormalTok{(ggstatsplot)}
\end{Highlighting}
\end{Shaded}

\begin{verbatim}
## Registered S3 method overwritten by 'broom.mixed':
##   method      from 
##   tidy.gamlss broom
\end{verbatim}

\begin{verbatim}
## Registered S3 methods overwritten by 'lme4':
##   method                          from
##   cooks.distance.influence.merMod car 
##   influence.merMod                car 
##   dfbeta.influence.merMod         car 
##   dfbetas.influence.merMod        car
\end{verbatim}

\begin{verbatim}
## In case you would like cite this package, cite it as:
##      Patil, I. (2018). ggstatsplot: "ggplot2" Based Plots with Statistical Details. CRAN.
##      Retrieved from https://cran.r-project.org/web/packages/ggstatsplot/index.html
\end{verbatim}

\begin{Shaded}
\begin{Highlighting}[]
\KeywordTok{library}\NormalTok{(jjstatsplot)}
\KeywordTok{library}\NormalTok{(lme4)}
\end{Highlighting}
\end{Shaded}

\begin{verbatim}
## Loading required package: Matrix
\end{verbatim}

\begin{Shaded}
\begin{Highlighting}[]
\KeywordTok{library}\NormalTok{(lmerTest)}
\end{Highlighting}
\end{Shaded}

\begin{verbatim}
## 
## Attaching package: 'lmerTest'
\end{verbatim}

\begin{verbatim}
## The following object is masked from 'package:lme4':
## 
##     lmer
\end{verbatim}

\begin{verbatim}
## The following object is masked from 'package:stats':
## 
##     step
\end{verbatim}

\begin{Shaded}
\begin{Highlighting}[]
\KeywordTok{library}\NormalTok{(ggplot2)}
\KeywordTok{library}\NormalTok{(rstatix)}
\end{Highlighting}
\end{Shaded}

\begin{verbatim}
## 
## Attaching package: 'rstatix'
\end{verbatim}

\begin{verbatim}
## The following objects are masked from 'package:plyr':
## 
##     desc, mutate
\end{verbatim}

\begin{verbatim}
## The following object is masked from 'package:stats':
## 
##     filter
\end{verbatim}

\begin{Shaded}
\begin{Highlighting}[]
\KeywordTok{library}\NormalTok{(coin)}
\end{Highlighting}
\end{Shaded}

\begin{verbatim}
## Loading required package: survival
\end{verbatim}

\begin{verbatim}
## 
## Attaching package: 'coin'
\end{verbatim}

\begin{verbatim}
## The following objects are masked from 'package:rstatix':
## 
##     chisq_test, friedman_test, kruskal_test, sign_test, wilcox_test
\end{verbatim}

\begin{Shaded}
\begin{Highlighting}[]
\KeywordTok{library}\NormalTok{(ARTool)}
\KeywordTok{library}\NormalTok{(ggpubr)}
\end{Highlighting}
\end{Shaded}

\begin{verbatim}
## 
## Attaching package: 'ggpubr'
\end{verbatim}

\begin{verbatim}
## The following object is masked from 'package:plyr':
## 
##     mutate
\end{verbatim}

\begin{Shaded}
\begin{Highlighting}[]
\KeywordTok{library}\NormalTok{(tidyverse)}
\end{Highlighting}
\end{Shaded}

\begin{verbatim}
## Found more than one class "atomicVector" in cache; using the first, from namespace 'Matrix'
\end{verbatim}

\begin{verbatim}
## Also defined by 'Rmpfr'
\end{verbatim}

\begin{verbatim}
## Found more than one class "atomicVector" in cache; using the first, from namespace 'Matrix'
\end{verbatim}

\begin{verbatim}
## Also defined by 'Rmpfr'
\end{verbatim}

\begin{verbatim}
## Found more than one class "atomicVector" in cache; using the first, from namespace 'Matrix'
\end{verbatim}

\begin{verbatim}
## Also defined by 'Rmpfr'
\end{verbatim}

\begin{verbatim}
## Found more than one class "atomicVector" in cache; using the first, from namespace 'Matrix'
\end{verbatim}

\begin{verbatim}
## Also defined by 'Rmpfr'
\end{verbatim}

\begin{verbatim}
## Found more than one class "atomicVector" in cache; using the first, from namespace 'Matrix'
\end{verbatim}

\begin{verbatim}
## Also defined by 'Rmpfr'
\end{verbatim}

\begin{verbatim}
## Found more than one class "atomicVector" in cache; using the first, from namespace 'Matrix'
\end{verbatim}

\begin{verbatim}
## Also defined by 'Rmpfr'
\end{verbatim}

\begin{verbatim}
## Found more than one class "atomicVector" in cache; using the first, from namespace 'Matrix'
\end{verbatim}

\begin{verbatim}
## Also defined by 'Rmpfr'
\end{verbatim}

\begin{verbatim}
## Found more than one class "atomicVector" in cache; using the first, from namespace 'Matrix'
\end{verbatim}

\begin{verbatim}
## Also defined by 'Rmpfr'
\end{verbatim}

\begin{verbatim}
## Found more than one class "atomicVector" in cache; using the first, from namespace 'Matrix'
\end{verbatim}

\begin{verbatim}
## Also defined by 'Rmpfr'
\end{verbatim}

\begin{verbatim}
## Found more than one class "atomicVector" in cache; using the first, from namespace 'Matrix'
\end{verbatim}

\begin{verbatim}
## Also defined by 'Rmpfr'
\end{verbatim}

\begin{verbatim}
## Found more than one class "atomicVector" in cache; using the first, from namespace 'Matrix'
\end{verbatim}

\begin{verbatim}
## Also defined by 'Rmpfr'
\end{verbatim}

\begin{verbatim}
## Found more than one class "atomicVector" in cache; using the first, from namespace 'Matrix'
\end{verbatim}

\begin{verbatim}
## Also defined by 'Rmpfr'
\end{verbatim}

\begin{verbatim}
## -- Attaching packages --------------------------------------- tidyverse 1.3.0 --
\end{verbatim}

\begin{verbatim}
## v tibble  3.0.4     v dplyr   1.0.2
## v tidyr   1.1.2     v stringr 1.4.0
## v purrr   0.3.4     v forcats 0.5.0
\end{verbatim}

\begin{verbatim}
## -- Conflicts ------------------------------------------ tidyverse_conflicts() --
## x dplyr::arrange()   masks plyr::arrange()
## x dplyr::between()   masks data.table::between()
## x purrr::compact()   masks plyr::compact()
## x dplyr::count()     masks plyr::count()
## x tidyr::expand()    masks Matrix::expand()
## x dplyr::failwith()  masks plyr::failwith()
## x dplyr::filter()    masks rstatix::filter(), stats::filter()
## x dplyr::first()     masks data.table::first()
## x dplyr::id()        masks plyr::id()
## x dplyr::lag()       masks stats::lag()
## x dplyr::last()      masks data.table::last()
## x dplyr::mutate()    masks ggpubr::mutate(), rstatix::mutate(), plyr::mutate()
## x tidyr::pack()      masks Matrix::pack()
## x dplyr::rename()    masks plyr::rename()
## x dplyr::summarise() masks plyr::summarise()
## x dplyr::summarize() masks plyr::summarize()
## x purrr::transpose() masks data.table::transpose()
## x tidyr::unpack()    masks Matrix::unpack()
\end{verbatim}

\begin{Shaded}
\begin{Highlighting}[]
\KeywordTok{library}\NormalTok{(dplyr)}
\KeywordTok{library}\NormalTok{(}\StringTok{"afex"}\NormalTok{)     }
\end{Highlighting}
\end{Shaded}

\begin{verbatim}
## ************
## Welcome to afex. For support visit: http://afex.singmann.science/
\end{verbatim}

\begin{verbatim}
## - Functions for ANOVAs: aov_car(), aov_ez(), and aov_4()
## - Methods for calculating p-values with mixed(): 'KR', 'S', 'LRT', and 'PB'
## - 'afex_aov' and 'mixed' objects can be passed to emmeans() for follow-up tests
## - NEWS: library('emmeans') now needs to be called explicitly!
## - Get and set global package options with: afex_options()
## - Set orthogonal sum-to-zero contrasts globally: set_sum_contrasts()
## - For example analyses see: browseVignettes("afex")
## ************
\end{verbatim}

\begin{verbatim}
## 
## Attaching package: 'afex'
\end{verbatim}

\begin{verbatim}
## The following object is masked from 'package:lme4':
## 
##     lmer
\end{verbatim}

\begin{Shaded}
\begin{Highlighting}[]
\KeywordTok{library}\NormalTok{(}\StringTok{"emmeans"}\NormalTok{)  }
\KeywordTok{library}\NormalTok{(}\StringTok{"multcomp"}\NormalTok{) }
\end{Highlighting}
\end{Shaded}

\begin{verbatim}
## Loading required package: mvtnorm
\end{verbatim}

\begin{verbatim}
## Loading required package: TH.data
\end{verbatim}

\begin{verbatim}
## Loading required package: MASS
\end{verbatim}

\begin{verbatim}
## 
## Attaching package: 'MASS'
\end{verbatim}

\begin{verbatim}
## The following object is masked from 'package:dplyr':
## 
##     select
\end{verbatim}

\begin{verbatim}
## The following object is masked from 'package:rstatix':
## 
##     select
\end{verbatim}

\begin{verbatim}
## 
## Attaching package: 'TH.data'
\end{verbatim}

\begin{verbatim}
## The following object is masked from 'package:MASS':
## 
##     geyser
\end{verbatim}

Import and merge all the csv files (x.csv where x = ID) of the folder.
Note, that the ID is has already been added as column

Discard first X trials per interpenetration feedback condition and then
create a summary table for each participant. You need to define
\textbf{nrTrialsPerBlockToRemove}.

\begin{Shaded}
\begin{Highlighting}[]
\NormalTok{data}\OperatorTok{$}\NormalTok{Part <-}\StringTok{ }\KeywordTok{as.factor}\NormalTok{(data}\OperatorTok{$}\NormalTok{Block }\OperatorTok{<}\StringTok{ }\DecValTok{4}\NormalTok{)}
\KeywordTok{levels}\NormalTok{(data}\OperatorTok{$}\NormalTok{Part)}
\end{Highlighting}
\end{Shaded}

\begin{verbatim}
## [1] "FALSE" "TRUE"
\end{verbatim}

\begin{Shaded}
\begin{Highlighting}[]
\NormalTok{data}\OperatorTok{$}\NormalTok{Part <-}\StringTok{ }\KeywordTok{factor}\NormalTok{(data}\OperatorTok{$}\NormalTok{Part,}\DataTypeTok{levels =} \KeywordTok{c}\NormalTok{(}\StringTok{"TRUE"}\NormalTok{,}\StringTok{"FALSE"}\NormalTok{),}
                  \DataTypeTok{labels =} \KeywordTok{c}\NormalTok{(}\StringTok{"Part 1"}\NormalTok{,}\StringTok{"Part 2"}\NormalTok{))}

\NormalTok{nrTrialsPerBlockToRemove <-}\StringTok{ }\DecValTok{1}
\CommentTok{#trialsToRemove <- seq(from = 1, to = nrTrialsPerBlockToRemove)}

\NormalTok{  data <-}\StringTok{  }\NormalTok{data }\OperatorTok
\StringTok{    }\KeywordTok{group_by}\NormalTok{(ID, Part, InterpenetrationFeedback, FullyShaded) }\OperatorTok\StringTok{ }\CommentTok{# I have added here the fully shaded }
\StringTok{    }\KeywordTok{slice}\NormalTok{(nrTrialsPerBlockToRemove}\OperatorTok{+}\DecValTok{1}\OperatorTok{:}\KeywordTok{n}\NormalTok{())}
  \CommentTok{# to double check we are discarding the right rows}
  \CommentTok{#print(data[[i]]$Trial)}

 \CommentTok{# This df will be used to create the subsets for 1st part and 2nd part of the experiment. }
  
\NormalTok{data}\OperatorTok{$}\NormalTok{InterpenetrationFeedback  <-}\StringTok{ }\KeywordTok{as.factor}\NormalTok{(data}\OperatorTok{$}\NormalTok{InterpenetrationFeedback)}
\NormalTok{data}\OperatorTok{$}\NormalTok{FullyShaded <-}\StringTok{ }\KeywordTok{as.factor}\NormalTok{(data}\OperatorTok{$}\NormalTok{FullyShaded)}
\end{Highlighting}
\end{Shaded}

\begin{Shaded}
\begin{Highlighting}[]
\NormalTok{ParsiDF <-}\StringTok{ }\NormalTok{data}


\CommentTok{# Exclude the IDs which produced the extreme values (i.e., = or > 3 coefficients from the mean)}
\NormalTok{ParsiDF}\OperatorTok{$}\NormalTok{ID[ParsiDF}\OperatorTok{$}\NormalTok{ID }\OperatorTok{==}\StringTok{ }\DecValTok{9}\NormalTok{] <-}\StringTok{ }\OtherTok{NA} 
\NormalTok{ParsiDF}\OperatorTok{$}\NormalTok{ID[ParsiDF}\OperatorTok{$}\NormalTok{ID }\OperatorTok{==}\StringTok{ }\DecValTok{17}\NormalTok{] <-}\StringTok{ }\OtherTok{NA}
\NormalTok{ParsiDF}\OperatorTok{$}\NormalTok{ID[ParsiDF}\OperatorTok{$}\NormalTok{ID }\OperatorTok{==}\StringTok{ }\DecValTok{20}\NormalTok{] <-}\StringTok{ }\OtherTok{NA}

\NormalTok{ParsiDF <-}\StringTok{ }\KeywordTok{na.omit}\NormalTok{(ParsiDF)}

\NormalTok{ParsiDF <-}\StringTok{ }\KeywordTok{aggregate}\NormalTok{(. }\OperatorTok{~}\StringTok{ }\NormalTok{ID }\OperatorTok{+}\StringTok{ }\NormalTok{Age }\OperatorTok{+}\StringTok{ }\NormalTok{Gender }\OperatorTok{+}\StringTok{ }\NormalTok{InterpenetrationFeedback }\OperatorTok{+}\StringTok{ }\NormalTok{Part, ParsiDF, mean)}

\CommentTok{#Before Conversion to logarithms (showing the abnormal distribution)}
\KeywordTok{shapiro_test}\NormalTok{(ParsiDF}\OperatorTok{$}\NormalTok{MaxInterpenetration)}
\end{Highlighting}
\end{Shaded}

\begin{verbatim}
## # A tibble: 1 x 3
##   variable                    statistic     p.value
##   <chr>                           <dbl>       <dbl>
## 1 ParsiDF$MaxInterpenetration     0.937 0.000000946
\end{verbatim}

\begin{Shaded}
\begin{Highlighting}[]
\KeywordTok{shapiro_test}\NormalTok{(ParsiDF}\OperatorTok{$}\NormalTok{AverageInterpenetration)}
\end{Highlighting}
\end{Shaded}

\begin{verbatim}
## # A tibble: 1 x 3
##   variable                        statistic     p.value
##   <chr>                               <dbl>       <dbl>
## 1 ParsiDF$AverageInterpenetration     0.936 0.000000844
\end{verbatim}

\begin{Shaded}
\begin{Highlighting}[]
\KeywordTok{ggqqplot}\NormalTok{(ParsiDF}\OperatorTok{$}\NormalTok{MaxInterpenetration)}
\end{Highlighting}
\end{Shaded}

\includegraphics{Report_files/figure-latex/unnamed-chunk-4-1.pdf}

\begin{Shaded}
\begin{Highlighting}[]
\KeywordTok{ggqqplot}\NormalTok{(ParsiDF}\OperatorTok{$}\NormalTok{AverageInterpenetration)}
\end{Highlighting}
\end{Shaded}

\includegraphics{Report_files/figure-latex/unnamed-chunk-4-2.pdf}

\begin{Shaded}
\begin{Highlighting}[]
\NormalTok{ParsiDF }\OperatorTok
\StringTok{  }\KeywordTok{group_by}\NormalTok{(InterpenetrationFeedback, Part) }\OperatorTok
\StringTok{  }\KeywordTok{shapiro_test}\NormalTok{(MaxInterpenetration) }
\end{Highlighting}
\end{Shaded}

\begin{verbatim}
## # A tibble: 8 x 5
##   InterpenetrationFeedback Part   variable            statistic       p
##   <fct>                    <fct>  <chr>                   <dbl>   <dbl>
## 1 Both                     Part 1 MaxInterpenetration     0.946 0.290  
## 2 Both                     Part 2 MaxInterpenetration     0.834 0.00225
## 3 Electrotactile           Part 1 MaxInterpenetration     0.967 0.677  
## 4 Electrotactile           Part 2 MaxInterpenetration     0.859 0.00611
## 5 NoFeedback               Part 1 MaxInterpenetration     0.969 0.720  
## 6 NoFeedback               Part 2 MaxInterpenetration     0.977 0.872  
## 7 Visual                   Part 1 MaxInterpenetration     0.860 0.00622
## 8 Visual                   Part 2 MaxInterpenetration     0.898 0.0327
\end{verbatim}

\begin{Shaded}
\begin{Highlighting}[]
\NormalTok{ParsiDF }\OperatorTok
\StringTok{  }\KeywordTok{group_by}\NormalTok{(InterpenetrationFeedback, Part) }\OperatorTok
\StringTok{  }\KeywordTok{shapiro_test}\NormalTok{(AverageInterpenetration)  }
\end{Highlighting}
\end{Shaded}

\begin{verbatim}
## # A tibble: 8 x 5
##   InterpenetrationFeedback Part   variable                statistic       p
##   <fct>                    <fct>  <chr>                       <dbl>   <dbl>
## 1 Both                     Part 1 AverageInterpenetration     0.947 0.301  
## 2 Both                     Part 2 AverageInterpenetration     0.827 0.00175
## 3 Electrotactile           Part 1 AverageInterpenetration     0.964 0.592  
## 4 Electrotactile           Part 2 AverageInterpenetration     0.877 0.0128 
## 5 NoFeedback               Part 1 AverageInterpenetration     0.964 0.598  
## 6 NoFeedback               Part 2 AverageInterpenetration     0.955 0.420  
## 7 Visual                   Part 1 AverageInterpenetration     0.851 0.00445
## 8 Visual                   Part 2 AverageInterpenetration     0.914 0.0655
\end{verbatim}

\begin{Shaded}
\begin{Highlighting}[]
\CommentTok{#After Conversion of the performance variables into logs (Normal Distribution)}
\NormalTok{ParsiDF}\OperatorTok{$}\NormalTok{MaxInterpenetration <-}\StringTok{ }\KeywordTok{log}\NormalTok{(ParsiDF}\OperatorTok{$}\NormalTok{MaxInterpenetration)}

\NormalTok{ParsiDF}\OperatorTok{$}\NormalTok{AverageInterpenetration <-}\StringTok{ }\KeywordTok{log}\NormalTok{(ParsiDF}\OperatorTok{$}\NormalTok{AverageInterpenetration)}

\KeywordTok{shapiro_test}\NormalTok{(ParsiDF}\OperatorTok{$}\NormalTok{MaxInterpenetration)}
\end{Highlighting}
\end{Shaded}

\begin{verbatim}
## # A tibble: 1 x 3
##   variable                    statistic p.value
##   <chr>                           <dbl>   <dbl>
## 1 ParsiDF$MaxInterpenetration     0.988   0.149
\end{verbatim}

\begin{Shaded}
\begin{Highlighting}[]
\KeywordTok{shapiro_test}\NormalTok{(ParsiDF}\OperatorTok{$}\NormalTok{AverageInterpenetration)}
\end{Highlighting}
\end{Shaded}

\begin{verbatim}
## # A tibble: 1 x 3
##   variable                        statistic p.value
##   <chr>                               <dbl>   <dbl>
## 1 ParsiDF$AverageInterpenetration     0.986  0.0873
\end{verbatim}

\begin{Shaded}
\begin{Highlighting}[]
\KeywordTok{ggqqplot}\NormalTok{(ParsiDF}\OperatorTok{$}\NormalTok{MaxInterpenetration)}
\end{Highlighting}
\end{Shaded}

\includegraphics{Report_files/figure-latex/unnamed-chunk-4-3.pdf}

\begin{Shaded}
\begin{Highlighting}[]
\KeywordTok{ggqqplot}\NormalTok{(ParsiDF}\OperatorTok{$}\NormalTok{AverageInterpenetration)}
\end{Highlighting}
\end{Shaded}

\includegraphics{Report_files/figure-latex/unnamed-chunk-4-4.pdf}

\begin{Shaded}
\begin{Highlighting}[]
\CommentTok{#Let's check the assumption for each interpenetration feedback and shade condition}
\NormalTok{ParsiDF }\OperatorTok
\StringTok{  }\KeywordTok{group_by}\NormalTok{(InterpenetrationFeedback, Part) }\OperatorTok
\StringTok{  }\KeywordTok{shapiro_test}\NormalTok{(MaxInterpenetration) }
\end{Highlighting}
\end{Shaded}

\begin{verbatim}
## # A tibble: 8 x 5
##   InterpenetrationFeedback Part   variable            statistic     p
##   <fct>                    <fct>  <chr>                   <dbl> <dbl>
## 1 Both                     Part 1 MaxInterpenetration     0.987 0.989
## 2 Both                     Part 2 MaxInterpenetration     0.966 0.646
## 3 Electrotactile           Part 1 MaxInterpenetration     0.931 0.143
## 4 Electrotactile           Part 2 MaxInterpenetration     0.975 0.831
## 5 NoFeedback               Part 1 MaxInterpenetration     0.958 0.468
## 6 NoFeedback               Part 2 MaxInterpenetration     0.951 0.352
## 7 Visual                   Part 1 MaxInterpenetration     0.963 0.582
## 8 Visual                   Part 2 MaxInterpenetration     0.976 0.851
\end{verbatim}

\begin{Shaded}
\begin{Highlighting}[]
\NormalTok{ParsiDF }\OperatorTok
\StringTok{  }\KeywordTok{group_by}\NormalTok{(InterpenetrationFeedback, Part) }\OperatorTok
\StringTok{  }\KeywordTok{shapiro_test}\NormalTok{(AverageInterpenetration) }
\end{Highlighting}
\end{Shaded}

\begin{verbatim}
## # A tibble: 8 x 5
##   InterpenetrationFeedback Part   variable                statistic     p
##   <fct>                    <fct>  <chr>                       <dbl> <dbl>
## 1 Both                     Part 1 AverageInterpenetration     0.976 0.854
## 2 Both                     Part 2 AverageInterpenetration     0.970 0.740
## 3 Electrotactile           Part 1 AverageInterpenetration     0.954 0.396
## 4 Electrotactile           Part 2 AverageInterpenetration     0.979 0.906
## 5 NoFeedback               Part 1 AverageInterpenetration     0.956 0.434
## 6 NoFeedback               Part 2 AverageInterpenetration     0.977 0.873
## 7 Visual                   Part 1 AverageInterpenetration     0.955 0.428
## 8 Visual                   Part 2 AverageInterpenetration     0.964 0.600
\end{verbatim}

\begin{Shaded}
\begin{Highlighting}[]
\KeywordTok{hist}\NormalTok{(ParsiDF}\OperatorTok{$}\NormalTok{MaxInterpenetration,}\DataTypeTok{main =} \KeywordTok{paste}\NormalTok{(}\StringTok{"Histogram of Maximum Interpenetration"}\NormalTok{) , }\DataTypeTok{xlab =} \StringTok{"Maximum Interpenetration"}\NormalTok{)}
\end{Highlighting}
\end{Shaded}

\includegraphics{Report_files/figure-latex/unnamed-chunk-4-5.pdf}

\begin{Shaded}
\begin{Highlighting}[]
\KeywordTok{hist}\NormalTok{(ParsiDF}\OperatorTok{$}\NormalTok{AverageInterpenetration, }\DataTypeTok{main =} \KeywordTok{paste}\NormalTok{(}\StringTok{"Histogram of Average Interpenetration"}\NormalTok{) , }\DataTypeTok{xlab =} \StringTok{"Average Interpenetration"}\NormalTok{)}
\end{Highlighting}
\end{Shaded}

\includegraphics{Report_files/figure-latex/unnamed-chunk-4-6.pdf}

Let's visualize the data per interpenetration feedback and/or part of
the experiment (part 1 \& part 2).

\begin{Shaded}
\begin{Highlighting}[]
\NormalTok{ParsiDFplots <-}\StringTok{ }\NormalTok{data }\CommentTok{# A dataframe just for the plots, so we show everything in real numbers and in centimeters!}
\NormalTok{ParsiDFplots}\OperatorTok{$}\NormalTok{ID[ParsiDFplots}\OperatorTok{$}\NormalTok{ID }\OperatorTok{==}\StringTok{ }\DecValTok{9}\NormalTok{] <-}\StringTok{ }\OtherTok{NA}
\NormalTok{ParsiDFplots}\OperatorTok{$}\NormalTok{ID[ParsiDFplots}\OperatorTok{$}\NormalTok{ID }\OperatorTok{==}\StringTok{ }\DecValTok{17}\NormalTok{] <-}\StringTok{ }\OtherTok{NA}
\NormalTok{ParsiDFplots}\OperatorTok{$}\NormalTok{ID[ParsiDFplots}\OperatorTok{$}\NormalTok{ID }\OperatorTok{==}\StringTok{ }\DecValTok{20}\NormalTok{] <-}\StringTok{ }\OtherTok{NA}
\NormalTok{ParsiDFplots <-}\StringTok{ }\KeywordTok{na.omit}\NormalTok{(ParsiDFplots)}
\NormalTok{ParsiDFplots <-}\StringTok{ }\KeywordTok{aggregate}\NormalTok{(. }\OperatorTok{~}\StringTok{ }\NormalTok{ID }\OperatorTok{+}\StringTok{ }\NormalTok{Age }\OperatorTok{+}\StringTok{ }\NormalTok{Gender }\OperatorTok{+}\StringTok{ }\NormalTok{InterpenetrationFeedback }\OperatorTok{+}\StringTok{ }\NormalTok{Part, ParsiDFplots, mean)}

\NormalTok{ParsiDFplots}\OperatorTok{$}\NormalTok{AverageInterpenetration <-}\DecValTok{100} \OperatorTok{*}\StringTok{ }\NormalTok{ParsiDFplots}\OperatorTok{$}\NormalTok{AverageInterpenetration }\CommentTok{#Converting meters to centimeters}
\NormalTok{ParsiDFplots}\OperatorTok{$}\NormalTok{MaxInterpenetration <-}\StringTok{ }\DecValTok{100} \OperatorTok{*}\StringTok{ }\NormalTok{ParsiDFplots}\OperatorTok{$}\NormalTok{MaxInterpenetration }\CommentTok{#Converting meters to centimeters}

\NormalTok{p1 <-}\StringTok{ }\NormalTok{ggstatsplot}\OperatorTok{::}\KeywordTok{ggbetweenstats}\NormalTok{(}
  \DataTypeTok{data =}\NormalTok{ ParsiDFplots,}
  \DataTypeTok{x =} \StringTok{"InterpenetrationFeedback"}\NormalTok{, }\CommentTok{#Indepedent Variable}
  \DataTypeTok{y =} \StringTok{"MaxInterpenetration"}\NormalTok{, }\CommentTok{# Depedent Variable}
  \DataTypeTok{grouping.var =} \StringTok{"Part"}\NormalTok{, }\CommentTok{# 2nd IV }
  \DataTypeTok{type =} \StringTok{"p"}\NormalTok{, }\CommentTok{# parametric test i.e., p values}
  \DataTypeTok{pairwise.comparisons =} \OtherTok{FALSE}\NormalTok{, }\CommentTok{#compute pairwise comparisons}
  \DataTypeTok{pairwise.display =} \StringTok{"significant"}\NormalTok{, }\CommentTok{# show only the significant ones}
  \DataTypeTok{p.adjust.method =} \StringTok{"bonferroni"}\NormalTok{, }\CommentTok{# correction of p-value}
  \DataTypeTok{effsize.type =} \StringTok{"unbiased"}\NormalTok{, }\CommentTok{# Calculates the Hedge's g for t tests and the partial Omega for ANOVA}
  \DataTypeTok{results.subtitle =} \OtherTok{FALSE}\NormalTok{,}
  \DataTypeTok{xlab =} \StringTok{"Type of Feedback"}\NormalTok{, }\CommentTok{#label of X axis}
  \DataTypeTok{ylab =} \StringTok{"Maximum Interpenetration"}\NormalTok{, }\CommentTok{#label of y axis}
  \DataTypeTok{sample.size.label =} \OtherTok{FALSE}\NormalTok{,}
  \DataTypeTok{var.equal =} \OtherTok{TRUE}\NormalTok{, }\CommentTok{#Assuming Equal variances}
  \DataTypeTok{mean.plotting =} \OtherTok{FALSE}\NormalTok{,}
  \DataTypeTok{mean.ci =} \OtherTok{TRUE}\NormalTok{, }\CommentTok{#display the confidence interval of the mean}
  \DataTypeTok{paired =} \OtherTok{TRUE}\NormalTok{, }\CommentTok{#indicating that we have a within subject design}
  \DataTypeTok{title.text =} \StringTok{"Interpenetration Box-Violin Plots"}\NormalTok{,}
  \DataTypeTok{caption.text =} \StringTok{"Note: Interpenetration distance is displayed in cm."}\NormalTok{,}
  \DataTypeTok{title.color =} \StringTok{"black"}\NormalTok{,}
  \DataTypeTok{caption.color =} \StringTok{"black"}
\NormalTok{  ) }

\NormalTok{p2 <-}\StringTok{ }\NormalTok{ggstatsplot}\OperatorTok{::}\KeywordTok{ggbetweenstats}\NormalTok{(}
  \DataTypeTok{data =}\NormalTok{ ParsiDFplots,}
  \DataTypeTok{x =} \StringTok{"InterpenetrationFeedback"}\NormalTok{,}
  \DataTypeTok{y =} \StringTok{"AverageInterpenetration"}\NormalTok{,}
  \DataTypeTok{grouping.var =} \StringTok{"Part"}\NormalTok{,}
  \DataTypeTok{type =} \StringTok{"p"}\NormalTok{,}
  \DataTypeTok{pairwise.comparisons =} \OtherTok{FALSE}\NormalTok{,}
  \DataTypeTok{pairwise.display =} \StringTok{"significant"}\NormalTok{,}
  \DataTypeTok{p.adjust.method =} \StringTok{"bonferroni"}\NormalTok{,}
  \DataTypeTok{effsize.type =} \StringTok{"unbiased"}\NormalTok{,}
  \DataTypeTok{results.subtitle =} \OtherTok{FALSE}\NormalTok{,}
  \DataTypeTok{xlab =} \StringTok{"Type of Feedback"}\NormalTok{,}
  \DataTypeTok{ylab =} \StringTok{"Average Interpenetration"}\NormalTok{,}
  \DataTypeTok{sample.size.label =} \OtherTok{FALSE}\NormalTok{,}
  \DataTypeTok{var.equal =} \OtherTok{TRUE}\NormalTok{,}
  \DataTypeTok{mean.plotting =} \OtherTok{FALSE}\NormalTok{,}
  \DataTypeTok{mean.ci =} \OtherTok{TRUE}\NormalTok{,}
  \DataTypeTok{paired =} \OtherTok{TRUE}\NormalTok{,}
  \DataTypeTok{title.text =} \StringTok{"Interpenetration Box-Violin Plots"}\NormalTok{,}
  \DataTypeTok{caption.text =} \StringTok{"Note: Interpenetration distance is displayed in cm."}\NormalTok{,}
  \DataTypeTok{title.color =} \StringTok{"black"}\NormalTok{,}
  \DataTypeTok{caption.color =} \StringTok{"black"}
\NormalTok{)}


\CommentTok{# Replicating the above but this time we look on the effect of the type of feedback on the DVs in 1st and 2nd Part of the experiment individually}
\NormalTok{p3 <-}\StringTok{ }\NormalTok{ggstatsplot}\OperatorTok{::}\KeywordTok{grouped_ggbetweenstats}\NormalTok{(}
  \DataTypeTok{data =}\NormalTok{ ParsiDFplots,}
  \DataTypeTok{x =} \StringTok{"InterpenetrationFeedback"}\NormalTok{,}
  \DataTypeTok{y =} \StringTok{"MaxInterpenetration"}\NormalTok{,}
  \DataTypeTok{grouping.var =} \StringTok{"Part"}\NormalTok{,}
  \DataTypeTok{type =} \StringTok{"p"}\NormalTok{,}
  \DataTypeTok{pairwise.comparisons =} \OtherTok{FALSE}\NormalTok{,}
  \DataTypeTok{pairwise.display =} \StringTok{"significant"}\NormalTok{,}
  \DataTypeTok{p.adjust.method =} \StringTok{"bonferroni"}\NormalTok{,}
  \DataTypeTok{effsize.type =} \StringTok{"unbiased"}\NormalTok{,}
  \DataTypeTok{results.subtitle =} \OtherTok{FALSE}\NormalTok{,}
  \DataTypeTok{xlab =} \StringTok{"Type of Feedback"}\NormalTok{,}
  \DataTypeTok{ylab =} \StringTok{"Maximum Interpenetration"}\NormalTok{,}
  \DataTypeTok{sample.size.label =} \OtherTok{FALSE}\NormalTok{,}
  \DataTypeTok{var.equal =} \OtherTok{TRUE}\NormalTok{,}
  \DataTypeTok{mean.plotting =} \OtherTok{FALSE}\NormalTok{,}
  \DataTypeTok{mean.ci =} \OtherTok{TRUE}\NormalTok{,}
  \DataTypeTok{paired =} \OtherTok{TRUE}\NormalTok{,}
  \DataTypeTok{title.text =} \StringTok{"Interpenetration Box-Violin Plots"}\NormalTok{,}
  \DataTypeTok{caption.text =} \StringTok{"Note: Interpenetration distance is displayed in cm."}\NormalTok{,}
  \DataTypeTok{title.color =} \StringTok{"black"}\NormalTok{,}
  \DataTypeTok{caption.color =} \StringTok{"black"}
\NormalTok{  ) }

\NormalTok{p4 <-}\StringTok{ }\NormalTok{ggstatsplot}\OperatorTok{::}\KeywordTok{grouped_ggbetweenstats}\NormalTok{(}
  \DataTypeTok{data =}\NormalTok{ ParsiDFplots,}
  \DataTypeTok{x =} \StringTok{"InterpenetrationFeedback"}\NormalTok{,}
  \DataTypeTok{y =} \StringTok{"AverageInterpenetration"}\NormalTok{,}
  \DataTypeTok{grouping.var =} \StringTok{"Part"}\NormalTok{,}
  \DataTypeTok{type =} \StringTok{"p"}\NormalTok{,}
  \DataTypeTok{pairwise.comparisons =} \OtherTok{FALSE}\NormalTok{,}
  \DataTypeTok{pairwise.display =} \StringTok{"significant"}\NormalTok{,}
  \DataTypeTok{p.adjust.method =} \StringTok{"bonferroni"}\NormalTok{,}
  \DataTypeTok{effsize.type =} \StringTok{"unbiased"}\NormalTok{,}
  \DataTypeTok{results.subtitle =} \OtherTok{FALSE}\NormalTok{,}
  \DataTypeTok{xlab =} \StringTok{"Type of Feedback"}\NormalTok{,}
  \DataTypeTok{ylab =} \StringTok{"Average Interpenetration"}\NormalTok{,}
  \DataTypeTok{sample.size.label =} \OtherTok{FALSE}\NormalTok{,}
  \DataTypeTok{var.equal =} \OtherTok{TRUE}\NormalTok{,}
  \DataTypeTok{mean.plotting =} \OtherTok{FALSE}\NormalTok{,}
  \DataTypeTok{mean.ci =} \OtherTok{TRUE}\NormalTok{,}
  \DataTypeTok{paired =} \OtherTok{TRUE}\NormalTok{,}
  \DataTypeTok{title.text =} \StringTok{"Interpenetration Box-Violin Plots"}\NormalTok{,}
  \DataTypeTok{caption.text =} \StringTok{"Note: Interpenetration distance is displayed in cm."}\NormalTok{,}
  \DataTypeTok{title.color =} \StringTok{"black"}\NormalTok{,}
  \DataTypeTok{caption.color =} \StringTok{"black"}
\NormalTok{) }

\CommentTok{# Lets check the effect of shaded condition on DVs}
\NormalTok{p5 <-}\StringTok{ }\NormalTok{ggstatsplot}\OperatorTok{::}\StringTok{ }\KeywordTok{ggbetweenstats}\NormalTok{(}
  \DataTypeTok{data =}\NormalTok{ ParsiDFplots,}
  \DataTypeTok{x =} \StringTok{"Part"}\NormalTok{,}
  \DataTypeTok{y =} \StringTok{"MaxInterpenetration"}\NormalTok{,}
  \DataTypeTok{grouping.var =} \StringTok{"InterpenetrationFeedback"}\NormalTok{,}
  \DataTypeTok{type =} \StringTok{"p"}\NormalTok{,}
  \DataTypeTok{pairwise.comparisons =} \OtherTok{FALSE}\NormalTok{,}
  \DataTypeTok{pairwise.display =} \StringTok{"significant"}\NormalTok{,}
  \DataTypeTok{p.adjust.method =} \StringTok{"bonferroni"}\NormalTok{,}
  \DataTypeTok{effsize.type =} \StringTok{"unbiased"}\NormalTok{,}
  \DataTypeTok{results.subtitle =} \OtherTok{FALSE}\NormalTok{,}
  \DataTypeTok{xlab =} \StringTok{"Order"}\NormalTok{,}
  \DataTypeTok{ylab =} \StringTok{"Maximum Interpenetration"}\NormalTok{,}
  \DataTypeTok{sample.size.label =} \OtherTok{FALSE}\NormalTok{,}
  \DataTypeTok{var.equal =} \OtherTok{TRUE}\NormalTok{,}
  \DataTypeTok{mean.plotting =} \OtherTok{FALSE}\NormalTok{,}
  \DataTypeTok{mean.ci =} \OtherTok{TRUE}\NormalTok{,}
  \DataTypeTok{paired =} \OtherTok{TRUE}\NormalTok{,}
  \DataTypeTok{title.text =} \StringTok{"Interpenetration Box-Violin Plots"}\NormalTok{,}
  \DataTypeTok{caption.text =} \StringTok{"Note: Interpenetration distance is displayed in cm."}\NormalTok{,}
  \DataTypeTok{title.color =} \StringTok{"black"}\NormalTok{,}
  \DataTypeTok{caption.color =} \StringTok{"black"}
\NormalTok{  ) }

\NormalTok{p6 <-}\StringTok{ }\NormalTok{ggstatsplot}\OperatorTok{::}\KeywordTok{ggbetweenstats}\NormalTok{(}
  \DataTypeTok{data =}\NormalTok{ ParsiDFplots,}
  \DataTypeTok{x =} \StringTok{"Part"}\NormalTok{,}
  \DataTypeTok{y =} \StringTok{"AverageInterpenetration"}\NormalTok{,}
  \DataTypeTok{grouping.var =} \StringTok{"InterpenetrationFeedback"}\NormalTok{,}
  \DataTypeTok{type =} \StringTok{"p"}\NormalTok{,}
  \DataTypeTok{pairwise.comparisons =} \OtherTok{FALSE}\NormalTok{,}
  \DataTypeTok{pairwise.display =} \StringTok{"significant"}\NormalTok{,}
  \DataTypeTok{p.adjust.method =} \StringTok{"bonferroni"}\NormalTok{,}
  \DataTypeTok{effsize.type =} \StringTok{"unbiased"}\NormalTok{,}
  \DataTypeTok{results.subtitle =} \OtherTok{FALSE}\NormalTok{,}
  \DataTypeTok{xlab =} \StringTok{"Order"}\NormalTok{,}
  \DataTypeTok{ylab =} \StringTok{"Average Interpenetration"}\NormalTok{,}
  \DataTypeTok{sample.size.label =} \OtherTok{FALSE}\NormalTok{,}
  \DataTypeTok{var.equal =} \OtherTok{TRUE}\NormalTok{,}
  \DataTypeTok{mean.plotting =} \OtherTok{FALSE}\NormalTok{,}
  \DataTypeTok{mean.ci =} \OtherTok{TRUE}\NormalTok{,}
  \DataTypeTok{paired =} \OtherTok{TRUE}\NormalTok{,}
  \DataTypeTok{title.text =} \StringTok{"Interpenetration Box-Violin Plots"}\NormalTok{,}
  \DataTypeTok{caption.text =} \StringTok{"Note: Interpenetration distance is displayed in cm."}\NormalTok{,}
  \DataTypeTok{title.color =} \StringTok{"black"}\NormalTok{,}
  \DataTypeTok{caption.color =} \StringTok{"black"}
\NormalTok{) }

\NormalTok{p7 <-}\StringTok{ }\NormalTok{ggstatsplot}\OperatorTok{::}\KeywordTok{grouped_ggbetweenstats}\NormalTok{(}
  \DataTypeTok{data =}\NormalTok{ ParsiDFplots,}
  \DataTypeTok{x =} \StringTok{"Part"}\NormalTok{,}
  \DataTypeTok{y =} \StringTok{"MaxInterpenetration"}\NormalTok{,}
  \DataTypeTok{grouping.var =} \StringTok{"InterpenetrationFeedback"}\NormalTok{,}
  \DataTypeTok{type =} \StringTok{"p"}\NormalTok{,}
  \DataTypeTok{pairwise.comparisons =} \OtherTok{FALSE}\NormalTok{,}
  \DataTypeTok{pairwise.display =} \StringTok{"significant"}\NormalTok{,}
  \DataTypeTok{p.adjust.method =} \StringTok{"bonferroni"}\NormalTok{,}
  \DataTypeTok{effsize.type =} \StringTok{"unbiased"}\NormalTok{,}
  \DataTypeTok{results.subtitle =} \OtherTok{FALSE}\NormalTok{,}
  \DataTypeTok{xlab =} \StringTok{"Order"}\NormalTok{,}
  \DataTypeTok{ylab =} \StringTok{"Maximum Interpenetration"}\NormalTok{,}
  \DataTypeTok{sample.size.label =} \OtherTok{FALSE}\NormalTok{,}
  \DataTypeTok{var.equal =} \OtherTok{TRUE}\NormalTok{,}
  \DataTypeTok{mean.plotting =} \OtherTok{FALSE}\NormalTok{,}
  \DataTypeTok{mean.ci =} \OtherTok{TRUE}\NormalTok{,}
  \DataTypeTok{paired =} \OtherTok{TRUE}\NormalTok{,}
  \DataTypeTok{title.text =} \StringTok{"Interpenetration Box-Violin Plots"}\NormalTok{,}
  \DataTypeTok{caption.text =} \StringTok{"Note: Interpenetration distance is displayed in cm."}\NormalTok{,}
  \DataTypeTok{title.color =} \StringTok{"black"}\NormalTok{,}
  \DataTypeTok{caption.color =} \StringTok{"black"}\NormalTok{) }

\NormalTok{p8 <-}\StringTok{ }\NormalTok{ggstatsplot}\OperatorTok{::}\KeywordTok{grouped_ggbetweenstats}\NormalTok{(}
  \DataTypeTok{data =}\NormalTok{ ParsiDFplots,}
  \DataTypeTok{x =} \StringTok{"Part"}\NormalTok{,}
  \DataTypeTok{y =} \StringTok{"AverageInterpenetration"}\NormalTok{,}
  \DataTypeTok{grouping.var =} \StringTok{"InterpenetrationFeedback"}\NormalTok{,}
  \DataTypeTok{type =} \StringTok{"p"}\NormalTok{,}
  \DataTypeTok{pairwise.comparisons =} \OtherTok{FALSE}\NormalTok{,}
  \DataTypeTok{pairwise.display =} \StringTok{"significant"}\NormalTok{,}
  \DataTypeTok{p.adjust.method =} \StringTok{"bonferroni"}\NormalTok{,}
  \DataTypeTok{effsize.type =} \StringTok{"unbiased"}\NormalTok{,}
  \DataTypeTok{results.subtitle =} \OtherTok{FALSE}\NormalTok{,}
  \DataTypeTok{xlab =} \StringTok{"Order"}\NormalTok{,}
  \DataTypeTok{ylab =} \StringTok{"Average Interpenetration"}\NormalTok{,}
  \DataTypeTok{sample.size.label =} \OtherTok{FALSE}\NormalTok{,}
  \DataTypeTok{var.equal =} \OtherTok{TRUE}\NormalTok{,}
  \DataTypeTok{mean.plotting =} \OtherTok{FALSE}\NormalTok{,}
  \DataTypeTok{mean.ci =} \OtherTok{TRUE}\NormalTok{,}
  \DataTypeTok{paired =} \OtherTok{TRUE}\NormalTok{,}
  \DataTypeTok{title.text =} \StringTok{"Interpenetration Box-Violin Plots"}\NormalTok{,}
  \DataTypeTok{caption.text =} \StringTok{"Note: Interpenetration distance is displayed in cm."}\NormalTok{,}
  \DataTypeTok{title.color =} \StringTok{"black"}\NormalTok{,}
  \DataTypeTok{caption.color =} \StringTok{"black"}
\NormalTok{  ) }
\NormalTok{p1}
\end{Highlighting}
\end{Shaded}

\includegraphics{Report_files/figure-latex/unnamed-chunk-5-1.pdf}

\begin{Shaded}
\begin{Highlighting}[]
\NormalTok{p2}
\end{Highlighting}
\end{Shaded}

\includegraphics{Report_files/figure-latex/unnamed-chunk-5-2.pdf}

\begin{Shaded}
\begin{Highlighting}[]
\NormalTok{p3 }
\end{Highlighting}
\end{Shaded}

\includegraphics{Report_files/figure-latex/unnamed-chunk-5-3.pdf}

\begin{Shaded}
\begin{Highlighting}[]
\NormalTok{p4}
\end{Highlighting}
\end{Shaded}

\includegraphics{Report_files/figure-latex/unnamed-chunk-5-4.pdf}

\begin{Shaded}
\begin{Highlighting}[]
\NormalTok{p5 }
\end{Highlighting}
\end{Shaded}

\includegraphics{Report_files/figure-latex/unnamed-chunk-5-5.pdf}

\begin{Shaded}
\begin{Highlighting}[]
\NormalTok{p6}
\end{Highlighting}
\end{Shaded}

\includegraphics{Report_files/figure-latex/unnamed-chunk-5-6.pdf}

\begin{Shaded}
\begin{Highlighting}[]
\NormalTok{p7}
\end{Highlighting}
\end{Shaded}

\includegraphics{Report_files/figure-latex/unnamed-chunk-5-7.pdf}

\begin{Shaded}
\begin{Highlighting}[]
\NormalTok{p8}
\end{Highlighting}
\end{Shaded}

\includegraphics{Report_files/figure-latex/unnamed-chunk-5-8.pdf}

Let's check the Two Way Repeated Measures ANOVA

\begin{Shaded}
\begin{Highlighting}[]
\NormalTok{aMax <-}\StringTok{ }\KeywordTok{aov_ez}\NormalTok{(}\StringTok{"ID"}\NormalTok{, }\StringTok{"MaxInterpenetration"}\NormalTok{, ParsiDF,}
             \DataTypeTok{within =} \KeywordTok{c}\NormalTok{(}\StringTok{"Part"}\NormalTok{, }\StringTok{"InterpenetrationFeedback"}\NormalTok{),}
             \DataTypeTok{anova_table =} \KeywordTok{list}\NormalTok{(}\DataTypeTok{es =} \StringTok{"pes"}\NormalTok{))}

\NormalTok{knitr}\OperatorTok{::}\KeywordTok{kable}\NormalTok{(}\KeywordTok{nice}\NormalTok{(aMax}\OperatorTok{$}\NormalTok{anova_table))}
\end{Highlighting}
\end{Shaded}

\begin{longtable}[]{@{}llllll@{}}
\toprule
Effect & df & MSE & F & pes & p.value\tabularnewline
\midrule
\endhead
Part & 1, 20 & 0.06 & 29.43 *** & .595 & \textless.001\tabularnewline
InterpenetrationFeedback & 1.88, 37.52 & 0.12 & 28.36 *** & .586 &
\textless.001\tabularnewline
Part:InterpenetrationFeedback & 2.06, 41.24 & 0.04 & 5.16 ** & .205 &
.009\tabularnewline
\bottomrule
\end{longtable}

\begin{Shaded}
\begin{Highlighting}[]
\NormalTok{aAv <-}\StringTok{ }\KeywordTok{aov_ez}\NormalTok{(}\StringTok{"ID"}\NormalTok{, }\StringTok{"AverageInterpenetration"}\NormalTok{, ParsiDF,}
             \DataTypeTok{within =} \KeywordTok{c}\NormalTok{(}\StringTok{"Part"}\NormalTok{, }\StringTok{"InterpenetrationFeedback"}\NormalTok{),}
             \DataTypeTok{anova_table =} \KeywordTok{list}\NormalTok{(}\DataTypeTok{es =} \StringTok{"pes"}\NormalTok{))}

\NormalTok{knitr}\OperatorTok{::}\KeywordTok{kable}\NormalTok{(}\KeywordTok{nice}\NormalTok{(aAv}\OperatorTok{$}\NormalTok{anova_table))}
\end{Highlighting}
\end{Shaded}

\begin{longtable}[]{@{}llllll@{}}
\toprule
Effect & df & MSE & F & pes & p.value\tabularnewline
\midrule
\endhead
Part & 1, 20 & 0.07 & 31.74 *** & .613 & \textless.001\tabularnewline
InterpenetrationFeedback & 1.97, 39.31 & 0.12 & 25.89 *** & .564 &
\textless.001\tabularnewline
Part:InterpenetrationFeedback & 2.11, 42.21 & 0.04 & 5.72 ** & .222 &
.006\tabularnewline
\bottomrule
\end{longtable}

\begin{Shaded}
\begin{Highlighting}[]
\NormalTok{effectsize}\OperatorTok{::}\KeywordTok{omega_squared}\NormalTok{(aMax, }\DataTypeTok{partial =} \OtherTok{TRUE}\NormalTok{, }\DataTypeTok{ci =} \FloatTok{0.95}\NormalTok{)}
\end{Highlighting}
\end{Shaded}

\begin{verbatim}
## Parameter                     | Omega2 (partial) |        95% CI
## ----------------------------------------------------------------
## Part                          |             0.56 | [ 0.24, 0.74]
## InterpenetrationFeedback      |             0.56 | [ 0.38, 0.68]
## Part:InterpenetrationFeedback |             0.16 | [-0.02, 0.32]
\end{verbatim}

\begin{Shaded}
\begin{Highlighting}[]
\NormalTok{effectsize}\OperatorTok{::}\KeywordTok{omega_squared}\NormalTok{(aAv, }\DataTypeTok{partial =} \OtherTok{TRUE}\NormalTok{, }\DataTypeTok{ci =} \FloatTok{0.95}\NormalTok{)}
\end{Highlighting}
\end{Shaded}

\begin{verbatim}
## Parameter                     | Omega2 (partial) |        95% CI
## ----------------------------------------------------------------
## Part                          |             0.58 | [ 0.26, 0.75]
## InterpenetrationFeedback      |             0.54 | [ 0.35, 0.66]
## Part:InterpenetrationFeedback |             0.18 | [-0.01, 0.34]
\end{verbatim}

We can see that every type of feedback as well as the interrelationship
with the part of the experiment have a large effect on DVs!!!!!!

Reference for interpreting Omega Squared Small effect: ω2 = 0.01; Medium
effect: ω2 = 0.06; Large effect: ω2 = 0.14.

Let's plot the main effects (Interpenetration Feedback OR Part of the
experiment).

\begin{Shaded}
\begin{Highlighting}[]
\CommentTok{# ANOVAs just for the plots}
\NormalTok{aMaxPlots <-}\StringTok{ }\KeywordTok{aov_ez}\NormalTok{(}\StringTok{"ID"}\NormalTok{, }\StringTok{"MaxInterpenetration"}\NormalTok{, ParsiDFplots,}
             \DataTypeTok{within =} \KeywordTok{c}\NormalTok{(}\StringTok{"Part"}\NormalTok{, }\StringTok{"InterpenetrationFeedback"}\NormalTok{),}
             \DataTypeTok{anova_table =} \KeywordTok{list}\NormalTok{(}\DataTypeTok{es =} \StringTok{"pes"}\NormalTok{))}

\NormalTok{aAvPlots <-}\StringTok{ }\KeywordTok{aov_ez}\NormalTok{(}\StringTok{"ID"}\NormalTok{, }\StringTok{"AverageInterpenetration"}\NormalTok{, ParsiDFplots,}
             \DataTypeTok{within =} \KeywordTok{c}\NormalTok{(}\StringTok{"Part"}\NormalTok{, }\StringTok{"InterpenetrationFeedback"}\NormalTok{),}
             \DataTypeTok{anova_table =} \KeywordTok{list}\NormalTok{(}\DataTypeTok{es =} \StringTok{"pes"}\NormalTok{))}

\CommentTok{#plots}
\KeywordTok{afex_plot}\NormalTok{(aMaxPlots, }\DataTypeTok{x =} \StringTok{"InterpenetrationFeedback"}\NormalTok{, }\DataTypeTok{error =} \StringTok{"within"}\NormalTok{, }
                \DataTypeTok{mapping =} \KeywordTok{c}\NormalTok{(}\StringTok{"linetype"}\NormalTok{, }\StringTok{"shape"}\NormalTok{, }\StringTok{"fill"}\NormalTok{),}
                \DataTypeTok{data_geom =}\NormalTok{ ggpol}\OperatorTok{::}\NormalTok{geom_boxjitter, }
                \DataTypeTok{data_arg =} \KeywordTok{list}\NormalTok{(}\DataTypeTok{width =} \FloatTok{0.5}\NormalTok{)) }\OperatorTok{+}
\StringTok{            }\KeywordTok{ylim}\NormalTok{(}\DecValTok{0}\NormalTok{, }\DecValTok{3}\NormalTok{)}
\end{Highlighting}
\end{Shaded}

\begin{verbatim}
## NOTE: Results may be misleading due to involvement in interactions
\end{verbatim}

\includegraphics{Report_files/figure-latex/unnamed-chunk-9-1.pdf}

\begin{Shaded}
\begin{Highlighting}[]
\KeywordTok{afex_plot}\NormalTok{(aMaxPlots, }\DataTypeTok{x =} \StringTok{"Part"}\NormalTok{, }\DataTypeTok{error =} \StringTok{"within"}\NormalTok{, }
                \DataTypeTok{mapping =} \KeywordTok{c}\NormalTok{(}\StringTok{"linetype"}\NormalTok{, }\StringTok{"shape"}\NormalTok{, }\StringTok{"fill"}\NormalTok{),}
                \DataTypeTok{data_geom =}\NormalTok{ ggpol}\OperatorTok{::}\NormalTok{geom_boxjitter, }
                \DataTypeTok{data_arg =} \KeywordTok{list}\NormalTok{(}\DataTypeTok{width =} \FloatTok{0.5}\NormalTok{))  }\OperatorTok{+}
\StringTok{            }\KeywordTok{ylim}\NormalTok{(}\DecValTok{0}\NormalTok{, }\DecValTok{3}\NormalTok{)}
\end{Highlighting}
\end{Shaded}

\begin{verbatim}
## NOTE: Results may be misleading due to involvement in interactions
\end{verbatim}

\includegraphics{Report_files/figure-latex/unnamed-chunk-9-2.pdf}

\begin{Shaded}
\begin{Highlighting}[]
\KeywordTok{afex_plot}\NormalTok{(aAvPlots, }\DataTypeTok{x =} \StringTok{"InterpenetrationFeedback"}\NormalTok{, }\DataTypeTok{error =} \StringTok{"within"}\NormalTok{, }
                \DataTypeTok{mapping =} \KeywordTok{c}\NormalTok{(}\StringTok{"linetype"}\NormalTok{, }\StringTok{"shape"}\NormalTok{, }\StringTok{"fill"}\NormalTok{),}
                \DataTypeTok{data_geom =}\NormalTok{ ggpol}\OperatorTok{::}\NormalTok{geom_boxjitter, }
                \DataTypeTok{data_arg =} \KeywordTok{list}\NormalTok{(}\DataTypeTok{width =} \FloatTok{0.5}\NormalTok{)) }\OperatorTok{+}
\StringTok{            }\KeywordTok{ylim}\NormalTok{(}\DecValTok{0}\NormalTok{, }\FloatTok{2.25}\NormalTok{)}
\end{Highlighting}
\end{Shaded}

\begin{verbatim}
## NOTE: Results may be misleading due to involvement in interactions
\end{verbatim}

\includegraphics{Report_files/figure-latex/unnamed-chunk-9-3.pdf}

\begin{Shaded}
\begin{Highlighting}[]
\KeywordTok{afex_plot}\NormalTok{(aAvPlots, }\DataTypeTok{x =} \StringTok{"Part"}\NormalTok{, }\DataTypeTok{error =} \StringTok{"within"}\NormalTok{, }
                \DataTypeTok{mapping =} \KeywordTok{c}\NormalTok{(}\StringTok{"linetype"}\NormalTok{, }\StringTok{"shape"}\NormalTok{, }\StringTok{"fill"}\NormalTok{),}
                \DataTypeTok{data_geom =}\NormalTok{ ggpol}\OperatorTok{::}\NormalTok{geom_boxjitter, }
                \DataTypeTok{data_arg =} \KeywordTok{list}\NormalTok{(}\DataTypeTok{width =} \FloatTok{0.5}\NormalTok{))  }\OperatorTok{+}
\StringTok{            }\KeywordTok{ylim}\NormalTok{(}\DecValTok{0}\NormalTok{, }\FloatTok{2.25}\NormalTok{)}
\end{Highlighting}
\end{Shaded}

\begin{verbatim}
## NOTE: Results may be misleading due to involvement in interactions
\end{verbatim}

\includegraphics{Report_files/figure-latex/unnamed-chunk-9-4.pdf}

Let's plot the main interaction effects (Interpenetration Feedback AND
Part of the experiment).

\begin{Shaded}
\begin{Highlighting}[]
\KeywordTok{afex_plot}\NormalTok{(aMaxPlots, }\DataTypeTok{x =} \StringTok{"InterpenetrationFeedback"}\NormalTok{, }\DataTypeTok{trace =} \StringTok{"Part"}\NormalTok{, }\DataTypeTok{error =} \StringTok{"within"}\NormalTok{, }
                \DataTypeTok{mapping =} \KeywordTok{c}\NormalTok{(}\StringTok{"linetype"}\NormalTok{, }\StringTok{"shape"}\NormalTok{, }\StringTok{"fill"}\NormalTok{),}
                \DataTypeTok{data_geom =}\NormalTok{ ggpol}\OperatorTok{::}\NormalTok{geom_boxjitter, }
                \DataTypeTok{data_arg =} \KeywordTok{list}\NormalTok{(}\DataTypeTok{width =} \FloatTok{0.5}\NormalTok{)) }\OperatorTok{+}
\StringTok{            }\KeywordTok{ylim}\NormalTok{(}\DecValTok{0}\NormalTok{, }\DecValTok{3}\NormalTok{)}
\end{Highlighting}
\end{Shaded}

\begin{verbatim}
## Warning: Removed 2 rows containing non-finite values (stat_box_jitter).
\end{verbatim}

\includegraphics{Report_files/figure-latex/unnamed-chunk-10-1.pdf}

\begin{Shaded}
\begin{Highlighting}[]
\KeywordTok{afex_plot}\NormalTok{(aMaxPlots, }\DataTypeTok{x =} \StringTok{"Part"}\NormalTok{, }\DataTypeTok{trace =} \StringTok{"InterpenetrationFeedback"}\NormalTok{, }\DataTypeTok{error =} \StringTok{"within"}\NormalTok{, }
                \DataTypeTok{mapping =} \KeywordTok{c}\NormalTok{(}\StringTok{"linetype"}\NormalTok{, }\StringTok{"shape"}\NormalTok{, }\StringTok{"fill"}\NormalTok{),}
                \DataTypeTok{data_geom =}\NormalTok{ ggpol}\OperatorTok{::}\NormalTok{geom_boxjitter, }
                \DataTypeTok{data_arg =} \KeywordTok{list}\NormalTok{(}\DataTypeTok{width =} \FloatTok{0.5}\NormalTok{))  }\OperatorTok{+}
\StringTok{            }\KeywordTok{ylim}\NormalTok{(}\DecValTok{0}\NormalTok{, }\DecValTok{3}\NormalTok{)}
\end{Highlighting}
\end{Shaded}

\begin{verbatim}
## Warning: Removed 2 rows containing non-finite values (stat_box_jitter).
\end{verbatim}

\includegraphics{Report_files/figure-latex/unnamed-chunk-10-2.pdf}

\begin{Shaded}
\begin{Highlighting}[]
\KeywordTok{afex_plot}\NormalTok{(aAvPlots, }\DataTypeTok{x =} \StringTok{"InterpenetrationFeedback"}\NormalTok{,  }\DataTypeTok{trace =} \StringTok{"Part"}\NormalTok{, }\DataTypeTok{error =} \StringTok{"within"}\NormalTok{, }
                \DataTypeTok{mapping =} \KeywordTok{c}\NormalTok{(}\StringTok{"linetype"}\NormalTok{, }\StringTok{"shape"}\NormalTok{, }\StringTok{"fill"}\NormalTok{),}
                \DataTypeTok{data_geom =}\NormalTok{ ggpol}\OperatorTok{::}\NormalTok{geom_boxjitter, }
                \DataTypeTok{data_arg =} \KeywordTok{list}\NormalTok{(}\DataTypeTok{width =} \FloatTok{0.5}\NormalTok{)) }\OperatorTok{+}
\StringTok{            }\KeywordTok{ylim}\NormalTok{(}\DecValTok{0}\NormalTok{, }\FloatTok{2.25}\NormalTok{)}
\end{Highlighting}
\end{Shaded}

\begin{verbatim}
## Warning: Removed 1 rows containing non-finite values (stat_box_jitter).
\end{verbatim}

\includegraphics{Report_files/figure-latex/unnamed-chunk-10-3.pdf}

\begin{Shaded}
\begin{Highlighting}[]
\KeywordTok{afex_plot}\NormalTok{(aAvPlots, }\DataTypeTok{x =} \StringTok{"Part"}\NormalTok{, }\DataTypeTok{trace =} \StringTok{"InterpenetrationFeedback"}\NormalTok{, }\DataTypeTok{error =} \StringTok{"within"}\NormalTok{, }
                \DataTypeTok{mapping =} \KeywordTok{c}\NormalTok{(}\StringTok{"linetype"}\NormalTok{, }\StringTok{"shape"}\NormalTok{, }\StringTok{"fill"}\NormalTok{),}
                \DataTypeTok{data_geom =}\NormalTok{ ggpol}\OperatorTok{::}\StringTok{ }\NormalTok{geom_boxjitter, }
                \DataTypeTok{data_arg =} \KeywordTok{list}\NormalTok{(}\DataTypeTok{width =} \FloatTok{0.5}\NormalTok{))  }\OperatorTok{+}
\StringTok{            }\KeywordTok{ylim}\NormalTok{(}\DecValTok{0}\NormalTok{, }\FloatTok{2.25}\NormalTok{)}
\end{Highlighting}
\end{Shaded}

\begin{verbatim}
## Warning: Removed 1 rows containing non-finite values (stat_box_jitter).
\end{verbatim}

\includegraphics{Report_files/figure-latex/unnamed-chunk-10-4.pdf}

Post-hoc Tests

\begin{Shaded}
\begin{Highlighting}[]
\StringTok{"}
\StringTok{# Main Effects}
\StringTok{lsmMaxFeedback <- lsmeans(aMax,~ InterpenetrationFeedback)}
\StringTok{contrast(lsmMax, method=pairwise, interaction = TRUE)}

\StringTok{lsmMaxPart <- lsmeans(aMax,~ Part)}
\StringTok{contrast(lsmMax, method=pairwise, interaction = TRUE)}

\StringTok{lsmMaxIntEff <- lsmeans(aMax,~ InterpenetrationFeedback:Part)}
\StringTok{contrast(lsmMax, method=pairwise, interaction = TRUE)}

\StringTok{lsmAvFeedback <- lsmeans(aAv,~ InterpenetrationFeedback)}
\StringTok{contrast(lsmAvIntEff, method=pairwise, interaction = TRUE)}

\StringTok{lsmAvPart <- lsmeans(aAv,~ Part)}
\StringTok{contrast(lsmAvIntEff, method=pairwise, interaction = TRUE)}

\StringTok{lsmAvIntEff <- lsmeans(aAv,~ InterpenetrationFeedback:Part)}
\StringTok{contrast(lsmAvIntEff, method=pairwise, interaction = TRUE)}
\StringTok{"}
\end{Highlighting}
\end{Shaded}

\begin{verbatim}
## [1] "\n# Main Effects\nlsmMaxFeedback <- lsmeans(aMax,~ InterpenetrationFeedback)\ncontrast(lsmMax, method=pairwise, interaction = TRUE)\n\nlsmMaxPart <- lsmeans(aMax,~ Part)\ncontrast(lsmMax, method=pairwise, interaction = TRUE)\n\nlsmMaxIntEff <- lsmeans(aMax,~ InterpenetrationFeedback:Part)\ncontrast(lsmMax, method=pairwise, interaction = TRUE)\n\nlsmAvFeedback <- lsmeans(aAv,~ InterpenetrationFeedback)\ncontrast(lsmAvIntEff, method=pairwise, interaction = TRUE)\n\nlsmAvPart <- lsmeans(aAv,~ Part)\ncontrast(lsmAvIntEff, method=pairwise, interaction = TRUE)\n\nlsmAvIntEff <- lsmeans(aAv,~ InterpenetrationFeedback:Part)\ncontrast(lsmAvIntEff, method=pairwise, interaction = TRUE)\n"
\end{verbatim}

\begin{Shaded}
\begin{Highlighting}[]
\CommentTok{#lsmMaxIntEff <- lsmeans(aMax,~ InterpenetrationFeedback:Part)}
\CommentTok{#contrast(lsmMaxIntEff, method= "pairwise", interaction = TRUE)}

\CommentTok{#lsmAvIntEff <- lsmeans(aAv,~ InterpenetrationFeedback:Part)}
\CommentTok{#contrast(lsmAvIntEff, method= "pairwise", interaction = TRUE)}

\CommentTok{#?lsmeans()}
\CommentTok{# Only Interaction Effects}

\KeywordTok{contrast}\NormalTok{(}\KeywordTok{emmeans}\NormalTok{(aMax,}\OperatorTok{~}\StringTok{ }\NormalTok{Part}\OperatorTok{:}\NormalTok{InterpenetrationFeedback), }
         \DataTypeTok{method=}\StringTok{"pairwise"}\NormalTok{, }\DataTypeTok{interaction=}\OtherTok{TRUE}\NormalTok{)}
\end{Highlighting}
\end{Shaded}

\begin{verbatim}
##  Part_pairwise   InterpenetrationFeedback_pairwise estimate     SE df t.ratio
##  Part.1 - Part.2 Both - Electrotactile              -0.1201 0.0718 60 -1.672 
##  Part.1 - Part.2 Both - NoFeedback                   0.1262 0.0718 60  1.757 
##  Part.1 - Part.2 Both - Visual                       0.1153 0.0718 60  1.605 
##  Part.1 - Part.2 Electrotactile - NoFeedback         0.2463 0.0718 60  3.429 
##  Part.1 - Part.2 Electrotactile - Visual             0.2354 0.0718 60  3.277 
##  Part.1 - Part.2 NoFeedback - Visual                -0.0109 0.0718 60 -0.152 
##  p.value
##  0.0997 
##  0.0841 
##  0.1137 
##  0.0011 
##  0.0017 
##  0.8800
\end{verbatim}

\begin{Shaded}
\begin{Highlighting}[]
\KeywordTok{contrast}\NormalTok{(}\KeywordTok{emmeans}\NormalTok{(aAv,}\OperatorTok{~}\StringTok{ }\NormalTok{Part}\OperatorTok{:}\NormalTok{InterpenetrationFeedback), }
         \DataTypeTok{method=}\StringTok{"pairwise"}\NormalTok{, }\DataTypeTok{interaction=}\OtherTok{TRUE}\NormalTok{)}
\end{Highlighting}
\end{Shaded}

\begin{verbatim}
##  Part_pairwise   InterpenetrationFeedback_pairwise estimate     SE df t.ratio
##  Part.1 - Part.2 Both - Electrotactile              -0.0928 0.0738 60 -1.258 
##  Part.1 - Part.2 Both - NoFeedback                   0.1674 0.0738 60  2.268 
##  Part.1 - Part.2 Both - Visual                       0.1509 0.0738 60  2.046 
##  Part.1 - Part.2 Electrotactile - NoFeedback         0.2602 0.0738 60  3.527 
##  Part.1 - Part.2 Electrotactile - Visual             0.2437 0.0738 60  3.304 
##  Part.1 - Part.2 NoFeedback - Visual                -0.0164 0.0738 60 -0.223 
##  p.value
##  0.2132 
##  0.0269 
##  0.0452 
##  0.0008 
##  0.0016 
##  0.8243
\end{verbatim}

The above compares the effect size of the chance of the types of
interpenetration feedback between part 1 and part 2 of the experiment!

we can see that for MAX INTERPENETRATION the significant comparisons
are: Estimate SE df t.ratio p.value 1) Part.1 - Part.2 Electrotactile -
NoFeedback 0.2463 0.0718 60 3.429 0.0011

\begin{enumerate}
\def\labelenumi{\arabic{enumi})}
\setcounter{enumi}{1}
\tightlist
\item
  Part.1 - Part.2 Electrotactile - Visual 0.2354 0.0718 60 3.277 0.0017
\end{enumerate}

This means that the improvement of the performance from part 1 to part 2
regarding the maximum interpenetration was significantly greater for the
``Electrotactile Feedback'' compared to the ``No Feedback'' and ``Visual
Feedback'' respectively! The rest of the comparisons were insignificant!

For the AVERAGE INTERPENETRATION the significant comparisons are:
Estimate SE df t.ratio p.value 1) Part.1 - Part.2 Both - NoFeedback
0.1674 0.0738 60 2.268 0.0269

\begin{enumerate}
\def\labelenumi{\arabic{enumi})}
\setcounter{enumi}{1}
\item
  Part.1 - Part.2 Both - Visual 0.1509 0.0738 60 2.046 0.0452
\item
  Part.1 - Part.2 Electrotactile - NoFeedback 0.2602 0.0738 60 3.527
  0.0008
\item
  Part.1 - Part.2 Electrotactile - Visual 0.2437 0.0738 60 3.304 0.0016
\end{enumerate}

This means that the improvement of the performance from part 1 to part 2
regarding the average interpenetration was significantly greater for the
``Electrotactile Feedback'' compared to the ``No Feedback'' and ``Visual
Feedback'' respectively!

Also, that the improvement of the performance from part 1 to part 2
regarding the average interpenetration was significantly greater for the
``Combined (i.e., Both) Feedback'' compared to the ``No Feedback'' and
``Visual Feedback'' respectively! The rest of the comparisons were
insignificant!

\end{document}
