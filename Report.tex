% Options for packages loaded elsewhere
\PassOptionsToPackage{unicode}{hyperref}
\PassOptionsToPackage{hyphens}{url}
%
\documentclass[
]{article}
\usepackage{lmodern}
\usepackage{amssymb,amsmath}
\usepackage{ifxetex,ifluatex}
\ifnum 0\ifxetex 1\fi\ifluatex 1\fi=0 % if pdftex
  \usepackage[T1]{fontenc}
  \usepackage[utf8]{inputenc}
  \usepackage{textcomp} % provide euro and other symbols
\else % if luatex or xetex
  \usepackage{unicode-math}
  \defaultfontfeatures{Scale=MatchLowercase}
  \defaultfontfeatures[\rmfamily]{Ligatures=TeX,Scale=1}
\fi
% Use upquote if available, for straight quotes in verbatim environments
\IfFileExists{upquote.sty}{\usepackage{upquote}}{}
\IfFileExists{microtype.sty}{% use microtype if available
  \usepackage[]{microtype}
  \UseMicrotypeSet[protrusion]{basicmath} % disable protrusion for tt fonts
}{}
\makeatletter
\@ifundefined{KOMAClassName}{% if non-KOMA class
  \IfFileExists{parskip.sty}{%
    \usepackage{parskip}
  }{% else
    \setlength{\parindent}{0pt}
    \setlength{\parskip}{6pt plus 2pt minus 1pt}}
}{% if KOMA class
  \KOMAoptions{parskip=half}}
\makeatother
\usepackage{xcolor}
\IfFileExists{xurl.sty}{\usepackage{xurl}}{} % add URL line breaks if available
\IfFileExists{bookmark.sty}{\usepackage{bookmark}}{\usepackage{hyperref}}
\hypersetup{
  pdftitle={Contact Experiment Data Analysis},
  hidelinks,
  pdfcreator={LaTeX via pandoc}}
\urlstyle{same} % disable monospaced font for URLs
\usepackage[margin=1in]{geometry}
\usepackage{color}
\usepackage{fancyvrb}
\newcommand{\VerbBar}{|}
\newcommand{\VERB}{\Verb[commandchars=\\\{\}]}
\DefineVerbatimEnvironment{Highlighting}{Verbatim}{commandchars=\\\{\}}
% Add ',fontsize=\small' for more characters per line
\usepackage{framed}
\definecolor{shadecolor}{RGB}{248,248,248}
\newenvironment{Shaded}{\begin{snugshade}}{\end{snugshade}}
\newcommand{\AlertTok}[1]{\textcolor[rgb]{0.94,0.16,0.16}{#1}}
\newcommand{\AnnotationTok}[1]{\textcolor[rgb]{0.56,0.35,0.01}{\textbf{\textit{#1}}}}
\newcommand{\AttributeTok}[1]{\textcolor[rgb]{0.77,0.63,0.00}{#1}}
\newcommand{\BaseNTok}[1]{\textcolor[rgb]{0.00,0.00,0.81}{#1}}
\newcommand{\BuiltInTok}[1]{#1}
\newcommand{\CharTok}[1]{\textcolor[rgb]{0.31,0.60,0.02}{#1}}
\newcommand{\CommentTok}[1]{\textcolor[rgb]{0.56,0.35,0.01}{\textit{#1}}}
\newcommand{\CommentVarTok}[1]{\textcolor[rgb]{0.56,0.35,0.01}{\textbf{\textit{#1}}}}
\newcommand{\ConstantTok}[1]{\textcolor[rgb]{0.00,0.00,0.00}{#1}}
\newcommand{\ControlFlowTok}[1]{\textcolor[rgb]{0.13,0.29,0.53}{\textbf{#1}}}
\newcommand{\DataTypeTok}[1]{\textcolor[rgb]{0.13,0.29,0.53}{#1}}
\newcommand{\DecValTok}[1]{\textcolor[rgb]{0.00,0.00,0.81}{#1}}
\newcommand{\DocumentationTok}[1]{\textcolor[rgb]{0.56,0.35,0.01}{\textbf{\textit{#1}}}}
\newcommand{\ErrorTok}[1]{\textcolor[rgb]{0.64,0.00,0.00}{\textbf{#1}}}
\newcommand{\ExtensionTok}[1]{#1}
\newcommand{\FloatTok}[1]{\textcolor[rgb]{0.00,0.00,0.81}{#1}}
\newcommand{\FunctionTok}[1]{\textcolor[rgb]{0.00,0.00,0.00}{#1}}
\newcommand{\ImportTok}[1]{#1}
\newcommand{\InformationTok}[1]{\textcolor[rgb]{0.56,0.35,0.01}{\textbf{\textit{#1}}}}
\newcommand{\KeywordTok}[1]{\textcolor[rgb]{0.13,0.29,0.53}{\textbf{#1}}}
\newcommand{\NormalTok}[1]{#1}
\newcommand{\OperatorTok}[1]{\textcolor[rgb]{0.81,0.36,0.00}{\textbf{#1}}}
\newcommand{\OtherTok}[1]{\textcolor[rgb]{0.56,0.35,0.01}{#1}}
\newcommand{\PreprocessorTok}[1]{\textcolor[rgb]{0.56,0.35,0.01}{\textit{#1}}}
\newcommand{\RegionMarkerTok}[1]{#1}
\newcommand{\SpecialCharTok}[1]{\textcolor[rgb]{0.00,0.00,0.00}{#1}}
\newcommand{\SpecialStringTok}[1]{\textcolor[rgb]{0.31,0.60,0.02}{#1}}
\newcommand{\StringTok}[1]{\textcolor[rgb]{0.31,0.60,0.02}{#1}}
\newcommand{\VariableTok}[1]{\textcolor[rgb]{0.00,0.00,0.00}{#1}}
\newcommand{\VerbatimStringTok}[1]{\textcolor[rgb]{0.31,0.60,0.02}{#1}}
\newcommand{\WarningTok}[1]{\textcolor[rgb]{0.56,0.35,0.01}{\textbf{\textit{#1}}}}
\usepackage{longtable,booktabs}
% Correct order of tables after \paragraph or \subparagraph
\usepackage{etoolbox}
\makeatletter
\patchcmd\longtable{\par}{\if@noskipsec\mbox{}\fi\par}{}{}
\makeatother
% Allow footnotes in longtable head/foot
\IfFileExists{footnotehyper.sty}{\usepackage{footnotehyper}}{\usepackage{footnote}}
\makesavenoteenv{longtable}
\usepackage{graphicx,grffile}
\makeatletter
\def\maxwidth{\ifdim\Gin@nat@width>\linewidth\linewidth\else\Gin@nat@width\fi}
\def\maxheight{\ifdim\Gin@nat@height>\textheight\textheight\else\Gin@nat@height\fi}
\makeatother
% Scale images if necessary, so that they will not overflow the page
% margins by default, and it is still possible to overwrite the defaults
% using explicit options in \includegraphics[width, height, ...]{}
\setkeys{Gin}{width=\maxwidth,height=\maxheight,keepaspectratio}
% Set default figure placement to htbp
\makeatletter
\def\fps@figure{htbp}
\makeatother
\setlength{\emergencystretch}{3em} % prevent overfull lines
\providecommand{\tightlist}{%
  \setlength{\itemsep}{0pt}\setlength{\parskip}{0pt}}
\setcounter{secnumdepth}{-\maxdimen} % remove section numbering

\title{Contact Experiment Data Analysis}
\author{}
\date{\vspace{-2.5em}}

\begin{document}
\maketitle

\#First Step is to load the necessary package, If you dont have them
just install them. For jjstatsplot you need to install it remotely.
\#Just remove the dash and press enter. Then Press 3 (none package to be
updated).

\begin{Shaded}
\begin{Highlighting}[]
\CommentTok{#remotes::install_github("sbalci/jjstatsplot") #Press 3 !!!!! i.e., installing/Updating none package! }
\KeywordTok{library}\NormalTok{(psych)}
\KeywordTok{library}\NormalTok{(jmv)}
\end{Highlighting}
\end{Shaded}

\begin{verbatim}
## 
## Attaching package: 'jmv'
\end{verbatim}

\begin{verbatim}
## The following object is masked from 'package:psych':
## 
##     pca
\end{verbatim}

\begin{Shaded}
\begin{Highlighting}[]
\KeywordTok{library}\NormalTok{(datasets)}
\KeywordTok{library}\NormalTok{(plyr)}
\KeywordTok{library}\NormalTok{(readr)}
\KeywordTok{library}\NormalTok{(dataframes2xls)}
\KeywordTok{library}\NormalTok{(data.table)}
\KeywordTok{library}\NormalTok{(plyr)}
\KeywordTok{library}\NormalTok{(ggstatsplot)}
\end{Highlighting}
\end{Shaded}

\begin{verbatim}
## Registered S3 method overwritten by 'broom.mixed':
##   method      from 
##   tidy.gamlss broom
\end{verbatim}

\begin{verbatim}
## Registered S3 methods overwritten by 'lme4':
##   method                          from
##   cooks.distance.influence.merMod car 
##   influence.merMod                car 
##   dfbeta.influence.merMod         car 
##   dfbetas.influence.merMod        car
\end{verbatim}

\begin{verbatim}
## In case you would like cite this package, cite it as:
##      Patil, I. (2018). ggstatsplot: "ggplot2" Based Plots with Statistical Details. CRAN.
##      Retrieved from https://cran.r-project.org/web/packages/ggstatsplot/index.html
\end{verbatim}

\begin{Shaded}
\begin{Highlighting}[]
\KeywordTok{library}\NormalTok{(jjstatsplot)}
\KeywordTok{library}\NormalTok{(lme4)}
\end{Highlighting}
\end{Shaded}

\begin{verbatim}
## Loading required package: Matrix
\end{verbatim}

\begin{Shaded}
\begin{Highlighting}[]
\KeywordTok{library}\NormalTok{(lmerTest)}
\end{Highlighting}
\end{Shaded}

\begin{verbatim}
## 
## Attaching package: 'lmerTest'
\end{verbatim}

\begin{verbatim}
## The following object is masked from 'package:lme4':
## 
##     lmer
\end{verbatim}

\begin{verbatim}
## The following object is masked from 'package:stats':
## 
##     step
\end{verbatim}

\begin{Shaded}
\begin{Highlighting}[]
\KeywordTok{library}\NormalTok{(ggplot2)}
\end{Highlighting}
\end{Shaded}

\begin{verbatim}
## 
## Attaching package: 'ggplot2'
\end{verbatim}

\begin{verbatim}
## The following objects are masked from 'package:psych':
## 
##     %+%, alpha
\end{verbatim}

\begin{Shaded}
\begin{Highlighting}[]
\KeywordTok{library}\NormalTok{(rstatix)}
\end{Highlighting}
\end{Shaded}

\begin{verbatim}
## 
## Attaching package: 'rstatix'
\end{verbatim}

\begin{verbatim}
## The following objects are masked from 'package:plyr':
## 
##     desc, mutate
\end{verbatim}

\begin{verbatim}
## The following object is masked from 'package:stats':
## 
##     filter
\end{verbatim}

\begin{Shaded}
\begin{Highlighting}[]
\KeywordTok{library}\NormalTok{(coin)}
\end{Highlighting}
\end{Shaded}

\begin{verbatim}
## Loading required package: survival
\end{verbatim}

\begin{verbatim}
## 
## Attaching package: 'coin'
\end{verbatim}

\begin{verbatim}
## The following objects are masked from 'package:rstatix':
## 
##     chisq_test, friedman_test, kruskal_test, sign_test, wilcox_test
\end{verbatim}

\begin{Shaded}
\begin{Highlighting}[]
\KeywordTok{library}\NormalTok{(ARTool)}
\KeywordTok{library}\NormalTok{(ggpubr)}
\end{Highlighting}
\end{Shaded}

\begin{verbatim}
## 
## Attaching package: 'ggpubr'
\end{verbatim}

\begin{verbatim}
## The following object is masked from 'package:plyr':
## 
##     mutate
\end{verbatim}

\begin{Shaded}
\begin{Highlighting}[]
\KeywordTok{library}\NormalTok{(tidyverse)}
\end{Highlighting}
\end{Shaded}

\begin{verbatim}
## Found more than one class "atomicVector" in cache; using the first, from namespace 'Matrix'
\end{verbatim}

\begin{verbatim}
## Also defined by 'Rmpfr'
\end{verbatim}

\begin{verbatim}
## Found more than one class "atomicVector" in cache; using the first, from namespace 'Matrix'
\end{verbatim}

\begin{verbatim}
## Also defined by 'Rmpfr'
\end{verbatim}

\begin{verbatim}
## Found more than one class "atomicVector" in cache; using the first, from namespace 'Matrix'
\end{verbatim}

\begin{verbatim}
## Also defined by 'Rmpfr'
\end{verbatim}

\begin{verbatim}
## Found more than one class "atomicVector" in cache; using the first, from namespace 'Matrix'
\end{verbatim}

\begin{verbatim}
## Also defined by 'Rmpfr'
\end{verbatim}

\begin{verbatim}
## Found more than one class "atomicVector" in cache; using the first, from namespace 'Matrix'
\end{verbatim}

\begin{verbatim}
## Also defined by 'Rmpfr'
\end{verbatim}

\begin{verbatim}
## Found more than one class "atomicVector" in cache; using the first, from namespace 'Matrix'
\end{verbatim}

\begin{verbatim}
## Also defined by 'Rmpfr'
\end{verbatim}

\begin{verbatim}
## Found more than one class "atomicVector" in cache; using the first, from namespace 'Matrix'
\end{verbatim}

\begin{verbatim}
## Also defined by 'Rmpfr'
\end{verbatim}

\begin{verbatim}
## Found more than one class "atomicVector" in cache; using the first, from namespace 'Matrix'
\end{verbatim}

\begin{verbatim}
## Also defined by 'Rmpfr'
\end{verbatim}

\begin{verbatim}
## Found more than one class "atomicVector" in cache; using the first, from namespace 'Matrix'
\end{verbatim}

\begin{verbatim}
## Also defined by 'Rmpfr'
\end{verbatim}

\begin{verbatim}
## Found more than one class "atomicVector" in cache; using the first, from namespace 'Matrix'
\end{verbatim}

\begin{verbatim}
## Also defined by 'Rmpfr'
\end{verbatim}

\begin{verbatim}
## Found more than one class "atomicVector" in cache; using the first, from namespace 'Matrix'
\end{verbatim}

\begin{verbatim}
## Also defined by 'Rmpfr'
\end{verbatim}

\begin{verbatim}
## Found more than one class "atomicVector" in cache; using the first, from namespace 'Matrix'
\end{verbatim}

\begin{verbatim}
## Also defined by 'Rmpfr'
\end{verbatim}

\begin{verbatim}
## -- Attaching packages --------------------------------------- tidyverse 1.3.0 --
\end{verbatim}

\begin{verbatim}
## v tibble  3.0.4     v dplyr   1.0.2
## v tidyr   1.1.2     v stringr 1.4.0
## v purrr   0.3.4     v forcats 0.5.0
\end{verbatim}

\begin{verbatim}
## -- Conflicts ------------------------------------------ tidyverse_conflicts() --
## x ggplot2::%+%()     masks psych::%+%()
## x coin::alpha()      masks ggplot2::alpha(), psych::alpha()
## x dplyr::arrange()   masks plyr::arrange()
## x dplyr::between()   masks data.table::between()
## x purrr::compact()   masks plyr::compact()
## x dplyr::count()     masks plyr::count()
## x tidyr::expand()    masks Matrix::expand()
## x dplyr::failwith()  masks plyr::failwith()
## x dplyr::filter()    masks rstatix::filter(), stats::filter()
## x dplyr::first()     masks data.table::first()
## x dplyr::id()        masks plyr::id()
## x dplyr::lag()       masks stats::lag()
## x dplyr::last()      masks data.table::last()
## x dplyr::mutate()    masks ggpubr::mutate(), rstatix::mutate(), plyr::mutate()
## x tidyr::pack()      masks Matrix::pack()
## x dplyr::rename()    masks plyr::rename()
## x dplyr::summarise() masks plyr::summarise()
## x dplyr::summarize() masks plyr::summarize()
## x purrr::transpose() masks data.table::transpose()
## x tidyr::unpack()    masks Matrix::unpack()
\end{verbatim}

\begin{Shaded}
\begin{Highlighting}[]
\KeywordTok{library}\NormalTok{(dplyr)}
\KeywordTok{library}\NormalTok{(}\StringTok{"afex"}\NormalTok{)     }
\end{Highlighting}
\end{Shaded}

\begin{verbatim}
## ************
## Welcome to afex. For support visit: http://afex.singmann.science/
\end{verbatim}

\begin{verbatim}
## - Functions for ANOVAs: aov_car(), aov_ez(), and aov_4()
## - Methods for calculating p-values with mixed(): 'KR', 'S', 'LRT', and 'PB'
## - 'afex_aov' and 'mixed' objects can be passed to emmeans() for follow-up tests
## - NEWS: library('emmeans') now needs to be called explicitly!
## - Get and set global package options with: afex_options()
## - Set orthogonal sum-to-zero contrasts globally: set_sum_contrasts()
## - For example analyses see: browseVignettes("afex")
## ************
\end{verbatim}

\begin{verbatim}
## 
## Attaching package: 'afex'
\end{verbatim}

\begin{verbatim}
## The following object is masked from 'package:lme4':
## 
##     lmer
\end{verbatim}

\begin{Shaded}
\begin{Highlighting}[]
\KeywordTok{library}\NormalTok{(}\StringTok{"emmeans"}\NormalTok{)  }
\KeywordTok{library}\NormalTok{(}\StringTok{"multcomp"}\NormalTok{) }
\end{Highlighting}
\end{Shaded}

\begin{verbatim}
## Loading required package: mvtnorm
\end{verbatim}

\begin{verbatim}
## Loading required package: TH.data
\end{verbatim}

\begin{verbatim}
## Loading required package: MASS
\end{verbatim}

\begin{verbatim}
## 
## Attaching package: 'MASS'
\end{verbatim}

\begin{verbatim}
## The following object is masked from 'package:dplyr':
## 
##     select
\end{verbatim}

\begin{verbatim}
## The following object is masked from 'package:rstatix':
## 
##     select
\end{verbatim}

\begin{verbatim}
## 
## Attaching package: 'TH.data'
\end{verbatim}

\begin{verbatim}
## The following object is masked from 'package:MASS':
## 
##     geyser
\end{verbatim}

\begin{Shaded}
\begin{Highlighting}[]
\KeywordTok{library}\NormalTok{(tinytex)}
\KeywordTok{library}\NormalTok{(rsconnect)}
\KeywordTok{library}\NormalTok{(shiny)}
\end{Highlighting}
\end{Shaded}

\begin{verbatim}
## 
## Attaching package: 'shiny'
\end{verbatim}

\begin{verbatim}
## The following object is masked from 'package:rsconnect':
## 
##     serverInfo
\end{verbatim}

\begin{Shaded}
\begin{Highlighting}[]
\KeywordTok{library}\NormalTok{(meta)}
\end{Highlighting}
\end{Shaded}

\begin{verbatim}
## Loading 'meta' package (version 4.15-1).
## Type 'help(meta)' for a brief overview.
\end{verbatim}

\begin{Shaded}
\begin{Highlighting}[]
\KeywordTok{library}\NormalTok{(DescTools)}
\end{Highlighting}
\end{Shaded}

\begin{verbatim}
## 
## Attaching package: 'DescTools'
\end{verbatim}

\begin{verbatim}
## The following object is masked from 'package:data.table':
## 
##     %like%
\end{verbatim}

\begin{verbatim}
## The following objects are masked from 'package:psych':
## 
##     AUC, ICC, SD
\end{verbatim}

\begin{Shaded}
\begin{Highlighting}[]
\KeywordTok{library}\NormalTok{(dplyr)}
\end{Highlighting}
\end{Shaded}

\#Import and merge all the csv files (x.csv where x = ID) of the folder.
Note that the ID is has already been added as column

\begin{Shaded}
\begin{Highlighting}[]
\NormalTok{data <-}\StringTok{ }\KeywordTok{list.files}\NormalTok{(}\DataTypeTok{path =} \StringTok{"C:/Users/pkourtes/Desktop/ContactExperiment/Results/Participants/MergingFolder"}\NormalTok{,     }\CommentTok{# Identify all csv files in folder}
                       \DataTypeTok{pattern =} \StringTok{"*.csv"}\NormalTok{, }\DataTypeTok{full.names =} \OtherTok{TRUE}\NormalTok{) }\OperatorTok\StringTok{ }
\StringTok{  }\KeywordTok{lapply}\NormalTok{(read_csv) }\OperatorTok\StringTok{                                            }\CommentTok{# Store all files in list}
\StringTok{  }\NormalTok{bind_rows                                                       }\CommentTok{# Combine data sets into a single data set }
\end{Highlighting}
\end{Shaded}

\begin{verbatim}
## 
## -- Column specification --------------------------------------------------------
## cols(
##   .default = col_double(),
##   InterpenetrationFeedback = col_character(),
##   FullyShaded = col_logical(),
##   Completed = col_logical(),
##   Skipped = col_logical(),
##   Gender = col_character()
## )
## i Use `spec()` for the full column specifications.
## 
## 
## -- Column specification --------------------------------------------------------
## cols(
##   .default = col_double(),
##   InterpenetrationFeedback = col_character(),
##   FullyShaded = col_logical(),
##   Completed = col_logical(),
##   Skipped = col_logical(),
##   Gender = col_character()
## )
## i Use `spec()` for the full column specifications.
## 
## 
## -- Column specification --------------------------------------------------------
## cols(
##   .default = col_double(),
##   InterpenetrationFeedback = col_character(),
##   FullyShaded = col_logical(),
##   Completed = col_logical(),
##   Skipped = col_logical(),
##   Gender = col_character()
## )
## i Use `spec()` for the full column specifications.
## 
## 
## -- Column specification --------------------------------------------------------
## cols(
##   .default = col_double(),
##   InterpenetrationFeedback = col_character(),
##   FullyShaded = col_logical(),
##   Completed = col_logical(),
##   Skipped = col_logical(),
##   Gender = col_character()
## )
## i Use `spec()` for the full column specifications.
## 
## 
## -- Column specification --------------------------------------------------------
## cols(
##   .default = col_double(),
##   InterpenetrationFeedback = col_character(),
##   FullyShaded = col_logical(),
##   Completed = col_logical(),
##   Skipped = col_logical(),
##   Gender = col_character()
## )
## i Use `spec()` for the full column specifications.
## 
## 
## -- Column specification --------------------------------------------------------
## cols(
##   .default = col_double(),
##   InterpenetrationFeedback = col_character(),
##   FullyShaded = col_logical(),
##   Completed = col_logical(),
##   Skipped = col_logical(),
##   Gender = col_character()
## )
## i Use `spec()` for the full column specifications.
## 
## 
## -- Column specification --------------------------------------------------------
## cols(
##   .default = col_double(),
##   InterpenetrationFeedback = col_character(),
##   FullyShaded = col_logical(),
##   Completed = col_logical(),
##   Skipped = col_logical(),
##   Gender = col_character()
## )
## i Use `spec()` for the full column specifications.
## 
## 
## -- Column specification --------------------------------------------------------
## cols(
##   .default = col_double(),
##   InterpenetrationFeedback = col_character(),
##   FullyShaded = col_logical(),
##   Completed = col_logical(),
##   Skipped = col_logical(),
##   Gender = col_character()
## )
## i Use `spec()` for the full column specifications.
## 
## 
## -- Column specification --------------------------------------------------------
## cols(
##   .default = col_double(),
##   InterpenetrationFeedback = col_character(),
##   FullyShaded = col_logical(),
##   Completed = col_logical(),
##   Skipped = col_logical(),
##   Gender = col_character()
## )
## i Use `spec()` for the full column specifications.
## 
## 
## -- Column specification --------------------------------------------------------
## cols(
##   .default = col_double(),
##   InterpenetrationFeedback = col_character(),
##   FullyShaded = col_logical(),
##   Completed = col_logical(),
##   Skipped = col_logical(),
##   Gender = col_character()
## )
## i Use `spec()` for the full column specifications.
## 
## 
## -- Column specification --------------------------------------------------------
## cols(
##   .default = col_double(),
##   InterpenetrationFeedback = col_character(),
##   FullyShaded = col_logical(),
##   Completed = col_logical(),
##   Skipped = col_logical(),
##   Gender = col_character()
## )
## i Use `spec()` for the full column specifications.
## 
## 
## -- Column specification --------------------------------------------------------
## cols(
##   .default = col_double(),
##   InterpenetrationFeedback = col_character(),
##   FullyShaded = col_logical(),
##   Completed = col_logical(),
##   Skipped = col_logical(),
##   Gender = col_character()
## )
## i Use `spec()` for the full column specifications.
## 
## 
## -- Column specification --------------------------------------------------------
## cols(
##   .default = col_double(),
##   InterpenetrationFeedback = col_character(),
##   FullyShaded = col_logical(),
##   Completed = col_logical(),
##   Skipped = col_logical(),
##   Gender = col_character()
## )
## i Use `spec()` for the full column specifications.
## 
## 
## -- Column specification --------------------------------------------------------
## cols(
##   .default = col_double(),
##   InterpenetrationFeedback = col_character(),
##   FullyShaded = col_logical(),
##   Completed = col_logical(),
##   Skipped = col_logical(),
##   Gender = col_character()
## )
## i Use `spec()` for the full column specifications.
## 
## 
## -- Column specification --------------------------------------------------------
## cols(
##   .default = col_double(),
##   InterpenetrationFeedback = col_character(),
##   FullyShaded = col_logical(),
##   Completed = col_logical(),
##   Skipped = col_logical(),
##   Gender = col_character()
## )
## i Use `spec()` for the full column specifications.
## 
## 
## -- Column specification --------------------------------------------------------
## cols(
##   .default = col_double(),
##   InterpenetrationFeedback = col_character(),
##   FullyShaded = col_logical(),
##   Completed = col_logical(),
##   Skipped = col_logical(),
##   Gender = col_character()
## )
## i Use `spec()` for the full column specifications.
## 
## 
## -- Column specification --------------------------------------------------------
## cols(
##   .default = col_double(),
##   InterpenetrationFeedback = col_character(),
##   FullyShaded = col_logical(),
##   Completed = col_logical(),
##   Skipped = col_logical(),
##   Gender = col_character()
## )
## i Use `spec()` for the full column specifications.
## 
## 
## -- Column specification --------------------------------------------------------
## cols(
##   .default = col_double(),
##   InterpenetrationFeedback = col_character(),
##   FullyShaded = col_logical(),
##   Completed = col_logical(),
##   Skipped = col_logical(),
##   Gender = col_character()
## )
## i Use `spec()` for the full column specifications.
## 
## 
## -- Column specification --------------------------------------------------------
## cols(
##   .default = col_double(),
##   InterpenetrationFeedback = col_character(),
##   FullyShaded = col_logical(),
##   Completed = col_logical(),
##   Skipped = col_logical(),
##   Gender = col_character()
## )
## i Use `spec()` for the full column specifications.
## 
## 
## -- Column specification --------------------------------------------------------
## cols(
##   .default = col_double(),
##   InterpenetrationFeedback = col_character(),
##   FullyShaded = col_logical(),
##   Completed = col_logical(),
##   Skipped = col_logical(),
##   Gender = col_character()
## )
## i Use `spec()` for the full column specifications.
## 
## 
## -- Column specification --------------------------------------------------------
## cols(
##   .default = col_double(),
##   InterpenetrationFeedback = col_character(),
##   FullyShaded = col_logical(),
##   Completed = col_logical(),
##   Skipped = col_logical(),
##   Gender = col_character()
## )
## i Use `spec()` for the full column specifications.
## 
## 
## -- Column specification --------------------------------------------------------
## cols(
##   .default = col_double(),
##   InterpenetrationFeedback = col_character(),
##   FullyShaded = col_logical(),
##   Completed = col_logical(),
##   Skipped = col_logical(),
##   Gender = col_character()
## )
## i Use `spec()` for the full column specifications.
## 
## 
## -- Column specification --------------------------------------------------------
## cols(
##   .default = col_double(),
##   InterpenetrationFeedback = col_character(),
##   FullyShaded = col_logical(),
##   Completed = col_logical(),
##   Skipped = col_logical(),
##   Gender = col_character()
## )
## i Use `spec()` for the full column specifications.
## 
## 
## -- Column specification --------------------------------------------------------
## cols(
##   .default = col_double(),
##   InterpenetrationFeedback = col_character(),
##   FullyShaded = col_logical(),
##   Completed = col_logical(),
##   Skipped = col_logical(),
##   Gender = col_character()
## )
## i Use `spec()` for the full column specifications.
\end{verbatim}

\begin{Shaded}
\begin{Highlighting}[]
\NormalTok{data }
\end{Highlighting}
\end{Shaded}

\begin{verbatim}
## # A tibble: 2,304 x 20
##       ID Block TrialInBlock Trial Interpenetratio~ FullyShaded NrOfAdjustments
##    <dbl> <dbl>        <dbl> <dbl> <chr>            <lgl>                 <dbl>
##  1     1     0            0     0 NoFeedback       FALSE                     1
##  2     1     0            1     1 NoFeedback       TRUE                      0
##  3     1     0            2     2 NoFeedback       TRUE                      0
##  4     1     0            3     3 NoFeedback       FALSE                     0
##  5     1     0            4     4 NoFeedback       TRUE                      0
##  6     1     0            5     5 NoFeedback       FALSE                     0
##  7     1     0            6     6 NoFeedback       FALSE                     0
##  8     1     0            7     7 NoFeedback       FALSE                     0
##  9     1     0            8     8 NoFeedback       TRUE                      0
## 10     1     0            9     9 NoFeedback       TRUE                      0
## # ... with 2,294 more rows, and 13 more variables: InterpenetrationTime <dbl>,
## #   TrialTime <dbl>, FirstContactTime <dbl>, MaxInterpenetration <dbl>,
## #   AverageInterpenetration <dbl>, AverageOffsetFromSurface <dbl>,
## #   ReleaseReactionTime <dbl>, Completed <lgl>, Attempts <dbl>, Skipped <lgl>,
## #   RespTime <dbl>, Age <dbl>, Gender <chr>
\end{verbatim}

Discard first X trials per interpenetration feedback condition and then
create a summary table for each participant. You need to define
\textbf{nrTrialsPerBlockToRemove}.

\begin{Shaded}
\begin{Highlighting}[]
\NormalTok{data}\OperatorTok{$}\NormalTok{Part <-}\StringTok{ }\KeywordTok{as.factor}\NormalTok{(data}\OperatorTok{$}\NormalTok{Block }\OperatorTok{<}\StringTok{ }\DecValTok{4}\NormalTok{)}
\KeywordTok{levels}\NormalTok{(data}\OperatorTok{$}\NormalTok{Part)}
\end{Highlighting}
\end{Shaded}

\begin{verbatim}
## [1] "FALSE" "TRUE"
\end{verbatim}

\begin{Shaded}
\begin{Highlighting}[]
\NormalTok{data}\OperatorTok{$}\NormalTok{Part <-}\StringTok{ }\KeywordTok{factor}\NormalTok{(data}\OperatorTok{$}\NormalTok{Part,}\DataTypeTok{levels =} \KeywordTok{c}\NormalTok{(}\StringTok{"TRUE"}\NormalTok{,}\StringTok{"FALSE"}\NormalTok{),}
                  \DataTypeTok{labels =} \KeywordTok{c}\NormalTok{(}\StringTok{"Part 1"}\NormalTok{,}\StringTok{"Part 2"}\NormalTok{))}

\NormalTok{nrTrialsPerBlockToRemove <-}\StringTok{ }\DecValTok{1}
\CommentTok{#trialsToRemove <- seq(from = 1, to = nrTrialsPerBlockToRemove)}

\NormalTok{  data <-}\StringTok{  }\NormalTok{data }\OperatorTok
\StringTok{    }\KeywordTok{group_by}\NormalTok{(ID, Part, InterpenetrationFeedback, FullyShaded) }\OperatorTok\StringTok{ }\CommentTok{# I have added here the fully shaded }
\StringTok{    }\KeywordTok{slice}\NormalTok{(nrTrialsPerBlockToRemove}\OperatorTok{+}\DecValTok{1}\OperatorTok{:}\KeywordTok{n}\NormalTok{())}
  \CommentTok{# to double check we are discarding the right rows}
  \CommentTok{#print(data[[i]]$Trial)}

 \CommentTok{# This df will be used to create the subsets for 1st part and 2nd part of the experiment. }
  
\NormalTok{data}\OperatorTok{$}\NormalTok{InterpenetrationFeedback  <-}\StringTok{ }\KeywordTok{as.factor}\NormalTok{(data}\OperatorTok{$}\NormalTok{InterpenetrationFeedback)}
\NormalTok{data}\OperatorTok{$}\NormalTok{FullyShaded <-}\StringTok{ }\KeywordTok{as.factor}\NormalTok{(data}\OperatorTok{$}\NormalTok{FullyShaded)}
\end{Highlighting}
\end{Shaded}

\begin{Shaded}
\begin{Highlighting}[]
\NormalTok{ParsiDF <-}\StringTok{ }\NormalTok{data}


\CommentTok{# Exclude the IDs which produced the extreme values (i.e., = or > 3 coefficients from the mean)}
\NormalTok{ParsiDF}\OperatorTok{$}\NormalTok{ID[ParsiDF}\OperatorTok{$}\NormalTok{ID }\OperatorTok{==}\StringTok{ }\DecValTok{9}\NormalTok{] <-}\StringTok{ }\OtherTok{NA} 
\NormalTok{ParsiDF}\OperatorTok{$}\NormalTok{ID[ParsiDF}\OperatorTok{$}\NormalTok{ID }\OperatorTok{==}\StringTok{ }\DecValTok{17}\NormalTok{] <-}\StringTok{ }\OtherTok{NA}
\NormalTok{ParsiDF}\OperatorTok{$}\NormalTok{ID[ParsiDF}\OperatorTok{$}\NormalTok{ID }\OperatorTok{==}\StringTok{ }\DecValTok{20}\NormalTok{] <-}\StringTok{ }\OtherTok{NA}

\NormalTok{ParsiDF <-}\StringTok{ }\KeywordTok{na.omit}\NormalTok{(ParsiDF)}



\NormalTok{ParsiDF <-}\StringTok{ }\KeywordTok{aggregate}\NormalTok{(. }\OperatorTok{~}\StringTok{ }\NormalTok{ID }\OperatorTok{+}\StringTok{ }\NormalTok{Age }\OperatorTok{+}\StringTok{ }\NormalTok{Gender }\OperatorTok{+}\StringTok{ }\NormalTok{InterpenetrationFeedback }\OperatorTok{+}\StringTok{ }\NormalTok{Part, ParsiDF, mean)}
\CommentTok{#describe(ParsiDF)}
\CommentTok{#Before Conversion to logarithms (showing the abnormal distribution)}
\KeywordTok{shapiro_test}\NormalTok{(ParsiDF}\OperatorTok{$}\NormalTok{MaxInterpenetration)}
\end{Highlighting}
\end{Shaded}

\begin{verbatim}
## # A tibble: 1 x 3
##   variable                    statistic     p.value
##   <chr>                           <dbl>       <dbl>
## 1 ParsiDF$MaxInterpenetration     0.937 0.000000946
\end{verbatim}

\begin{Shaded}
\begin{Highlighting}[]
\KeywordTok{shapiro_test}\NormalTok{(ParsiDF}\OperatorTok{$}\NormalTok{AverageInterpenetration)}
\end{Highlighting}
\end{Shaded}

\begin{verbatim}
## # A tibble: 1 x 3
##   variable                        statistic     p.value
##   <chr>                               <dbl>       <dbl>
## 1 ParsiDF$AverageInterpenetration     0.936 0.000000844
\end{verbatim}

\begin{Shaded}
\begin{Highlighting}[]
\KeywordTok{ggqqplot}\NormalTok{(ParsiDF}\OperatorTok{$}\NormalTok{MaxInterpenetration)}
\end{Highlighting}
\end{Shaded}

\includegraphics{Report_files/figure-latex/unnamed-chunk-4-1.pdf}

\begin{Shaded}
\begin{Highlighting}[]
\KeywordTok{ggqqplot}\NormalTok{(ParsiDF}\OperatorTok{$}\NormalTok{AverageInterpenetration)}
\end{Highlighting}
\end{Shaded}

\includegraphics{Report_files/figure-latex/unnamed-chunk-4-2.pdf}

\begin{Shaded}
\begin{Highlighting}[]
\NormalTok{ParsiDF }\OperatorTok
\StringTok{  }\KeywordTok{group_by}\NormalTok{(InterpenetrationFeedback, Part) }\OperatorTok
\StringTok{  }\KeywordTok{shapiro_test}\NormalTok{(MaxInterpenetration) }
\end{Highlighting}
\end{Shaded}

\begin{verbatim}
## # A tibble: 8 x 5
##   InterpenetrationFeedback Part   variable            statistic       p
##   <fct>                    <fct>  <chr>                   <dbl>   <dbl>
## 1 Both                     Part 1 MaxInterpenetration     0.946 0.290  
## 2 Both                     Part 2 MaxInterpenetration     0.834 0.00225
## 3 Electrotactile           Part 1 MaxInterpenetration     0.967 0.677  
## 4 Electrotactile           Part 2 MaxInterpenetration     0.859 0.00611
## 5 NoFeedback               Part 1 MaxInterpenetration     0.969 0.720  
## 6 NoFeedback               Part 2 MaxInterpenetration     0.977 0.872  
## 7 Visual                   Part 1 MaxInterpenetration     0.860 0.00622
## 8 Visual                   Part 2 MaxInterpenetration     0.898 0.0327
\end{verbatim}

\begin{Shaded}
\begin{Highlighting}[]
\NormalTok{ParsiDF }\OperatorTok
\StringTok{  }\KeywordTok{group_by}\NormalTok{(InterpenetrationFeedback, Part) }\OperatorTok
\StringTok{  }\KeywordTok{shapiro_test}\NormalTok{(AverageInterpenetration)  }
\end{Highlighting}
\end{Shaded}

\begin{verbatim}
## # A tibble: 8 x 5
##   InterpenetrationFeedback Part   variable                statistic       p
##   <fct>                    <fct>  <chr>                       <dbl>   <dbl>
## 1 Both                     Part 1 AverageInterpenetration     0.947 0.301  
## 2 Both                     Part 2 AverageInterpenetration     0.827 0.00175
## 3 Electrotactile           Part 1 AverageInterpenetration     0.964 0.592  
## 4 Electrotactile           Part 2 AverageInterpenetration     0.877 0.0128 
## 5 NoFeedback               Part 1 AverageInterpenetration     0.964 0.598  
## 6 NoFeedback               Part 2 AverageInterpenetration     0.955 0.420  
## 7 Visual                   Part 1 AverageInterpenetration     0.851 0.00445
## 8 Visual                   Part 2 AverageInterpenetration     0.914 0.0655
\end{verbatim}

\begin{Shaded}
\begin{Highlighting}[]
\CommentTok{#After Conversion of the performance variables into logs (Normal Distribution)}
\NormalTok{ParsiDF}\OperatorTok{$}\NormalTok{MaxInterpenetration <-}\StringTok{ }\KeywordTok{log}\NormalTok{(ParsiDF}\OperatorTok{$}\NormalTok{MaxInterpenetration)}

\NormalTok{ParsiDF}\OperatorTok{$}\NormalTok{AverageInterpenetration <-}\StringTok{ }\KeywordTok{log}\NormalTok{(ParsiDF}\OperatorTok{$}\NormalTok{AverageInterpenetration)}

\KeywordTok{shapiro_test}\NormalTok{(ParsiDF}\OperatorTok{$}\NormalTok{MaxInterpenetration)}
\end{Highlighting}
\end{Shaded}

\begin{verbatim}
## # A tibble: 1 x 3
##   variable                    statistic p.value
##   <chr>                           <dbl>   <dbl>
## 1 ParsiDF$MaxInterpenetration     0.988   0.149
\end{verbatim}

\begin{Shaded}
\begin{Highlighting}[]
\KeywordTok{shapiro_test}\NormalTok{(ParsiDF}\OperatorTok{$}\NormalTok{AverageInterpenetration)}
\end{Highlighting}
\end{Shaded}

\begin{verbatim}
## # A tibble: 1 x 3
##   variable                        statistic p.value
##   <chr>                               <dbl>   <dbl>
## 1 ParsiDF$AverageInterpenetration     0.986  0.0873
\end{verbatim}

\begin{Shaded}
\begin{Highlighting}[]
\KeywordTok{ggqqplot}\NormalTok{(ParsiDF}\OperatorTok{$}\NormalTok{MaxInterpenetration)}
\end{Highlighting}
\end{Shaded}

\includegraphics{Report_files/figure-latex/unnamed-chunk-4-3.pdf}

\begin{Shaded}
\begin{Highlighting}[]
\KeywordTok{ggqqplot}\NormalTok{(ParsiDF}\OperatorTok{$}\NormalTok{AverageInterpenetration)}
\end{Highlighting}
\end{Shaded}

\includegraphics{Report_files/figure-latex/unnamed-chunk-4-4.pdf}

\begin{Shaded}
\begin{Highlighting}[]
\CommentTok{#Let's check the assumption for each interpenetration feedback and shade condition}
\NormalTok{ParsiDF }\OperatorTok
\StringTok{  }\KeywordTok{group_by}\NormalTok{(InterpenetrationFeedback, Part) }\OperatorTok
\StringTok{  }\KeywordTok{shapiro_test}\NormalTok{(MaxInterpenetration) }
\end{Highlighting}
\end{Shaded}

\begin{verbatim}
## # A tibble: 8 x 5
##   InterpenetrationFeedback Part   variable            statistic     p
##   <fct>                    <fct>  <chr>                   <dbl> <dbl>
## 1 Both                     Part 1 MaxInterpenetration     0.987 0.989
## 2 Both                     Part 2 MaxInterpenetration     0.966 0.646
## 3 Electrotactile           Part 1 MaxInterpenetration     0.931 0.143
## 4 Electrotactile           Part 2 MaxInterpenetration     0.975 0.831
## 5 NoFeedback               Part 1 MaxInterpenetration     0.958 0.468
## 6 NoFeedback               Part 2 MaxInterpenetration     0.951 0.352
## 7 Visual                   Part 1 MaxInterpenetration     0.963 0.582
## 8 Visual                   Part 2 MaxInterpenetration     0.976 0.851
\end{verbatim}

\begin{Shaded}
\begin{Highlighting}[]
\NormalTok{ParsiDF }\OperatorTok
\StringTok{  }\KeywordTok{group_by}\NormalTok{(InterpenetrationFeedback, Part) }\OperatorTok
\StringTok{  }\KeywordTok{shapiro_test}\NormalTok{(AverageInterpenetration) }
\end{Highlighting}
\end{Shaded}

\begin{verbatim}
## # A tibble: 8 x 5
##   InterpenetrationFeedback Part   variable                statistic     p
##   <fct>                    <fct>  <chr>                       <dbl> <dbl>
## 1 Both                     Part 1 AverageInterpenetration     0.976 0.854
## 2 Both                     Part 2 AverageInterpenetration     0.970 0.740
## 3 Electrotactile           Part 1 AverageInterpenetration     0.954 0.396
## 4 Electrotactile           Part 2 AverageInterpenetration     0.979 0.906
## 5 NoFeedback               Part 1 AverageInterpenetration     0.956 0.434
## 6 NoFeedback               Part 2 AverageInterpenetration     0.977 0.873
## 7 Visual                   Part 1 AverageInterpenetration     0.955 0.428
## 8 Visual                   Part 2 AverageInterpenetration     0.964 0.600
\end{verbatim}

\begin{Shaded}
\begin{Highlighting}[]
\KeywordTok{hist}\NormalTok{(ParsiDF}\OperatorTok{$}\NormalTok{MaxInterpenetration,}\DataTypeTok{main =} \KeywordTok{paste}\NormalTok{(}\StringTok{"Histogram of Maximum Interpenetration"}\NormalTok{) , }\DataTypeTok{xlab =} \StringTok{"Maximum Interpenetration"}\NormalTok{)}
\end{Highlighting}
\end{Shaded}

\includegraphics{Report_files/figure-latex/unnamed-chunk-4-5.pdf}

\begin{Shaded}
\begin{Highlighting}[]
\KeywordTok{hist}\NormalTok{(ParsiDF}\OperatorTok{$}\NormalTok{AverageInterpenetration, }\DataTypeTok{main =} \KeywordTok{paste}\NormalTok{(}\StringTok{"Histogram of Average Interpenetration"}\NormalTok{) , }\DataTypeTok{xlab =} \StringTok{"Average Interpenetration"}\NormalTok{)}
\end{Highlighting}
\end{Shaded}

\includegraphics{Report_files/figure-latex/unnamed-chunk-4-6.pdf}

\begin{Shaded}
\begin{Highlighting}[]
\NormalTok{Descriptive <-}\StringTok{ }\KeywordTok{describeBy}\NormalTok{(ParsiDF, }\DataTypeTok{group =}\NormalTok{ ParsiDF}\OperatorTok{$}\NormalTok{InterpenetrationFeedback)}
\NormalTok{Descriptive}
\end{Highlighting}
\end{Shaded}

\begin{verbatim}
## 
##  Descriptive statistics by group 
## group: Both
##                           vars  n  mean    sd median trimmed   mad   min   max
## ID                           1 42 12.10  7.18  12.00   12.00  8.90  1.00 24.00
## Age                          2 42 25.67  2.76  25.00   25.24  1.48 23.00 33.00
## Gender*                      3 42  1.76  0.53   2.00    1.76  0.00  1.00  3.00
## InterpenetrationFeedback*    4 42  1.00  0.00   1.00    1.00  0.00  1.00  1.00
## Part*                        5 42  1.50  0.51   1.50    1.50  0.74  1.00  2.00
## Block                        6 42  3.43  2.29   3.50    3.41  2.97  0.00  7.00
## TrialInBlock                 7 42  6.42  0.11   6.50    6.44  0.00  6.10  6.50
## Trial                        8 42 47.56 27.41  48.40   47.37 35.43  6.10 90.50
## FullyShaded                  9 42  1.50  0.00   1.50    1.50  0.00  1.50  1.50
## NrOfAdjustments             10 42  0.05  0.12   0.00    0.02  0.00  0.00  0.50
## InterpenetrationTime        11 42  3.62  0.23   3.62    3.62  0.18  2.97  4.52
## TrialTime                   12 42  8.22  2.18   7.66    7.85  0.81  6.57 20.52
## FirstContactTime            13 42  2.97  1.74   2.56    2.67  0.73  1.55 12.54
## MaxInterpenetration         14 42 -4.64  0.38  -4.64   -4.64  0.39 -5.42 -3.74
## AverageInterpenetration     15 42 -4.97  0.43  -4.99   -4.98  0.43 -5.78 -4.04
## AverageOffsetFromSurface    16 42  0.01  0.00   0.01    0.01  0.00  0.00  0.02
## ReleaseReactionTime         17 42  0.62  0.21   0.60    0.61  0.14  0.07  1.51
## Completed                   18 42  1.00  0.00   1.00    1.00  0.00  1.00  1.00
## Attempts                    19 42  1.00  0.00   1.00    1.00  0.00  1.00  1.00
## Skipped                     20 42  0.00  0.00   0.00    0.00  0.00  0.00  0.00
## RespTime                    21 42  8.19  2.18   7.64    7.83  0.81  6.55 20.50
##                           range  skew kurtosis   se
## ID                        23.00  0.10    -1.29 1.11
## Age                       10.00  1.43     0.84 0.43
## Gender*                    2.00 -0.18    -0.38 0.08
## InterpenetrationFeedback*  0.00   NaN      NaN 0.00
## Part*                      1.00  0.00    -2.05 0.08
## Block                      7.00  0.01    -1.37 0.35
## TrialInBlock               0.40 -1.40     1.31 0.02
## Trial                     84.40  0.01    -1.36 4.23
## FullyShaded                0.00   NaN      NaN 0.00
## NrOfAdjustments            0.50  2.38     4.77 0.02
## InterpenetrationTime       1.55  0.66     4.99 0.04
## TrialTime                 13.94  4.29    21.28 0.34
## FirstContactTime          10.98  4.01    19.01 0.27
## MaxInterpenetration        1.68  0.12    -0.26 0.06
## AverageInterpenetration    1.74  0.17    -0.66 0.07
## AverageOffsetFromSurface   0.02  1.17     1.43 0.00
## ReleaseReactionTime        1.44  1.32     6.23 0.03
## Completed                  0.00   NaN      NaN 0.00
## Attempts                   0.00   NaN      NaN 0.00
## Skipped                    0.00   NaN      NaN 0.00
## RespTime                  13.94  4.29    21.28 0.34
## ------------------------------------------------------------ 
## group: Electrotactile
##                           vars  n  mean    sd median trimmed   mad   min   max
## ID                           1 42 12.10  7.18  12.00   12.00  8.90  1.00 24.00
## Age                          2 42 25.67  2.76  25.00   25.24  1.48 23.00 33.00
## Gender*                      3 42  1.76  0.53   2.00    1.76  0.00  1.00  3.00
## InterpenetrationFeedback*    4 42  2.00  0.00   2.00    2.00  0.00  2.00  2.00
## Part*                        5 42  1.50  0.51   1.50    1.50  0.74  1.00  2.00
## Block                        6 42  3.48  2.35   3.50    3.47  2.97  0.00  7.00
## TrialInBlock                 7 42  6.43  0.09   6.50    6.44  0.00  6.20  6.50
## Trial                        8 42 48.14 28.19  48.40   48.07 35.58  6.30 90.50
## FullyShaded                  9 42  1.50  0.00   1.50    1.50  0.00  1.50  1.50
## NrOfAdjustments             10 42  0.02  0.05   0.00    0.01  0.00  0.00  0.20
## InterpenetrationTime        11 42  3.61  0.15   3.61    3.60  0.11  3.24  4.05
## TrialTime                   12 42  7.87  1.30   7.51    7.66  0.56  6.33 12.92
## FirstContactTime            13 42  2.74  1.10   2.46    2.54  0.50  1.49  6.28
## MaxInterpenetration         14 42 -4.40  0.45  -4.41   -4.39  0.53 -5.35 -3.60
## AverageInterpenetration     15 42 -4.72  0.47  -4.65   -4.72  0.53 -5.66 -3.80
## AverageOffsetFromSurface    16 42  0.01  0.00   0.01    0.01  0.00  0.00  0.02
## ReleaseReactionTime         17 42  0.60  0.15   0.60    0.59  0.11  0.22  1.04
## Completed                   18 42  1.00  0.00   1.00    1.00  0.00  1.00  1.00
## Attempts                    19 42  1.00  0.00   1.00    1.00  0.00  1.00  1.00
## Skipped                     20 42  0.00  0.00   0.00    0.00  0.00  0.00  0.00
## RespTime                    21 42  7.85  1.30   7.49    7.63  0.56  6.31 12.90
##                           range  skew kurtosis   se
## ID                        23.00  0.10    -1.29 1.11
## Age                       10.00  1.43     0.84 0.43
## Gender*                    2.00 -0.18    -0.38 0.08
## InterpenetrationFeedback*  0.00   NaN      NaN 0.00
## Part*                      1.00  0.00    -2.05 0.08
## Block                      7.00  0.01    -1.28 0.36
## TrialInBlock               0.30 -0.87    -0.56 0.01
## Trial                     84.20  0.01    -1.28 4.35
## FullyShaded                0.00   NaN      NaN 0.00
## NrOfAdjustments            0.20  2.25     4.43 0.01
## InterpenetrationTime       0.82  0.76     1.62 0.02
## TrialTime                  6.59  1.89     4.06 0.20
## FirstContactTime           4.79  1.63     2.16 0.17
## MaxInterpenetration        1.75 -0.13    -0.91 0.07
## AverageInterpenetration    1.86 -0.02    -0.94 0.07
## AverageOffsetFromSurface   0.01  0.62     0.10 0.00
## ReleaseReactionTime        0.82  0.76     1.84 0.02
## Completed                  0.00   NaN      NaN 0.00
## Attempts                   0.00   NaN      NaN 0.00
## Skipped                    0.00   NaN      NaN 0.00
## RespTime                   6.59  1.89     4.06 0.20
## ------------------------------------------------------------ 
## group: NoFeedback
##                           vars  n  mean    sd median trimmed   mad   min   max
## ID                           1 42 12.10  7.18  12.00   12.00  8.90  1.00 24.00
## Age                          2 42 25.67  2.76  25.00   25.24  1.48 23.00 33.00
## Gender*                      3 42  1.76  0.53   2.00    1.76  0.00  1.00  3.00
## InterpenetrationFeedback*    4 42  3.00  0.00   3.00    3.00  0.00  3.00  3.00
## Part*                        5 42  1.50  0.51   1.50    1.50  0.74  1.00  2.00
## Block                        6 42  3.57  2.29   3.50    3.59  2.97  0.00  7.00
## TrialInBlock                 7 42  6.44  0.10   6.50    6.46  0.00  6.10  6.50
## Trial                        8 42 49.30 27.45  48.30   49.49 35.58  6.40 90.50
## FullyShaded                  9 42  1.50  0.00   1.50    1.50  0.00  1.50  1.50
## NrOfAdjustments             10 42  0.02  0.06   0.00    0.01  0.00  0.00  0.30
## InterpenetrationTime        11 42  3.62  0.14   3.59    3.61  0.13  3.26  3.95
## TrialTime                   12 42  7.92  1.59   7.58    7.69  1.02  5.99 13.22
## FirstContactTime            13 42  2.80  1.38   2.49    2.59  1.01  1.22  7.51
## MaxInterpenetration         14 42 -4.12  0.34  -4.09   -4.10  0.29 -4.93 -3.48
## AverageInterpenetration     15 42 -4.45  0.33  -4.44   -4.44  0.27 -5.24 -3.85
## AverageOffsetFromSurface    16 42  0.01  0.00   0.01    0.01  0.00  0.00  0.01
## ReleaseReactionTime         17 42  0.61  0.16   0.59    0.60  0.13  0.25  1.03
## Completed                   18 42  1.00  0.00   1.00    1.00  0.00  1.00  1.00
## Attempts                    19 42  1.00  0.00   1.00    1.00  0.00  1.00  1.00
## Skipped                     20 42  0.00  0.00   0.00    0.00  0.00  0.00  0.00
## RespTime                    21 42  7.90  1.59   7.55    7.66  1.02  5.96 13.19
##                           range  skew kurtosis   se
## ID                        23.00  0.10    -1.29 1.11
## Age                       10.00  1.43     0.84 0.43
## Gender*                    2.00 -0.18    -0.38 0.08
## InterpenetrationFeedback*  0.00   NaN      NaN 0.00
## Part*                      1.00  0.00    -2.05 0.08
## Block                      7.00 -0.01    -1.37 0.35
## TrialInBlock               0.40 -1.86     2.99 0.01
## Trial                     84.10 -0.01    -1.37 4.24
## FullyShaded                0.00   NaN      NaN 0.00
## NrOfAdjustments            0.30  2.86     8.42 0.01
## InterpenetrationTime       0.68  0.23    -0.10 0.02
## TrialTime                  7.23  1.51     2.04 0.25
## FirstContactTime           6.29  1.46     1.82 0.21
## MaxInterpenetration        1.46 -0.38    -0.05 0.05
## AverageInterpenetration    1.39 -0.23    -0.31 0.05
## AverageOffsetFromSurface   0.01  1.13     1.36 0.00
## ReleaseReactionTime        0.78  0.45     0.18 0.02
## Completed                  0.00   NaN      NaN 0.00
## Attempts                   0.00   NaN      NaN 0.00
## Skipped                    0.00   NaN      NaN 0.00
## RespTime                   7.23  1.51     2.04 0.25
## ------------------------------------------------------------ 
## group: Visual
##                           vars  n  mean    sd median trimmed   mad   min   max
## ID                           1 42 12.10  7.18  12.00   12.00  8.90  1.00 24.00
## Age                          2 42 25.67  2.76  25.00   25.24  1.48 23.00 33.00
## Gender*                      3 42  1.76  0.53   2.00    1.76  0.00  1.00  3.00
## InterpenetrationFeedback*    4 42  4.00  0.00   4.00    4.00  0.00  4.00  4.00
## Part*                        5 42  1.50  0.51   1.50    1.50  0.74  1.00  2.00
## Block                        6 42  3.52  2.35   3.50    3.53  2.97  0.00  7.00
## TrialInBlock                 7 42  6.42  0.11   6.50    6.44  0.00  6.00  6.50
## Trial                        8 42 48.70 28.17  48.35   48.76 35.43  6.40 90.50
## FullyShaded                  9 42  1.50  0.00   1.50    1.50  0.00  1.50  1.50
## NrOfAdjustments             10 42  0.02  0.05   0.00    0.01  0.00  0.00  0.20
## InterpenetrationTime        11 42  3.65  0.15   3.62    3.64  0.09  3.35  4.05
## TrialTime                   12 42  7.97  1.17   7.82    7.80  0.90  6.37 12.04
## FirstContactTime            13 42  2.84  1.03   2.65    2.69  0.72  1.36  6.56
## MaxInterpenetration         14 42 -4.54  0.41  -4.62   -4.56  0.42 -5.22 -3.48
## AverageInterpenetration     15 42 -4.84  0.45  -4.86   -4.87  0.49 -5.55 -3.76
## AverageOffsetFromSurface    16 42  0.01  0.00   0.01    0.01  0.00  0.00  0.01
## ReleaseReactionTime         17 42  0.65  0.15   0.61    0.63  0.09  0.42  1.13
## Completed                   18 42  1.00  0.00   1.00    1.00  0.00  1.00  1.00
## Attempts                    19 42  1.00  0.00   1.00    1.00  0.00  1.00  1.00
## Skipped                     20 42  0.00  0.00   0.00    0.00  0.00  0.00  0.00
## RespTime                    21 42  7.95  1.17   7.79    7.78  0.90  6.34 12.02
##                           range  skew kurtosis   se
## ID                        23.00  0.10    -1.29 1.11
## Age                       10.00  1.43     0.84 0.43
## Gender*                    2.00 -0.18    -0.38 0.08
## InterpenetrationFeedback*  0.00   NaN      NaN 0.00
## Part*                      1.00  0.00    -2.05 0.08
## Block                      7.00 -0.01    -1.28 0.36
## TrialInBlock               0.50 -1.65     2.74 0.02
## Trial                     84.10 -0.01    -1.28 4.35
## FullyShaded                0.00   NaN      NaN 0.00
## NrOfAdjustments            0.20  2.29     4.29 0.01
## InterpenetrationTime       0.70  0.81     0.74 0.02
## TrialTime                  5.67  1.45     2.26 0.18
## FirstContactTime           5.21  1.55     2.79 0.16
## MaxInterpenetration        1.73  0.48    -0.49 0.06
## AverageInterpenetration    1.79  0.42    -0.68 0.07
## AverageOffsetFromSurface   0.01  0.59     1.09 0.00
## ReleaseReactionTime        0.71  1.25     1.71 0.02
## Completed                  0.00   NaN      NaN 0.00
## Attempts                   0.00   NaN      NaN 0.00
## Skipped                    0.00   NaN      NaN 0.00
## RespTime                   5.67  1.45     2.26 0.18
\end{verbatim}

\#Let's visualize the data per interpenetration feedback and/or part of
the experiment (part 1 \& part 2).

\begin{Shaded}
\begin{Highlighting}[]
\NormalTok{ParsiDFplots <-}\StringTok{ }\NormalTok{data }\CommentTok{# A dataframe just for the plots, so we show everything in real numbers and in centimeters!}
\NormalTok{ParsiDFplots}\OperatorTok{$}\NormalTok{ID[ParsiDFplots}\OperatorTok{$}\NormalTok{ID }\OperatorTok{==}\StringTok{ }\DecValTok{9}\NormalTok{] <-}\StringTok{ }\OtherTok{NA}
\NormalTok{ParsiDFplots}\OperatorTok{$}\NormalTok{ID[ParsiDFplots}\OperatorTok{$}\NormalTok{ID }\OperatorTok{==}\StringTok{ }\DecValTok{17}\NormalTok{] <-}\StringTok{ }\OtherTok{NA}
\NormalTok{ParsiDFplots}\OperatorTok{$}\NormalTok{ID[ParsiDFplots}\OperatorTok{$}\NormalTok{ID }\OperatorTok{==}\StringTok{ }\DecValTok{20}\NormalTok{] <-}\StringTok{ }\OtherTok{NA}
\NormalTok{ParsiDFplots <-}\StringTok{ }\KeywordTok{na.omit}\NormalTok{(ParsiDFplots)}
\NormalTok{ParsiDFplots <-}\StringTok{ }\KeywordTok{aggregate}\NormalTok{(. }\OperatorTok{~}\StringTok{ }\NormalTok{ID }\OperatorTok{+}\StringTok{ }\NormalTok{Age }\OperatorTok{+}\StringTok{ }\NormalTok{Gender }\OperatorTok{+}\StringTok{ }\NormalTok{InterpenetrationFeedback }\OperatorTok{+}\StringTok{ }\NormalTok{Part, ParsiDFplots, mean)}

\NormalTok{ParsiDFplots}\OperatorTok{$}\NormalTok{AverageInterpenetration <-}\DecValTok{100} \OperatorTok{*}\StringTok{ }\NormalTok{ParsiDFplots}\OperatorTok{$}\NormalTok{AverageInterpenetration }\CommentTok{#Converting meters to centimeters}
\NormalTok{ParsiDFplots}\OperatorTok{$}\NormalTok{MaxInterpenetration <-}\StringTok{ }\DecValTok{100} \OperatorTok{*}\StringTok{ }\NormalTok{ParsiDFplots}\OperatorTok{$}\NormalTok{MaxInterpenetration }\CommentTok{#Converting meters to centimeters}

\NormalTok{p1 <-}\StringTok{ }\NormalTok{ggstatsplot}\OperatorTok{::}\KeywordTok{ggbetweenstats}\NormalTok{(}
  \DataTypeTok{data =}\NormalTok{ ParsiDFplots,}
  \DataTypeTok{x =} \StringTok{"InterpenetrationFeedback"}\NormalTok{, }\CommentTok{#Indepedent Variable}
  \DataTypeTok{y =} \StringTok{"MaxInterpenetration"}\NormalTok{, }\CommentTok{# Depedent Variable}
  \DataTypeTok{grouping.var =} \StringTok{"Part"}\NormalTok{, }\CommentTok{# 2nd IV }
  \DataTypeTok{type =} \StringTok{"p"}\NormalTok{, }\CommentTok{# parametric test i.e., p values}
  \DataTypeTok{pairwise.comparisons =} \OtherTok{FALSE}\NormalTok{, }\CommentTok{#compute pairwise comparisons}
  \DataTypeTok{pairwise.display =} \StringTok{"significant"}\NormalTok{, }\CommentTok{# show only the significant ones}
  \DataTypeTok{p.adjust.method =} \StringTok{"bonferroni"}\NormalTok{, }\CommentTok{# correction of p-value}
  \DataTypeTok{effsize.type =} \StringTok{"unbiased"}\NormalTok{, }\CommentTok{# Calculates the Hedge's g for t tests and the partial Omega for ANOVA}
  \DataTypeTok{results.subtitle =} \OtherTok{FALSE}\NormalTok{,}
  \DataTypeTok{xlab =} \StringTok{"Type of Feedback"}\NormalTok{, }\CommentTok{#label of X axis}
  \DataTypeTok{ylab =} \StringTok{"Maximum Interpenetration"}\NormalTok{, }\CommentTok{#label of y axis}
  \DataTypeTok{sample.size.label =} \OtherTok{FALSE}\NormalTok{,}
  \DataTypeTok{var.equal =} \OtherTok{TRUE}\NormalTok{, }\CommentTok{#Assuming Equal variances}
  \DataTypeTok{mean.plotting =} \OtherTok{FALSE}\NormalTok{,}
  \DataTypeTok{mean.ci =} \OtherTok{TRUE}\NormalTok{, }\CommentTok{#display the confidence interval of the mean}
  \DataTypeTok{paired =} \OtherTok{TRUE}\NormalTok{, }\CommentTok{#indicating that we have a within subject design}
  \DataTypeTok{title.text =} \StringTok{"Interpenetration Box-Violin Plots"}\NormalTok{,}
  \DataTypeTok{caption.text =} \StringTok{"Note: Interpenetration distance is displayed in cm."}\NormalTok{,}
  \DataTypeTok{title.color =} \StringTok{"black"}\NormalTok{,}
  \DataTypeTok{caption.color =} \StringTok{"black"}
\NormalTok{  ) }

\NormalTok{p2 <-}\StringTok{ }\NormalTok{ggstatsplot}\OperatorTok{::}\KeywordTok{ggbetweenstats}\NormalTok{(}
  \DataTypeTok{data =}\NormalTok{ ParsiDFplots,}
  \DataTypeTok{x =} \StringTok{"InterpenetrationFeedback"}\NormalTok{,}
  \DataTypeTok{y =} \StringTok{"AverageInterpenetration"}\NormalTok{,}
  \DataTypeTok{grouping.var =} \StringTok{"Part"}\NormalTok{,}
  \DataTypeTok{type =} \StringTok{"p"}\NormalTok{,}
  \DataTypeTok{pairwise.comparisons =} \OtherTok{FALSE}\NormalTok{,}
  \DataTypeTok{pairwise.display =} \StringTok{"significant"}\NormalTok{,}
  \DataTypeTok{p.adjust.method =} \StringTok{"bonferroni"}\NormalTok{,}
  \DataTypeTok{effsize.type =} \StringTok{"unbiased"}\NormalTok{,}
  \DataTypeTok{results.subtitle =} \OtherTok{FALSE}\NormalTok{,}
  \DataTypeTok{xlab =} \StringTok{"Type of Feedback"}\NormalTok{,}
  \DataTypeTok{ylab =} \StringTok{"Average Interpenetration"}\NormalTok{,}
  \DataTypeTok{sample.size.label =} \OtherTok{FALSE}\NormalTok{,}
  \DataTypeTok{var.equal =} \OtherTok{TRUE}\NormalTok{,}
  \DataTypeTok{mean.plotting =} \OtherTok{FALSE}\NormalTok{,}
  \DataTypeTok{mean.ci =} \OtherTok{TRUE}\NormalTok{,}
  \DataTypeTok{paired =} \OtherTok{TRUE}\NormalTok{,}
  \DataTypeTok{title.text =} \StringTok{"Interpenetration Box-Violin Plots"}\NormalTok{,}
  \DataTypeTok{caption.text =} \StringTok{"Note: Interpenetration distance is displayed in cm."}\NormalTok{,}
  \DataTypeTok{title.color =} \StringTok{"black"}\NormalTok{,}
  \DataTypeTok{caption.color =} \StringTok{"black"}
\NormalTok{)}


\CommentTok{# Replicating the above but this time we look on the effect of the type of feedback on the DVs in 1st and 2nd Part of the experiment individually}
\NormalTok{p3 <-}\StringTok{ }\NormalTok{ggstatsplot}\OperatorTok{::}\KeywordTok{grouped_ggbetweenstats}\NormalTok{(}
  \DataTypeTok{data =}\NormalTok{ ParsiDFplots,}
  \DataTypeTok{x =} \StringTok{"InterpenetrationFeedback"}\NormalTok{,}
  \DataTypeTok{y =} \StringTok{"MaxInterpenetration"}\NormalTok{,}
  \DataTypeTok{grouping.var =} \StringTok{"Part"}\NormalTok{,}
  \DataTypeTok{type =} \StringTok{"p"}\NormalTok{,}
  \DataTypeTok{pairwise.comparisons =} \OtherTok{FALSE}\NormalTok{,}
  \DataTypeTok{pairwise.display =} \StringTok{"significant"}\NormalTok{,}
  \DataTypeTok{p.adjust.method =} \StringTok{"bonferroni"}\NormalTok{,}
  \DataTypeTok{effsize.type =} \StringTok{"unbiased"}\NormalTok{,}
  \DataTypeTok{results.subtitle =} \OtherTok{FALSE}\NormalTok{,}
  \DataTypeTok{xlab =} \StringTok{"Type of Feedback"}\NormalTok{,}
  \DataTypeTok{ylab =} \StringTok{"Maximum Interpenetration"}\NormalTok{,}
  \DataTypeTok{sample.size.label =} \OtherTok{FALSE}\NormalTok{,}
  \DataTypeTok{var.equal =} \OtherTok{TRUE}\NormalTok{,}
  \DataTypeTok{mean.plotting =} \OtherTok{FALSE}\NormalTok{,}
  \DataTypeTok{mean.ci =} \OtherTok{TRUE}\NormalTok{,}
  \DataTypeTok{paired =} \OtherTok{TRUE}\NormalTok{,}
  \DataTypeTok{title.text =} \StringTok{"Interpenetration Box-Violin Plots"}\NormalTok{,}
  \DataTypeTok{caption.text =} \StringTok{"Note: Interpenetration distance is displayed in cm."}\NormalTok{,}
  \DataTypeTok{title.color =} \StringTok{"black"}\NormalTok{,}
  \DataTypeTok{caption.color =} \StringTok{"black"}
\NormalTok{  ) }

\NormalTok{p4 <-}\StringTok{ }\NormalTok{ggstatsplot}\OperatorTok{::}\KeywordTok{grouped_ggbetweenstats}\NormalTok{(}
  \DataTypeTok{data =}\NormalTok{ ParsiDFplots,}
  \DataTypeTok{x =} \StringTok{"InterpenetrationFeedback"}\NormalTok{,}
  \DataTypeTok{y =} \StringTok{"AverageInterpenetration"}\NormalTok{,}
  \DataTypeTok{grouping.var =} \StringTok{"Part"}\NormalTok{,}
  \DataTypeTok{type =} \StringTok{"p"}\NormalTok{,}
  \DataTypeTok{pairwise.comparisons =} \OtherTok{FALSE}\NormalTok{,}
  \DataTypeTok{pairwise.display =} \StringTok{"significant"}\NormalTok{,}
  \DataTypeTok{p.adjust.method =} \StringTok{"bonferroni"}\NormalTok{,}
  \DataTypeTok{effsize.type =} \StringTok{"unbiased"}\NormalTok{,}
  \DataTypeTok{results.subtitle =} \OtherTok{FALSE}\NormalTok{,}
  \DataTypeTok{xlab =} \StringTok{"Type of Feedback"}\NormalTok{,}
  \DataTypeTok{ylab =} \StringTok{"Average Interpenetration"}\NormalTok{,}
  \DataTypeTok{sample.size.label =} \OtherTok{FALSE}\NormalTok{,}
  \DataTypeTok{var.equal =} \OtherTok{TRUE}\NormalTok{,}
  \DataTypeTok{mean.plotting =} \OtherTok{FALSE}\NormalTok{,}
  \DataTypeTok{mean.ci =} \OtherTok{TRUE}\NormalTok{,}
  \DataTypeTok{paired =} \OtherTok{TRUE}\NormalTok{,}
  \DataTypeTok{title.text =} \StringTok{"Interpenetration Box-Violin Plots"}\NormalTok{,}
  \DataTypeTok{caption.text =} \StringTok{"Note: Interpenetration distance is displayed in cm."}\NormalTok{,}
  \DataTypeTok{title.color =} \StringTok{"black"}\NormalTok{,}
  \DataTypeTok{caption.color =} \StringTok{"black"}
\NormalTok{) }

\CommentTok{# Lets check the effect of shaded condition on DVs}
\NormalTok{p5 <-}\StringTok{ }\NormalTok{ggstatsplot}\OperatorTok{::}\StringTok{ }\KeywordTok{ggbetweenstats}\NormalTok{(}
  \DataTypeTok{data =}\NormalTok{ ParsiDFplots,}
  \DataTypeTok{x =} \StringTok{"Part"}\NormalTok{,}
  \DataTypeTok{y =} \StringTok{"MaxInterpenetration"}\NormalTok{,}
  \DataTypeTok{grouping.var =} \StringTok{"InterpenetrationFeedback"}\NormalTok{,}
  \DataTypeTok{type =} \StringTok{"p"}\NormalTok{,}
  \DataTypeTok{pairwise.comparisons =} \OtherTok{FALSE}\NormalTok{,}
  \DataTypeTok{pairwise.display =} \StringTok{"significant"}\NormalTok{,}
  \DataTypeTok{p.adjust.method =} \StringTok{"bonferroni"}\NormalTok{,}
  \DataTypeTok{effsize.type =} \StringTok{"unbiased"}\NormalTok{,}
  \DataTypeTok{results.subtitle =} \OtherTok{FALSE}\NormalTok{,}
  \DataTypeTok{xlab =} \StringTok{"Order"}\NormalTok{,}
  \DataTypeTok{ylab =} \StringTok{"Maximum Interpenetration"}\NormalTok{,}
  \DataTypeTok{sample.size.label =} \OtherTok{FALSE}\NormalTok{,}
  \DataTypeTok{var.equal =} \OtherTok{TRUE}\NormalTok{,}
  \DataTypeTok{mean.plotting =} \OtherTok{FALSE}\NormalTok{,}
  \DataTypeTok{mean.ci =} \OtherTok{TRUE}\NormalTok{,}
  \DataTypeTok{paired =} \OtherTok{TRUE}\NormalTok{,}
  \DataTypeTok{title.text =} \StringTok{"Interpenetration Box-Violin Plots"}\NormalTok{,}
  \DataTypeTok{caption.text =} \StringTok{"Note: Interpenetration distance is displayed in cm."}\NormalTok{,}
  \DataTypeTok{title.color =} \StringTok{"black"}\NormalTok{,}
  \DataTypeTok{caption.color =} \StringTok{"black"}
\NormalTok{  ) }

\NormalTok{p6 <-}\StringTok{ }\NormalTok{ggstatsplot}\OperatorTok{::}\KeywordTok{ggbetweenstats}\NormalTok{(}
  \DataTypeTok{data =}\NormalTok{ ParsiDFplots,}
  \DataTypeTok{x =} \StringTok{"Part"}\NormalTok{,}
  \DataTypeTok{y =} \StringTok{"AverageInterpenetration"}\NormalTok{,}
  \DataTypeTok{grouping.var =} \StringTok{"InterpenetrationFeedback"}\NormalTok{,}
  \DataTypeTok{type =} \StringTok{"p"}\NormalTok{,}
  \DataTypeTok{pairwise.comparisons =} \OtherTok{FALSE}\NormalTok{,}
  \DataTypeTok{pairwise.display =} \StringTok{"significant"}\NormalTok{,}
  \DataTypeTok{p.adjust.method =} \StringTok{"bonferroni"}\NormalTok{,}
  \DataTypeTok{effsize.type =} \StringTok{"unbiased"}\NormalTok{,}
  \DataTypeTok{results.subtitle =} \OtherTok{FALSE}\NormalTok{,}
  \DataTypeTok{xlab =} \StringTok{"Order"}\NormalTok{,}
  \DataTypeTok{ylab =} \StringTok{"Average Interpenetration"}\NormalTok{,}
  \DataTypeTok{sample.size.label =} \OtherTok{FALSE}\NormalTok{,}
  \DataTypeTok{var.equal =} \OtherTok{TRUE}\NormalTok{,}
  \DataTypeTok{mean.plotting =} \OtherTok{FALSE}\NormalTok{,}
  \DataTypeTok{mean.ci =} \OtherTok{TRUE}\NormalTok{,}
  \DataTypeTok{paired =} \OtherTok{TRUE}\NormalTok{,}
  \DataTypeTok{title.text =} \StringTok{"Interpenetration Box-Violin Plots"}\NormalTok{,}
  \DataTypeTok{caption.text =} \StringTok{"Note: Interpenetration distance is displayed in cm."}\NormalTok{,}
  \DataTypeTok{title.color =} \StringTok{"black"}\NormalTok{,}
  \DataTypeTok{caption.color =} \StringTok{"black"}
\NormalTok{) }

\NormalTok{p7 <-}\StringTok{ }\NormalTok{ggstatsplot}\OperatorTok{::}\KeywordTok{grouped_ggbetweenstats}\NormalTok{(}
  \DataTypeTok{data =}\NormalTok{ ParsiDFplots,}
  \DataTypeTok{x =} \StringTok{"Part"}\NormalTok{,}
  \DataTypeTok{y =} \StringTok{"MaxInterpenetration"}\NormalTok{,}
  \DataTypeTok{grouping.var =} \StringTok{"InterpenetrationFeedback"}\NormalTok{,}
  \DataTypeTok{type =} \StringTok{"p"}\NormalTok{,}
  \DataTypeTok{pairwise.comparisons =} \OtherTok{FALSE}\NormalTok{,}
  \DataTypeTok{pairwise.display =} \StringTok{"significant"}\NormalTok{,}
  \DataTypeTok{p.adjust.method =} \StringTok{"bonferroni"}\NormalTok{,}
  \DataTypeTok{effsize.type =} \StringTok{"unbiased"}\NormalTok{,}
  \DataTypeTok{results.subtitle =} \OtherTok{FALSE}\NormalTok{,}
  \DataTypeTok{xlab =} \StringTok{"Order"}\NormalTok{,}
  \DataTypeTok{ylab =} \StringTok{"Maximum Interpenetration"}\NormalTok{,}
  \DataTypeTok{sample.size.label =} \OtherTok{FALSE}\NormalTok{,}
  \DataTypeTok{var.equal =} \OtherTok{TRUE}\NormalTok{,}
  \DataTypeTok{mean.plotting =} \OtherTok{FALSE}\NormalTok{,}
  \DataTypeTok{mean.ci =} \OtherTok{TRUE}\NormalTok{,}
  \DataTypeTok{paired =} \OtherTok{TRUE}\NormalTok{,}
  \DataTypeTok{title.text =} \StringTok{"Interpenetration Box-Violin Plots"}\NormalTok{,}
  \DataTypeTok{caption.text =} \StringTok{"Note: Interpenetration distance is displayed in cm."}\NormalTok{,}
  \DataTypeTok{title.color =} \StringTok{"black"}\NormalTok{,}
  \DataTypeTok{caption.color =} \StringTok{"black"}\NormalTok{) }

\NormalTok{p8 <-}\StringTok{ }\NormalTok{ggstatsplot}\OperatorTok{::}\KeywordTok{grouped_ggbetweenstats}\NormalTok{(}
  \DataTypeTok{data =}\NormalTok{ ParsiDFplots,}
  \DataTypeTok{x =} \StringTok{"Part"}\NormalTok{,}
  \DataTypeTok{y =} \StringTok{"AverageInterpenetration"}\NormalTok{,}
  \DataTypeTok{grouping.var =} \StringTok{"InterpenetrationFeedback"}\NormalTok{,}
  \DataTypeTok{type =} \StringTok{"p"}\NormalTok{,}
  \DataTypeTok{pairwise.comparisons =} \OtherTok{FALSE}\NormalTok{,}
  \DataTypeTok{pairwise.display =} \StringTok{"significant"}\NormalTok{,}
  \DataTypeTok{p.adjust.method =} \StringTok{"bonferroni"}\NormalTok{,}
  \DataTypeTok{effsize.type =} \StringTok{"unbiased"}\NormalTok{,}
  \DataTypeTok{results.subtitle =} \OtherTok{FALSE}\NormalTok{,}
  \DataTypeTok{xlab =} \StringTok{"Order"}\NormalTok{,}
  \DataTypeTok{ylab =} \StringTok{"Average Interpenetration"}\NormalTok{,}
  \DataTypeTok{sample.size.label =} \OtherTok{FALSE}\NormalTok{,}
  \DataTypeTok{var.equal =} \OtherTok{TRUE}\NormalTok{,}
  \DataTypeTok{mean.plotting =} \OtherTok{FALSE}\NormalTok{,}
  \DataTypeTok{mean.ci =} \OtherTok{TRUE}\NormalTok{,}
  \DataTypeTok{paired =} \OtherTok{TRUE}\NormalTok{,}
  \DataTypeTok{title.text =} \StringTok{"Interpenetration Box-Violin Plots"}\NormalTok{,}
  \DataTypeTok{caption.text =} \StringTok{"Note: Interpenetration distance is displayed in cm."}\NormalTok{,}
  \DataTypeTok{title.color =} \StringTok{"black"}\NormalTok{,}
  \DataTypeTok{caption.color =} \StringTok{"black"}
\NormalTok{  ) }
\NormalTok{p1}
\end{Highlighting}
\end{Shaded}

\includegraphics{Report_files/figure-latex/unnamed-chunk-5-1.pdf}

\begin{Shaded}
\begin{Highlighting}[]
\NormalTok{p2}
\end{Highlighting}
\end{Shaded}

\includegraphics{Report_files/figure-latex/unnamed-chunk-5-2.pdf}

\begin{Shaded}
\begin{Highlighting}[]
\NormalTok{p3 }
\end{Highlighting}
\end{Shaded}

\includegraphics{Report_files/figure-latex/unnamed-chunk-5-3.pdf}

\begin{Shaded}
\begin{Highlighting}[]
\NormalTok{p4}
\end{Highlighting}
\end{Shaded}

\includegraphics{Report_files/figure-latex/unnamed-chunk-5-4.pdf}

\begin{Shaded}
\begin{Highlighting}[]
\NormalTok{p5 }
\end{Highlighting}
\end{Shaded}

\includegraphics{Report_files/figure-latex/unnamed-chunk-5-5.pdf}

\begin{Shaded}
\begin{Highlighting}[]
\NormalTok{p6}
\end{Highlighting}
\end{Shaded}

\includegraphics{Report_files/figure-latex/unnamed-chunk-5-6.pdf}

\begin{Shaded}
\begin{Highlighting}[]
\NormalTok{p7}
\end{Highlighting}
\end{Shaded}

\includegraphics{Report_files/figure-latex/unnamed-chunk-5-7.pdf}

\begin{Shaded}
\begin{Highlighting}[]
\NormalTok{p8}
\end{Highlighting}
\end{Shaded}

\includegraphics{Report_files/figure-latex/unnamed-chunk-5-8.pdf}

Let's check the Two Way Repeated Measures ANOVA

\begin{Shaded}
\begin{Highlighting}[]
\NormalTok{aMax <-}\StringTok{ }\KeywordTok{aov_ez}\NormalTok{(}\StringTok{"ID"}\NormalTok{, }\StringTok{"MaxInterpenetration"}\NormalTok{, ParsiDF,}
             \DataTypeTok{within =} \KeywordTok{c}\NormalTok{(}\StringTok{"Part"}\NormalTok{, }\StringTok{"InterpenetrationFeedback"}\NormalTok{),}
             \DataTypeTok{anova_table =} \KeywordTok{list}\NormalTok{(}\DataTypeTok{es =} \StringTok{"pes"}\NormalTok{))}

\NormalTok{knitr}\OperatorTok{::}\KeywordTok{kable}\NormalTok{(}\KeywordTok{nice}\NormalTok{(aMax}\OperatorTok{$}\NormalTok{anova_table))}
\end{Highlighting}
\end{Shaded}

\begin{longtable}[]{@{}llllll@{}}
\toprule
Effect & df & MSE & F & pes & p.value\tabularnewline
\midrule
\endhead
Part & 1, 20 & 0.06 & 29.43 *** & .595 & \textless.001\tabularnewline
InterpenetrationFeedback & 1.88, 37.52 & 0.12 & 28.36 *** & .586 &
\textless.001\tabularnewline
Part:InterpenetrationFeedback & 2.06, 41.24 & 0.04 & 5.16 ** & .205 &
.009\tabularnewline
\bottomrule
\end{longtable}

\begin{Shaded}
\begin{Highlighting}[]
\NormalTok{aAv <-}\StringTok{ }\KeywordTok{aov_ez}\NormalTok{(}\StringTok{"ID"}\NormalTok{, }\StringTok{"AverageInterpenetration"}\NormalTok{, ParsiDF,}
             \DataTypeTok{within =} \KeywordTok{c}\NormalTok{(}\StringTok{"Part"}\NormalTok{, }\StringTok{"InterpenetrationFeedback"}\NormalTok{),}
             \DataTypeTok{anova_table =} \KeywordTok{list}\NormalTok{(}\DataTypeTok{es =} \StringTok{"pes"}\NormalTok{))}

\NormalTok{knitr}\OperatorTok{::}\KeywordTok{kable}\NormalTok{(}\KeywordTok{nice}\NormalTok{(aAv}\OperatorTok{$}\NormalTok{anova_table))}
\end{Highlighting}
\end{Shaded}

\begin{longtable}[]{@{}llllll@{}}
\toprule
Effect & df & MSE & F & pes & p.value\tabularnewline
\midrule
\endhead
Part & 1, 20 & 0.07 & 31.74 *** & .613 & \textless.001\tabularnewline
InterpenetrationFeedback & 1.97, 39.31 & 0.12 & 25.89 *** & .564 &
\textless.001\tabularnewline
Part:InterpenetrationFeedback & 2.11, 42.21 & 0.04 & 5.72 ** & .222 &
.006\tabularnewline
\bottomrule
\end{longtable}

\#The Effect Sizes of the above ANOVAs: 1) Max Interpenetration 2)
Average Interpenetration

\begin{Shaded}
\begin{Highlighting}[]
\NormalTok{effectsize}\OperatorTok{::}\KeywordTok{omega_squared}\NormalTok{(aMax, }\DataTypeTok{partial =} \OtherTok{TRUE}\NormalTok{, }\DataTypeTok{ci =} \FloatTok{0.95}\NormalTok{)}
\end{Highlighting}
\end{Shaded}

\begin{verbatim}
## Parameter                     | Omega2 (partial) |        95% CI
## ----------------------------------------------------------------
## Part                          |             0.56 | [ 0.24, 0.74]
## InterpenetrationFeedback      |             0.56 | [ 0.38, 0.68]
## Part:InterpenetrationFeedback |             0.16 | [-0.02, 0.32]
\end{verbatim}

\begin{Shaded}
\begin{Highlighting}[]
\NormalTok{effectsize}\OperatorTok{::}\KeywordTok{omega_squared}\NormalTok{(aAv, }\DataTypeTok{partial =} \OtherTok{TRUE}\NormalTok{, }\DataTypeTok{ci =} \FloatTok{0.95}\NormalTok{)}
\end{Highlighting}
\end{Shaded}

\begin{verbatim}
## Parameter                     | Omega2 (partial) |        95% CI
## ----------------------------------------------------------------
## Part                          |             0.58 | [ 0.26, 0.75]
## InterpenetrationFeedback      |             0.54 | [ 0.35, 0.66]
## Part:InterpenetrationFeedback |             0.18 | [-0.01, 0.34]
\end{verbatim}

\#We can see that every type of feedback as well as the
interrelationship with the part of the experiment have a large effect on
DVs!!!!!!

\#Reference for interpreting Omega Squared

\#Small effect: ω2 = 0.01;

\#Medium effect: ω2 = 0.06;

\#Large effect: ω2 = 0.14.

Let's plot the main effects (Interpenetration Feedback OR Part of the
experiment).

\begin{Shaded}
\begin{Highlighting}[]
\CommentTok{# ANOVAs just for the plots}
\NormalTok{aMaxPlots <-}\StringTok{ }\KeywordTok{aov_ez}\NormalTok{(}\StringTok{"ID"}\NormalTok{, }\StringTok{"MaxInterpenetration"}\NormalTok{, ParsiDFplots,}
             \DataTypeTok{within =} \KeywordTok{c}\NormalTok{(}\StringTok{"Part"}\NormalTok{, }\StringTok{"InterpenetrationFeedback"}\NormalTok{),}
             \DataTypeTok{anova_table =} \KeywordTok{list}\NormalTok{(}\DataTypeTok{es =} \StringTok{"pes"}\NormalTok{))}

\NormalTok{aAvPlots <-}\StringTok{ }\KeywordTok{aov_ez}\NormalTok{(}\StringTok{"ID"}\NormalTok{, }\StringTok{"AverageInterpenetration"}\NormalTok{, ParsiDFplots,}
             \DataTypeTok{within =} \KeywordTok{c}\NormalTok{(}\StringTok{"Part"}\NormalTok{, }\StringTok{"InterpenetrationFeedback"}\NormalTok{),}
             \DataTypeTok{anova_table =} \KeywordTok{list}\NormalTok{(}\DataTypeTok{es =} \StringTok{"pes"}\NormalTok{))}

\CommentTok{#plots}
\KeywordTok{afex_plot}\NormalTok{(aMaxPlots, }\DataTypeTok{x =} \StringTok{"InterpenetrationFeedback"}\NormalTok{, }\DataTypeTok{error =} \StringTok{"within"}\NormalTok{, }
                \DataTypeTok{mapping =} \KeywordTok{c}\NormalTok{(}\StringTok{"linetype"}\NormalTok{, }\StringTok{"shape"}\NormalTok{, }\StringTok{"fill"}\NormalTok{),}
                \DataTypeTok{data_geom =}\NormalTok{ ggpol}\OperatorTok{::}\NormalTok{geom_boxjitter, }
                \DataTypeTok{data_arg =} \KeywordTok{list}\NormalTok{(}\DataTypeTok{width =} \FloatTok{0.5}\NormalTok{)) }\OperatorTok{+}
\StringTok{            }\KeywordTok{ylim}\NormalTok{(}\DecValTok{0}\NormalTok{, }\DecValTok{3}\NormalTok{)}
\end{Highlighting}
\end{Shaded}

\begin{verbatim}
## NOTE: Results may be misleading due to involvement in interactions
\end{verbatim}

\includegraphics{Report_files/figure-latex/unnamed-chunk-9-1.pdf}

\begin{Shaded}
\begin{Highlighting}[]
\KeywordTok{afex_plot}\NormalTok{(aMaxPlots, }\DataTypeTok{x =} \StringTok{"Part"}\NormalTok{, }\DataTypeTok{error =} \StringTok{"within"}\NormalTok{, }
                \DataTypeTok{mapping =} \KeywordTok{c}\NormalTok{(}\StringTok{"linetype"}\NormalTok{, }\StringTok{"shape"}\NormalTok{, }\StringTok{"fill"}\NormalTok{),}
                \DataTypeTok{data_geom =}\NormalTok{ ggpol}\OperatorTok{::}\NormalTok{geom_boxjitter, }
                \DataTypeTok{data_arg =} \KeywordTok{list}\NormalTok{(}\DataTypeTok{width =} \FloatTok{0.5}\NormalTok{))  }\OperatorTok{+}
\StringTok{            }\KeywordTok{ylim}\NormalTok{(}\DecValTok{0}\NormalTok{, }\DecValTok{3}\NormalTok{)}
\end{Highlighting}
\end{Shaded}

\begin{verbatim}
## NOTE: Results may be misleading due to involvement in interactions
\end{verbatim}

\includegraphics{Report_files/figure-latex/unnamed-chunk-9-2.pdf}

\begin{Shaded}
\begin{Highlighting}[]
\KeywordTok{afex_plot}\NormalTok{(aAvPlots, }\DataTypeTok{x =} \StringTok{"InterpenetrationFeedback"}\NormalTok{, }\DataTypeTok{error =} \StringTok{"within"}\NormalTok{, }
                \DataTypeTok{mapping =} \KeywordTok{c}\NormalTok{(}\StringTok{"linetype"}\NormalTok{, }\StringTok{"shape"}\NormalTok{, }\StringTok{"fill"}\NormalTok{),}
                \DataTypeTok{data_geom =}\NormalTok{ ggpol}\OperatorTok{::}\NormalTok{geom_boxjitter, }
                \DataTypeTok{data_arg =} \KeywordTok{list}\NormalTok{(}\DataTypeTok{width =} \FloatTok{0.5}\NormalTok{)) }\OperatorTok{+}
\StringTok{            }\KeywordTok{ylim}\NormalTok{(}\DecValTok{0}\NormalTok{, }\FloatTok{2.25}\NormalTok{)}
\end{Highlighting}
\end{Shaded}

\begin{verbatim}
## NOTE: Results may be misleading due to involvement in interactions
\end{verbatim}

\includegraphics{Report_files/figure-latex/unnamed-chunk-9-3.pdf}

\begin{Shaded}
\begin{Highlighting}[]
\KeywordTok{afex_plot}\NormalTok{(aAvPlots, }\DataTypeTok{x =} \StringTok{"Part"}\NormalTok{, }\DataTypeTok{error =} \StringTok{"within"}\NormalTok{, }
                \DataTypeTok{mapping =} \KeywordTok{c}\NormalTok{(}\StringTok{"linetype"}\NormalTok{, }\StringTok{"shape"}\NormalTok{, }\StringTok{"fill"}\NormalTok{),}
                \DataTypeTok{data_geom =}\NormalTok{ ggpol}\OperatorTok{::}\NormalTok{geom_boxjitter, }
                \DataTypeTok{data_arg =} \KeywordTok{list}\NormalTok{(}\DataTypeTok{width =} \FloatTok{0.5}\NormalTok{))  }\OperatorTok{+}
\StringTok{            }\KeywordTok{ylim}\NormalTok{(}\DecValTok{0}\NormalTok{, }\FloatTok{2.25}\NormalTok{)}
\end{Highlighting}
\end{Shaded}

\begin{verbatim}
## NOTE: Results may be misleading due to involvement in interactions
\end{verbatim}

\includegraphics{Report_files/figure-latex/unnamed-chunk-9-4.pdf}

Let's plot the main interaction effects (Interpenetration Feedback AND
Part of the experiment).

\begin{Shaded}
\begin{Highlighting}[]
\KeywordTok{afex_plot}\NormalTok{(aMaxPlots, }\DataTypeTok{x =} \StringTok{"InterpenetrationFeedback"}\NormalTok{, }\DataTypeTok{trace =} \StringTok{"Part"}\NormalTok{, }\DataTypeTok{error =} \StringTok{"within"}\NormalTok{, }
                \DataTypeTok{mapping =} \KeywordTok{c}\NormalTok{(}\StringTok{"linetype"}\NormalTok{, }\StringTok{"shape"}\NormalTok{, }\StringTok{"fill"}\NormalTok{),}
                \DataTypeTok{data_geom =}\NormalTok{ ggpol}\OperatorTok{::}\NormalTok{geom_boxjitter, }
                \DataTypeTok{data_arg =} \KeywordTok{list}\NormalTok{(}\DataTypeTok{width =} \FloatTok{0.5}\NormalTok{)) }\OperatorTok{+}
\StringTok{            }\KeywordTok{ylim}\NormalTok{(}\DecValTok{0}\NormalTok{, }\DecValTok{3}\NormalTok{)}
\end{Highlighting}
\end{Shaded}

\begin{verbatim}
## Warning: Removed 2 rows containing non-finite values (stat_box_jitter).
\end{verbatim}

\includegraphics{Report_files/figure-latex/unnamed-chunk-10-1.pdf}

\begin{Shaded}
\begin{Highlighting}[]
\KeywordTok{afex_plot}\NormalTok{(aMaxPlots, }\DataTypeTok{x =} \StringTok{"Part"}\NormalTok{, }\DataTypeTok{trace =} \StringTok{"InterpenetrationFeedback"}\NormalTok{, }\DataTypeTok{error =} \StringTok{"within"}\NormalTok{, }
                \DataTypeTok{mapping =} \KeywordTok{c}\NormalTok{(}\StringTok{"linetype"}\NormalTok{, }\StringTok{"shape"}\NormalTok{, }\StringTok{"fill"}\NormalTok{),}
                \DataTypeTok{data_geom =}\NormalTok{ ggpol}\OperatorTok{::}\NormalTok{geom_boxjitter, }
                \DataTypeTok{data_arg =} \KeywordTok{list}\NormalTok{(}\DataTypeTok{width =} \FloatTok{0.5}\NormalTok{))  }\OperatorTok{+}
\StringTok{            }\KeywordTok{ylim}\NormalTok{(}\DecValTok{0}\NormalTok{, }\DecValTok{3}\NormalTok{)}
\end{Highlighting}
\end{Shaded}

\begin{verbatim}
## Warning: Removed 2 rows containing non-finite values (stat_box_jitter).
\end{verbatim}

\includegraphics{Report_files/figure-latex/unnamed-chunk-10-2.pdf}

\begin{Shaded}
\begin{Highlighting}[]
\KeywordTok{afex_plot}\NormalTok{(aAvPlots, }\DataTypeTok{x =} \StringTok{"InterpenetrationFeedback"}\NormalTok{,  }\DataTypeTok{trace =} \StringTok{"Part"}\NormalTok{, }\DataTypeTok{error =} \StringTok{"within"}\NormalTok{, }
                \DataTypeTok{mapping =} \KeywordTok{c}\NormalTok{(}\StringTok{"linetype"}\NormalTok{, }\StringTok{"shape"}\NormalTok{, }\StringTok{"fill"}\NormalTok{),}
                \DataTypeTok{data_geom =}\NormalTok{ ggpol}\OperatorTok{::}\NormalTok{geom_boxjitter, }
                \DataTypeTok{data_arg =} \KeywordTok{list}\NormalTok{(}\DataTypeTok{width =} \FloatTok{0.5}\NormalTok{)) }\OperatorTok{+}
\StringTok{            }\KeywordTok{ylim}\NormalTok{(}\DecValTok{0}\NormalTok{, }\FloatTok{2.25}\NormalTok{)}
\end{Highlighting}
\end{Shaded}

\begin{verbatim}
## Warning: Removed 1 rows containing non-finite values (stat_box_jitter).
\end{verbatim}

\includegraphics{Report_files/figure-latex/unnamed-chunk-10-3.pdf}

\begin{Shaded}
\begin{Highlighting}[]
\KeywordTok{afex_plot}\NormalTok{(aAvPlots, }\DataTypeTok{x =} \StringTok{"Part"}\NormalTok{, }\DataTypeTok{trace =} \StringTok{"InterpenetrationFeedback"}\NormalTok{, }\DataTypeTok{error =} \StringTok{"within"}\NormalTok{, }
                \DataTypeTok{mapping =} \KeywordTok{c}\NormalTok{(}\StringTok{"linetype"}\NormalTok{, }\StringTok{"shape"}\NormalTok{, }\StringTok{"fill"}\NormalTok{),}
                \DataTypeTok{data_geom =}\NormalTok{ ggpol}\OperatorTok{::}\StringTok{ }\NormalTok{geom_boxjitter, }
                \DataTypeTok{data_arg =} \KeywordTok{list}\NormalTok{(}\DataTypeTok{width =} \FloatTok{0.5}\NormalTok{))  }\OperatorTok{+}
\StringTok{            }\KeywordTok{ylim}\NormalTok{(}\DecValTok{0}\NormalTok{, }\FloatTok{2.25}\NormalTok{)}
\end{Highlighting}
\end{Shaded}

\begin{verbatim}
## Warning: Removed 1 rows containing non-finite values (stat_box_jitter).
\end{verbatim}

\includegraphics{Report_files/figure-latex/unnamed-chunk-10-4.pdf}

Post-hoc Tests

\begin{Shaded}
\begin{Highlighting}[]
\CommentTok{################################# Maximum Interpenetration #######################}
\NormalTok{aMaxemm <-}\StringTok{ }\KeywordTok{emmeans}\NormalTok{(aMax,}\OperatorTok{~}\StringTok{ }\NormalTok{Part}\OperatorTok{:}\NormalTok{InterpenetrationFeedback,}
 \DataTypeTok{method=}\StringTok{"pairwise"}\NormalTok{, }\DataTypeTok{interaction=}\OtherTok{TRUE}\NormalTok{, }\DataTypeTok{adjust =} \StringTok{"bonf"}\NormalTok{)}

\KeywordTok{pairs}\NormalTok{(aMaxemm, }\DataTypeTok{adjust =} \StringTok{"bonf"}\NormalTok{)}
\end{Highlighting}
\end{Shaded}

\begin{verbatim}
##  contrast                                      estimate     SE   df t.ratio
##  Part.1 Both - Part.2 Both                       0.2322 0.0576 69.6  4.030 
##  Part.1 Both - Part.1 Electrotactile            -0.2972 0.0701 98.0 -4.242 
##  Part.1 Both - Part.2 Electrotactile             0.0551 0.0751 99.4  0.734 
##  Part.1 Both - Part.1 NoFeedback                -0.4568 0.0701 98.0 -6.521 
##  Part.1 Both - Part.2 NoFeedback                -0.3508 0.0751 99.4 -4.669 
##  Part.1 Both - Part.1 Visual                    -0.0381 0.0701 98.0 -0.543 
##  Part.1 Both - Part.2 Visual                     0.0788 0.0751 99.4  1.049 
##  Part.2 Both - Part.1 Electrotactile            -0.5293 0.0751 99.4 -7.044 
##  Part.2 Both - Part.2 Electrotactile            -0.1770 0.0701 98.0 -2.527 
##  Part.2 Both - Part.1 NoFeedback                -0.6890 0.0751 99.4 -9.169 
##  Part.2 Both - Part.2 NoFeedback                -0.5830 0.0701 98.0 -8.322 
##  Part.2 Both - Part.1 Visual                    -0.2702 0.0751 99.4 -3.596 
##  Part.2 Both - Part.2 Visual                    -0.1534 0.0701 98.0 -2.189 
##  Part.1 Electrotactile - Part.2 Electrotactile   0.3523 0.0576 69.6  6.115 
##  Part.1 Electrotactile - Part.1 NoFeedback      -0.1596 0.0701 98.0 -2.279 
##  Part.1 Electrotactile - Part.2 NoFeedback      -0.0537 0.0751 99.4 -0.714 
##  Part.1 Electrotactile - Part.1 Visual           0.2591 0.0701 98.0  3.698 
##  Part.1 Electrotactile - Part.2 Visual           0.3760 0.0751 99.4  5.003 
##  Part.2 Electrotactile - Part.1 NoFeedback      -0.5119 0.0751 99.4 -6.813 
##  Part.2 Electrotactile - Part.2 NoFeedback      -0.4060 0.0701 98.0 -5.795 
##  Part.2 Electrotactile - Part.1 Visual          -0.0932 0.0751 99.4 -1.240 
##  Part.2 Electrotactile - Part.2 Visual           0.0237 0.0701 98.0  0.338 
##  Part.1 NoFeedback - Part.2 NoFeedback           0.1060 0.0576 69.6  1.839 
##  Part.1 NoFeedback - Part.1 Visual               0.4187 0.0701 98.0  5.977 
##  Part.1 NoFeedback - Part.2 Visual               0.5356 0.0751 99.4  7.128 
##  Part.2 NoFeedback - Part.1 Visual               0.3128 0.0751 99.4  4.162 
##  Part.2 NoFeedback - Part.2 Visual               0.4296 0.0701 98.0  6.133 
##  Part.1 Visual - Part.2 Visual                   0.1169 0.0576 69.6  2.029 
##  p.value
##  0.0039 
##  0.0014 
##  1.0000 
##  <.0001 
##  0.0003 
##  1.0000 
##  1.0000 
##  <.0001 
##  0.3668 
##  <.0001 
##  <.0001 
##  0.0141 
##  0.8667 
##  <.0001 
##  0.6958 
##  1.0000 
##  0.0100 
##  0.0001 
##  <.0001 
##  <.0001 
##  1.0000 
##  1.0000 
##  1.0000 
##  <.0001 
##  <.0001 
##  0.0019 
##  <.0001 
##  1.0000 
## 
## P value adjustment: bonferroni method for 28 tests
\end{verbatim}

\begin{Shaded}
\begin{Highlighting}[]
\KeywordTok{require}\NormalTok{(esvis)}
\end{Highlighting}
\end{Shaded}

\begin{verbatim}
## Loading required package: esvis
\end{verbatim}

\begin{Shaded}
\begin{Highlighting}[]
\NormalTok{EffectSizeMax <-}\StringTok{ }\KeywordTok{hedg_g}\NormalTok{(ParsiDF,MaxInterpenetration }\OperatorTok{~}\StringTok{ }\NormalTok{InterpenetrationFeedback }\OperatorTok{+}\StringTok{ }\NormalTok{Part, }\DataTypeTok{keep_d =} \OtherTok{FALSE}\NormalTok{) }\CommentTok{#Calculates the hedge's g per pair! }

\NormalTok{EffectSizeMax}
\end{Highlighting}
\end{Shaded}

\begin{verbatim}
## # A tibble: 56 x 5
## # Groups:   InterpenetrationFeedback_ref [4]
##    InterpenetrationFeedback~ Part_ref InterpenetrationFeedback~ Part_foc  hedg_g
##    <chr>                     <chr>    <chr>                     <chr>      <dbl>
##  1 Both                      Part 1   Both                      Part 2    0.617 
##  2 Both                      Part 1   Electrotactile            Part 1   -0.728 
##  3 Both                      Part 1   Electrotactile            Part 2    0.135 
##  4 Both                      Part 1   NoFeedback                Part 1   -1.17  
##  5 Both                      Part 1   NoFeedback                Part 2   -1.05  
##  6 Both                      Part 1   Visual                    Part 1   -0.0910
##  7 Both                      Part 1   Visual                    Part 2    0.202 
##  8 Both                      Part 2   Both                      Part 1   -0.617 
##  9 Both                      Part 2   Electrotactile            Part 1   -1.32  
## 10 Both                      Part 2   Electrotactile            Part 2   -0.441 
## # ... with 46 more rows
\end{verbatim}

\begin{Shaded}
\begin{Highlighting}[]
\CommentTok{################################# Average Interpenetration #######################}
\NormalTok{aAvemm <-}\StringTok{ }\KeywordTok{emmeans}\NormalTok{(aAv,}\OperatorTok{~}\StringTok{ }\NormalTok{Part}\OperatorTok{:}\NormalTok{InterpenetrationFeedback,}
 \DataTypeTok{method=}\StringTok{"pairwise"}\NormalTok{, }\DataTypeTok{interaction=} \OtherTok{TRUE}\NormalTok{, }\DataTypeTok{adjust =} \StringTok{"bonf"}\NormalTok{)}

\KeywordTok{pairs}\NormalTok{(aAvemm, }\DataTypeTok{adjust =} \StringTok{"bonf"}\NormalTok{)}
\end{Highlighting}
\end{Shaded}

\begin{verbatim}
##  contrast                                      estimate     SE   df t.ratio
##  Part.1 Both - Part.2 Both                      0.28574 0.0608 66.2  4.698 
##  Part.1 Both - Part.1 Electrotactile           -0.29512 0.0721 97.8 -4.095 
##  Part.1 Both - Part.2 Electrotactile            0.08345 0.0786 97.7  1.062 
##  Part.1 Both - Part.1 NoFeedback               -0.43712 0.0721 97.8 -6.066 
##  Part.1 Both - Part.2 NoFeedback               -0.31874 0.0786 97.7 -4.058 
##  Part.1 Both - Part.1 Visual                   -0.04879 0.0721 97.8 -0.677 
##  Part.1 Both - Part.2 Visual                    0.08604 0.0786 97.7  1.095 
##  Part.2 Both - Part.1 Electrotactile           -0.58085 0.0786 97.7 -7.395 
##  Part.2 Both - Part.2 Electrotactile           -0.20229 0.0721 97.8 -2.807 
##  Part.2 Both - Part.1 NoFeedback               -0.72285 0.0786 97.7 -9.202 
##  Part.2 Both - Part.2 NoFeedback               -0.60448 0.0721 97.8 -8.388 
##  Part.2 Both - Part.1 Visual                   -0.33452 0.0786 97.7 -4.259 
##  Part.2 Both - Part.2 Visual                   -0.19970 0.0721 97.8 -2.771 
##  Part.1 Electrotactile - Part.2 Electrotactile  0.37856 0.0608 66.2  6.225 
##  Part.1 Electrotactile - Part.1 NoFeedback     -0.14200 0.0721 97.8 -1.970 
##  Part.1 Electrotactile - Part.2 NoFeedback     -0.02363 0.0786 97.7 -0.301 
##  Part.1 Electrotactile - Part.1 Visual          0.24633 0.0721 97.8  3.418 
##  Part.1 Electrotactile - Part.2 Visual          0.38115 0.0786 97.7  4.852 
##  Part.2 Electrotactile - Part.1 NoFeedback     -0.52056 0.0786 97.7 -6.627 
##  Part.2 Electrotactile - Part.2 NoFeedback     -0.40219 0.0721 97.8 -5.581 
##  Part.2 Electrotactile - Part.1 Visual         -0.13223 0.0786 97.7 -1.683 
##  Part.2 Electrotactile - Part.2 Visual          0.00259 0.0721 97.8  0.036 
##  Part.1 NoFeedback - Part.2 NoFeedback          0.11838 0.0608 66.2  1.946 
##  Part.1 NoFeedback - Part.1 Visual              0.38833 0.0721 97.8  5.389 
##  Part.1 NoFeedback - Part.2 Visual              0.52315 0.0786 97.7  6.660 
##  Part.2 NoFeedback - Part.1 Visual              0.26995 0.0786 97.7  3.437 
##  Part.2 NoFeedback - Part.2 Visual              0.40478 0.0721 97.8  5.617 
##  Part.1 Visual - Part.2 Visual                  0.13482 0.0608 66.2  2.217 
##  p.value
##  0.0004 
##  0.0024 
##  1.0000 
##  <.0001 
##  0.0028 
##  1.0000 
##  1.0000 
##  <.0001 
##  0.1690 
##  <.0001 
##  <.0001 
##  0.0013 
##  0.1872 
##  <.0001 
##  1.0000 
##  1.0000 
##  0.0258 
##  0.0001 
##  <.0001 
##  <.0001 
##  1.0000 
##  1.0000 
##  1.0000 
##  <.0001 
##  <.0001 
##  0.0243 
##  <.0001 
##  0.8420 
## 
## P value adjustment: bonferroni method for 28 tests
\end{verbatim}

\begin{Shaded}
\begin{Highlighting}[]
\NormalTok{EffectSizeAv <-}\StringTok{ }\KeywordTok{hedg_g}\NormalTok{(ParsiDF,AverageInterpenetration }\OperatorTok{~}\StringTok{ }\NormalTok{InterpenetrationFeedback }\OperatorTok{+}\StringTok{ }\NormalTok{Part, }\DataTypeTok{keep_d =} \OtherTok{FALSE}\NormalTok{)}
\NormalTok{EffectSizeAv}
\end{Highlighting}
\end{Shaded}

\begin{verbatim}
## # A tibble: 56 x 5
## # Groups:   InterpenetrationFeedback_ref [4]
##    InterpenetrationFeedback_~ Part_ref InterpenetrationFeedback~ Part_foc hedg_g
##    <chr>                      <chr>    <chr>                     <chr>     <dbl>
##  1 Both                       Part 1   Both                      Part 2    0.689
##  2 Both                       Part 1   Electrotactile            Part 1   -0.674
##  3 Both                       Part 1   Electrotactile            Part 2    0.197
##  4 Both                       Part 1   NoFeedback                Part 1   -1.10 
##  5 Both                       Part 1   NoFeedback                Part 2   -0.882
##  6 Both                       Part 1   Visual                    Part 1   -0.108
##  7 Both                       Part 1   Visual                    Part 2    0.203
##  8 Both                       Part 2   Both                      Part 1   -0.689
##  9 Both                       Part 2   Electrotactile            Part 1   -1.33 
## 10 Both                       Part 2   Electrotactile            Part 2   -0.482
## # ... with 46 more rows
\end{verbatim}

\begin{Shaded}
\begin{Highlighting}[]
\CommentTok{############ Change in the Performance (Part 1 vs Part2) per Interpenetration Feedback #############}

\KeywordTok{contrast}\NormalTok{(}\KeywordTok{emmeans}\NormalTok{(aMax,}\OperatorTok{~}\StringTok{ }\NormalTok{Part}\OperatorTok{:}\NormalTok{InterpenetrationFeedback), }
         \DataTypeTok{method=}\StringTok{"pairwise"}\NormalTok{, }\DataTypeTok{interaction=}\OtherTok{TRUE}\NormalTok{, }\DataTypeTok{adjust =} \StringTok{"bonf"}\NormalTok{)}
\end{Highlighting}
\end{Shaded}

\begin{verbatim}
##  Part_pairwise   InterpenetrationFeedback_pairwise estimate     SE df t.ratio
##  Part.1 - Part.2 Both - Electrotactile              -0.1201 0.0718 60 -1.672 
##  Part.1 - Part.2 Both - NoFeedback                   0.1262 0.0718 60  1.757 
##  Part.1 - Part.2 Both - Visual                       0.1153 0.0718 60  1.605 
##  Part.1 - Part.2 Electrotactile - NoFeedback         0.2463 0.0718 60  3.429 
##  Part.1 - Part.2 Electrotactile - Visual             0.2354 0.0718 60  3.277 
##  Part.1 - Part.2 NoFeedback - Visual                -0.0109 0.0718 60 -0.152 
##  p.value
##  0.5982 
##  0.5043 
##  0.6822 
##  0.0066 
##  0.0105 
##  1.0000 
## 
## P value adjustment: bonferroni method for 6 tests
\end{verbatim}

\begin{Shaded}
\begin{Highlighting}[]
\KeywordTok{contrast}\NormalTok{(}\KeywordTok{emmeans}\NormalTok{(aAv,}\OperatorTok{~}\StringTok{ }\NormalTok{Part}\OperatorTok{:}\NormalTok{InterpenetrationFeedback), }
         \DataTypeTok{method=}\StringTok{"pairwise"}\NormalTok{, }\DataTypeTok{interaction=}\OtherTok{TRUE}\NormalTok{, }\DataTypeTok{adjust =} \StringTok{"bonf"}\NormalTok{)}
\end{Highlighting}
\end{Shaded}

\begin{verbatim}
##  Part_pairwise   InterpenetrationFeedback_pairwise estimate     SE df t.ratio
##  Part.1 - Part.2 Both - Electrotactile              -0.0928 0.0738 60 -1.258 
##  Part.1 - Part.2 Both - NoFeedback                   0.1674 0.0738 60  2.268 
##  Part.1 - Part.2 Both - Visual                       0.1509 0.0738 60  2.046 
##  Part.1 - Part.2 Electrotactile - NoFeedback         0.2602 0.0738 60  3.527 
##  Part.1 - Part.2 Electrotactile - Visual             0.2437 0.0738 60  3.304 
##  Part.1 - Part.2 NoFeedback - Visual                -0.0164 0.0738 60 -0.223 
##  p.value
##  1.0000 
##  0.1615 
##  0.2712 
##  0.0049 
##  0.0097 
##  1.0000 
## 
## P value adjustment: bonferroni method for 6 tests
\end{verbatim}

\begin{Shaded}
\begin{Highlighting}[]
\CommentTok{#Calculating the effect sizes of the significant comparisons}
\KeywordTok{require}\NormalTok{(esc)}
\end{Highlighting}
\end{Shaded}

\begin{verbatim}
## Loading required package: esc
\end{verbatim}

\begin{verbatim}
## 
## Attaching package: 'esc'
\end{verbatim}

\begin{verbatim}
## The following objects are masked from 'package:rstatix':
## 
##     cohens_d, eta_squared
\end{verbatim}

\begin{Shaded}
\begin{Highlighting}[]
\KeywordTok{hedges_g}\NormalTok{(}\DataTypeTok{d =} \FloatTok{0.63133007}\NormalTok{, }\DataTypeTok{totaln =} \DecValTok{60}\NormalTok{)}
\end{Highlighting}
\end{Shaded}

\begin{verbatim}
## [1] 0.623131
\end{verbatim}

\begin{Shaded}
\begin{Highlighting}[]
\KeywordTok{hedges_g}\NormalTok{(}\DataTypeTok{d =} \FloatTok{0.60334460}\NormalTok{, }\DataTypeTok{totaln =} \DecValTok{60}\NormalTok{)}
\end{Highlighting}
\end{Shaded}

\begin{verbatim}
## [1] 0.595509
\end{verbatim}

\begin{Shaded}
\begin{Highlighting}[]
\KeywordTok{hedges_g}\NormalTok{(}\DataTypeTok{d =} \FloatTok{0.64937334}\NormalTok{, }\DataTypeTok{totaln =} \DecValTok{60}\NormalTok{)}
\end{Highlighting}
\end{Shaded}

\begin{verbatim}
## [1] 0.6409399
\end{verbatim}

\begin{Shaded}
\begin{Highlighting}[]
\KeywordTok{hedges_g}\NormalTok{(}\DataTypeTok{d =} \FloatTok{0.60831571}\NormalTok{, }\DataTypeTok{totaln =} \DecValTok{60}\NormalTok{)}
\end{Highlighting}
\end{Shaded}

\begin{verbatim}
## [1] 0.6004155
\end{verbatim}

\hypertarget{for-maximum-interpenetration}{%
\paragraph{For Maximum
Interpenetration}\label{for-maximum-interpenetration}}

\hypertarget{significant-comparisons}{%
\paragraph{Significant Comparisons:}\label{significant-comparisons}}

\#Part.1 Both - Part.2 Both p = 0.0039 Hedge's g = 0.61676553

\#Part.1 Both - Part.1 Electrotactile p = 0.0014 Hedge's g = -0.72831779

\#Part.1 Both - Part.1 NoFeedback p \textless.0001 Hedge's g =
-1.16801037

\#Part.2 Both - Part.2 NoFeedback p \textless.0001 Hedge's g =
-1.79394003

\#Part.1 Electrotactile - Part.2 Electrotactile p \textless.0001 Hedge's
g = 0.81636215

\#Part.1 Electrotactile - Part.1 Visual p = 0.0100 Hedge's g =
0.58901263

\#Part.2 Electrotactile - Part.2 NoFeedback p \textless.0001 Hedge's g =
-1.11877925

\#Part.1 NoFeedback - Part.1 Visual p \textless.0001 Hedge's g =
0.98702238

\#Part.2 NoFeedback - Part.2 Visual p \textless.0001 Hedge's g =
1.26405828

\#Hedge's G Interpretation (Note: + or - just shows the direction!):

\#Small effect (cannot be discerned by the naked eye) = 0.2

\#Medium Effect = 0.5

\#Large Effect (can be seen by the naked eye) = 0.8

\hypertarget{for-average-interpenetration}{%
\paragraph{For Average
Interpenetration}\label{for-average-interpenetration}}

\hypertarget{significant-comparisons-1}{%
\paragraph{Significant Comparisons:}\label{significant-comparisons-1}}

\#Part.1 Both - Part.2 Both p = 0.0004 Hedge's g = 0.688559174

\#Part.1 Both - Part.1 Electrotactile p = 0.0024 Hedge's g =
-0.673535198

\#Part.1 Both - Part.1 NoFeedback p \textless{} .0001 Hedge's g =
-1.102751598

\#Part.2 Both - Part.2 NoFeedback p \textless{} .0001 Hedge's g =
-1.688918621

\#Part.1 Electrotactile - Part.2 Electrotactile p \textless{} .0001
Hedge's g = 0.855157215

\#Part.1 Electrotactile - Part.1 Visual p = 0.0258 Hedge's g =
0.524527360

\#Part.2 Electrotactile - Part.2 NoFeedback p \textless{} .0001 Hedge's
g = -1.095756543

\#Part.1 NoFeedback - Part.1 Visual p \textless{} .0001 Hedge's g =
0.901187692

\#Part.2 NoFeedback - Part.2 Visual p \textless{} .0001 Hedge's g =
1.096379269

\#Key Findings:

\#Electrotactile in Part 1 has significant differences against Visual
and combined feedback (moderate effects), while is not significant
different from No Feedback.

\#However, in part 2, Electrotactile is significant different from No
Feedback, while it does not show significant differences against Visual
and Combined feedback.

\#Both (i.e., combined) and Visual feedback are significantly different
against No Feedback in part 1 and part 2.

\#Notably, ONLY Both (combined) and Electrotactile feedback show a
significant improvement from part 1 to part 2.

\#This explains why the comparison between combined feedback against No
Feedback has far greater effect size in part 2.

\#Importantly, this explains why the Electrotactile feedback becomes
significantly different against No Feedback in part 2, as well as the
absence of differences against Visual and Combined.

\#Since the Visual and No feedback do not improve in part 2 (i.e, no
differences between part 1 and part 2), we may infer that the
practice/order effect does not affect significantly the performance.

\#Also, since the Visual feedback do not improve in part 2, then we may
infer that the significant improvement that we observe for combined
feedback in part 2 (i.e., part 1 vs part 2) is predominantly attributed
to the improvement of the electrotactile feedback.

\#Hence, either the calibration of, or the familiarization with, or
both, concerning the electrotactile feedback, is the reason that we
observe these effects of electrotactile and combined feedback in part 2.

\hypertarget{regarding-the-change-of-the-performance-for-each-type-of-interpenetration-feedback-between-part-1-and-part-2-of-the-experiment}{%
\subparagraph{Regarding the change of the performance for each type of
interpenetration feedback between part 1 and part 2 of the
experiment!}\label{regarding-the-change-of-the-performance-for-each-type-of-interpenetration-feedback-between-part-1-and-part-2-of-the-experiment}}

\hypertarget{we-can-see-that-for-max-interpenetration-the-significant-comparisons-are}{%
\paragraph{we can see that for MAX INTERPENETRATION the significant
comparisons
are:}\label{we-can-see-that-for-max-interpenetration-the-significant-comparisons-are}}

\#1) Part.1 - Part.2 Electrotactile - NoFeedback p = 0.0066 Hedge's g =
0.623131

\#2) Part.1 - Part.2 Electrotactile - Visual p = 0.0105 Hedge's g =
0.595509

\#This means that the improvement of the performance from part 1 to part
2 regarding the maximum interpenetration was significantly greater for
the ``Electrotactile Feedback'' compared to the ``No Feedback'' and
``Visual Feedback'' respectively! (Medium to Large effects)

\#The rest of the comparisons were insignificant!

\#Hedge's G Interpretation (Note: + or - just shows the direction!):

\#Small effect (cannot be discerned by the naked eye) = 0.2

\#Medium Effect = 0.5

\#Large Effect (can be seen by the naked eye) = 0.8

\#For the AVERAGE INTERPENETRATION the significant comparisons are:

\#1) Part.1 - Part.2 Electrotactile - NoFeedback p = 0.0049 Hedge's g =
0.6409399

\#2) Part.1 - Part.2 Electrotactile - Visual p = 0.0097 Hedge's g =
0.6004155

\#This means that the improvement of the performance from part 1 to part
2 regarding the average interpenetration was significantly greater for
the ``Electrotactile Feedback'' compared to the ``No Feedback'' and
``Visual Feedback'' respectively! (Medium to Large effects)

\#The rest of the comparisons were insignificant!

Intensities of the Electrotactile Feedback

\begin{Shaded}
\begin{Highlighting}[]
\KeywordTok{library}\NormalTok{(readr)}

\NormalTok{intensities <-}\StringTok{ }\KeywordTok{read_csv}\NormalTok{(}\StringTok{"C:/repos/contactExperimentRNotebook/intensities.csv"}\NormalTok{)}
\end{Highlighting}
\end{Shaded}

\begin{verbatim}
## 
## -- Column specification --------------------------------------------------------
## cols(
##   ParticipantID = col_double(),
##   Calibration = col_character(),
##   Sensation = col_double(),
##   Pain = col_double(),
##   ActualValue = col_double()
## )
\end{verbatim}

\begin{Shaded}
\begin{Highlighting}[]
\KeywordTok{pairwise_t_test}\NormalTok{(}\DataTypeTok{data =}\NormalTok{ intensities, Sensation }\OperatorTok{~}\StringTok{ }\NormalTok{Calibration, }\DataTypeTok{p.adjust.method =} \StringTok{"bonferroni"}\NormalTok{, }\DataTypeTok{paired =} \OtherTok{TRUE}\NormalTok{, }\DataTypeTok{alternative =} \StringTok{"two.sided"}\NormalTok{, }\DataTypeTok{detailed =} \OtherTok{TRUE}\NormalTok{)}
\end{Highlighting}
\end{Shaded}

\begin{verbatim}
## # A tibble: 3 x 15
##   estimate .y.   group1 group2    n1    n2 statistic       p    df conf.low
## *    <dbl> <chr> <chr>  <chr>  <int> <int>     <dbl>   <dbl> <dbl>    <dbl>
## 1   0.617  Sens~ final  initi~    24    24     4.50  1.63e-4    23    0.333
## 2   0.0542 Sens~ final  middle    24    24     0.414 6.83e-1    23   -0.217
## 3  -0.562  Sens~ initi~ middle    24    24    -8.47  1.60e-8    23   -0.700
## # ... with 5 more variables: conf.high <dbl>, method <chr>, alternative <chr>,
## #   p.adj <dbl>, p.adj.signif <chr>
\end{verbatim}

\begin{Shaded}
\begin{Highlighting}[]
\KeywordTok{pairwise_t_test}\NormalTok{(}\DataTypeTok{data =}\NormalTok{ intensities, Pain }\OperatorTok{~}\StringTok{ }\NormalTok{Calibration, }\DataTypeTok{p.adjust.method =} \StringTok{"bonferroni"}\NormalTok{, }\DataTypeTok{paired =} \OtherTok{TRUE}\NormalTok{, }\DataTypeTok{alternative =} \StringTok{"two.sided"}\NormalTok{, }\DataTypeTok{detailed =} \OtherTok{TRUE}\NormalTok{)}
\end{Highlighting}
\end{Shaded}

\begin{verbatim}
## # A tibble: 3 x 15
##   estimate .y.   group1 group2    n1    n2 statistic       p    df conf.low
## *    <dbl> <chr> <chr>  <chr>  <int> <int>     <dbl>   <dbl> <dbl>    <dbl>
## 1    0.938 Pain  final  initi~    24    24     4.63  1.17e-4    23    0.519
## 2   -0.129 Pain  final  middle    24    24    -0.647 5.24e-1    23   -0.542
## 3   -1.07  Pain  initi~ middle    24    24    -4.83  7.10e-5    23   -1.52 
## # ... with 5 more variables: conf.high <dbl>, method <chr>, alternative <chr>,
## #   p.adj <dbl>, p.adj.signif <chr>
\end{verbatim}

\begin{Shaded}
\begin{Highlighting}[]
\KeywordTok{pairwise_t_test}\NormalTok{(}\DataTypeTok{data =}\NormalTok{ intensities, ActualValue }\OperatorTok{~}\StringTok{ }\NormalTok{Calibration, }\DataTypeTok{p.adjust.method =} \StringTok{"bonferroni"}\NormalTok{, }\DataTypeTok{paired =} \OtherTok{TRUE}\NormalTok{, }\DataTypeTok{alternative =} \StringTok{"two.sided"}\NormalTok{, }\DataTypeTok{detailed =} \OtherTok{TRUE}\NormalTok{)}
\end{Highlighting}
\end{Shaded}

\begin{verbatim}
## # A tibble: 3 x 15
##   estimate .y.   group1 group2    n1    n2 statistic       p    df conf.low
## *    <dbl> <chr> <chr>  <chr>  <int> <int>     <dbl>   <dbl> <dbl>    <dbl>
## 1   0.817  Actu~ final  initi~    24    24     5.20  2.87e-5    23    0.492
## 2  -0.0458 Actu~ final  middle    24    24    -0.300 7.67e-1    23   -0.362
## 3  -0.862  Actu~ initi~ middle    24    24    -5.90  5.22e-6    23   -1.17 
## # ... with 5 more variables: conf.high <dbl>, method <chr>, alternative <chr>,
## #   p.adj <dbl>, p.adj.signif <chr>
\end{verbatim}

\#OK, for every DV (i.e., Sensation, Pain , and Actual Value) we have
the same results

\#Significant differences between

\#Final and Initial !!!!!!

\#Middle and Initial !!!!!!

\#Non-Significant differences between

\#Middle and Final calibration

\#I interpret them as follows: During the 1st part (i.e., until the
middle calibration ) the individuals get familiarized with the
electrotactile feedback. Then, the variation drops significantly! So,
the most reliable calibration appears the middle one (considering that
the final doesnt differ, so it seems redundant). Importantly, these
results utterly support the familiarization hypothesis regarding the
results of the ANOVAs and Comparisons between the feedback types, as
well as it explains why the electrotactile and both have greater effect
size on the 2nd part!!!

\#Let's check the effect sizes of the significant comparisons (i.e,
Final vs Initial, Midlle vs Initial) now.

\begin{Shaded}
\begin{Highlighting}[]
\NormalTok{MidVSInitial <-}\StringTok{ }\KeywordTok{filter}\NormalTok{(intensities, Calibration }\OperatorTok{!=}\StringTok{ "final"}\NormalTok{)}

\NormalTok{FinalVSInitial <-}\StringTok{ }\KeywordTok{filter}\NormalTok{(intensities, Calibration }\OperatorTok{!=}\StringTok{ "middle"}\NormalTok{)}

\NormalTok{effectsize}\OperatorTok{::}\KeywordTok{hedges_g}\NormalTok{(}\StringTok{"Sensation"}\NormalTok{, }\StringTok{"Calibration"}\NormalTok{, }\DataTypeTok{data =}\NormalTok{ MidVSInitial, }\DataTypeTok{correction =} \OtherTok{TRUE}\NormalTok{, }\DataTypeTok{paired =} \OtherTok{TRUE}\NormalTok{,)}
\end{Highlighting}
\end{Shaded}

\begin{verbatim}
## Hedge's g |         95% CI
## --------------------------
##     -1.55 | [-2.15, -0.99]
\end{verbatim}

\begin{Shaded}
\begin{Highlighting}[]
\NormalTok{effectsize}\OperatorTok{::}\KeywordTok{hedges_g}\NormalTok{(}\StringTok{"Pain"}\NormalTok{, }\StringTok{"Calibration"}\NormalTok{, }\DataTypeTok{data =}\NormalTok{ MidVSInitial, }\DataTypeTok{correction =} \OtherTok{TRUE}\NormalTok{, }\DataTypeTok{paired =} \OtherTok{TRUE}\NormalTok{,)}
\end{Highlighting}
\end{Shaded}

\begin{verbatim}
## Hedge's g |         95% CI
## --------------------------
##     -0.88 | [-1.34, -0.45]
\end{verbatim}

\begin{Shaded}
\begin{Highlighting}[]
\NormalTok{effectsize}\OperatorTok{::}\KeywordTok{hedges_g}\NormalTok{(}\StringTok{"ActualValue"}\NormalTok{, }\StringTok{"Calibration"}\NormalTok{, }\DataTypeTok{data =}\NormalTok{ MidVSInitial, }\DataTypeTok{correction =} \OtherTok{TRUE}\NormalTok{, }\DataTypeTok{paired =} \OtherTok{TRUE}\NormalTok{,)}
\end{Highlighting}
\end{Shaded}

\begin{verbatim}
## Hedge's g |         95% CI
## --------------------------
##     -1.08 | [-1.57, -0.61]
\end{verbatim}

\begin{Shaded}
\begin{Highlighting}[]
\NormalTok{effectsize}\OperatorTok{::}\KeywordTok{hedges_g}\NormalTok{(}\StringTok{"Sensation"}\NormalTok{, }\StringTok{"Calibration"}\NormalTok{, }\DataTypeTok{data =}\NormalTok{ FinalVSInitial, }\DataTypeTok{correction =} \OtherTok{TRUE}\NormalTok{, }\DataTypeTok{paired =} \OtherTok{TRUE}\NormalTok{,)}
\end{Highlighting}
\end{Shaded}

\begin{verbatim}
## Hedge's g |       95% CI
## ------------------------
##      0.82 | [0.39, 1.27]
\end{verbatim}

\begin{Shaded}
\begin{Highlighting}[]
\NormalTok{effectsize}\OperatorTok{::}\KeywordTok{hedges_g}\NormalTok{(}\StringTok{"Pain"}\NormalTok{, }\StringTok{"Calibration"}\NormalTok{, }\DataTypeTok{data =}\NormalTok{ FinalVSInitial, }\DataTypeTok{correction =} \OtherTok{TRUE}\NormalTok{, }\DataTypeTok{paired =} \OtherTok{TRUE}\NormalTok{,)}
\end{Highlighting}
\end{Shaded}

\begin{verbatim}
## Hedge's g |       95% CI
## ------------------------
##      0.85 | [0.42, 1.30]
\end{verbatim}

\begin{Shaded}
\begin{Highlighting}[]
\NormalTok{effectsize}\OperatorTok{::}\KeywordTok{hedges_g}\NormalTok{(}\StringTok{"ActualValue"}\NormalTok{, }\StringTok{"Calibration"}\NormalTok{, }\DataTypeTok{data =}\NormalTok{ FinalVSInitial, }\DataTypeTok{correction =} \OtherTok{TRUE}\NormalTok{, }\DataTypeTok{paired =} \OtherTok{TRUE}\NormalTok{,)}
\end{Highlighting}
\end{Shaded}

\begin{verbatim}
## Hedge's g |       95% CI
## ------------------------
##      0.95 | [0.50, 1.42]
\end{verbatim}

\#The + or - is just the direction of the SDs change proportionally to
how the comparison is called (e..g, Final vs Initial or Initial vs
Final) so just ignore it.

\#The important is the value of hedges g.

\#We have a Large (or a Very large in some) Effect in every significant
comparison.

\#Hedge's G Interpretation:

\#Small effect (cannot be discerned by the naked eye) = 0.2

\#Medium Effect = 0.5

\#Large Effect (can be seen by the naked eye) = 0.8

Let's visualize the comparisons

\begin{Shaded}
\begin{Highlighting}[]
\NormalTok{intensities}\OperatorTok{$}\NormalTok{Calibration <-}\StringTok{ }\KeywordTok{as.ordered}\NormalTok{(intensities}\OperatorTok{$}\NormalTok{Calibration)}
\KeywordTok{levels}\NormalTok{(intensities}\OperatorTok{$}\NormalTok{Calibration)}
\end{Highlighting}
\end{Shaded}

\begin{verbatim}
## [1] "final"   "initial" "middle"
\end{verbatim}

\begin{Shaded}
\begin{Highlighting}[]
\NormalTok{intensities}\OperatorTok{$}\NormalTok{Calibration <-}\StringTok{ }\KeywordTok{factor}\NormalTok{(intensities}\OperatorTok{$}\NormalTok{Calibration,}\DataTypeTok{levels =} \KeywordTok{c}\NormalTok{(}\StringTok{"initial"}\NormalTok{,}\StringTok{"middle"}\NormalTok{, }\StringTok{"final"}\NormalTok{),}
                  \DataTypeTok{labels =} \KeywordTok{c}\NormalTok{(}\StringTok{"initial"}\NormalTok{,}\StringTok{"middle"}\NormalTok{, }\StringTok{"final"}\NormalTok{))}
\KeywordTok{ordered}\NormalTok{(intensities}\OperatorTok{$}\NormalTok{Calibration)}
\end{Highlighting}
\end{Shaded}

\begin{verbatim}
##  [1] initial middle  final   initial middle  final   initial middle  final  
## [10] initial middle  final   initial middle  final   initial middle  final  
## [19] initial middle  final   initial middle  final   initial middle  final  
## [28] initial middle  final   initial middle  final   initial middle  final  
## [37] initial middle  final   initial middle  final   initial middle  final  
## [46] initial middle  final   initial middle  final   initial middle  final  
## [55] initial middle  final   initial middle  final   initial middle  final  
## [64] initial middle  final   initial middle  final   initial middle  final  
## Levels: initial < middle < final
\end{verbatim}

\begin{Shaded}
\begin{Highlighting}[]
\NormalTok{p9 <-}\StringTok{ }\NormalTok{ggstatsplot}\OperatorTok{::}\KeywordTok{ggbetweenstats}\NormalTok{(}
  \DataTypeTok{data =}\NormalTok{ intensities,}
  \DataTypeTok{x =} \StringTok{"Calibration"}\NormalTok{,}
  \DataTypeTok{y =} \StringTok{"Sensation"}\NormalTok{,}
  \DataTypeTok{type =} \StringTok{"p"}\NormalTok{,}
  \DataTypeTok{pairwise.comparisons =} \OtherTok{FALSE}\NormalTok{,}
  \DataTypeTok{pairwise.display =} \StringTok{"significant"}\NormalTok{,}
  \DataTypeTok{p.adjust.method =} \StringTok{"bonferroni"}\NormalTok{,}
  \DataTypeTok{effsize.type =} \StringTok{"unbiased"}\NormalTok{,}
  \DataTypeTok{results.subtitle =} \OtherTok{FALSE}\NormalTok{,}
  \DataTypeTok{xlab =} \StringTok{"Calibration"}\NormalTok{,}
  \DataTypeTok{ylab =} \StringTok{"Sensation Threshold"}\NormalTok{,}
  \DataTypeTok{sample.size.label =} \OtherTok{FALSE}\NormalTok{,}
  \DataTypeTok{var.equal =} \OtherTok{TRUE}\NormalTok{,}
  \DataTypeTok{mean.plotting =} \OtherTok{FALSE}\NormalTok{,}
  \DataTypeTok{mean.ci =} \OtherTok{TRUE}\NormalTok{,}
  \DataTypeTok{paired =} \OtherTok{TRUE}\NormalTok{,}
  \DataTypeTok{title.text =} \StringTok{"Sensation Threshold Per Calibration Stage Box-Violin Plots"}\NormalTok{,}
  \DataTypeTok{title.color =} \StringTok{"black"}\NormalTok{,}
  \DataTypeTok{caption.color =} \StringTok{"black"}
\NormalTok{  )}
\NormalTok{p9}
\end{Highlighting}
\end{Shaded}

\includegraphics{Report_files/figure-latex/unnamed-chunk-14-1.pdf}

\begin{Shaded}
\begin{Highlighting}[]
\NormalTok{p10 <-}\StringTok{ }\NormalTok{ggstatsplot}\OperatorTok{::}\KeywordTok{ggbetweenstats}\NormalTok{(}
  \DataTypeTok{data =}\NormalTok{ intensities,}
  \DataTypeTok{x =} \StringTok{"Calibration"}\NormalTok{,}
  \DataTypeTok{y =} \StringTok{"Pain"}\NormalTok{,}
  \DataTypeTok{type =} \StringTok{"p"}\NormalTok{,}
  \DataTypeTok{pairwise.comparisons =} \OtherTok{FALSE}\NormalTok{,}
  \DataTypeTok{pairwise.display =} \StringTok{"significant"}\NormalTok{,}
  \DataTypeTok{p.adjust.method =} \StringTok{"bonferroni"}\NormalTok{,}
  \DataTypeTok{effsize.type =} \StringTok{"unbiased"}\NormalTok{,}
  \DataTypeTok{results.subtitle =} \OtherTok{FALSE}\NormalTok{,}
  \DataTypeTok{xlab =} \StringTok{"Calibration"}\NormalTok{,}
  \DataTypeTok{ylab =} \StringTok{"Pain Threshold"}\NormalTok{,}
  \DataTypeTok{sample.size.label =} \OtherTok{FALSE}\NormalTok{,}
  \DataTypeTok{var.equal =} \OtherTok{TRUE}\NormalTok{,}
  \DataTypeTok{mean.plotting =} \OtherTok{FALSE}\NormalTok{,}
  \DataTypeTok{mean.ci =} \OtherTok{TRUE}\NormalTok{,}
  \DataTypeTok{paired =} \OtherTok{TRUE}\NormalTok{,}
  \DataTypeTok{title.text =} \StringTok{"Pain Threshold Per Calibration Stage Box-Violin Plots"}\NormalTok{,}
  \DataTypeTok{title.color =} \StringTok{"black"}\NormalTok{,}
  \DataTypeTok{caption.color =} \StringTok{"black"}
\NormalTok{  )}
\NormalTok{p10}
\end{Highlighting}
\end{Shaded}

\includegraphics{Report_files/figure-latex/unnamed-chunk-14-2.pdf}

\begin{Shaded}
\begin{Highlighting}[]
\NormalTok{p11 <-}\StringTok{ }\NormalTok{ggstatsplot}\OperatorTok{::}\KeywordTok{ggbetweenstats}\NormalTok{(}
  \DataTypeTok{data =}\NormalTok{ intensities,}
  \DataTypeTok{x =} \StringTok{"Calibration"}\NormalTok{,}
  \DataTypeTok{y =} \StringTok{"ActualValue"}\NormalTok{,}
  \DataTypeTok{type =} \StringTok{"p"}\NormalTok{,}
  \DataTypeTok{pairwise.comparisons =} \OtherTok{FALSE}\NormalTok{,}
  \DataTypeTok{pairwise.display =} \StringTok{"significant"}\NormalTok{,}
  \DataTypeTok{p.adjust.method =} \StringTok{"bonferroni"}\NormalTok{,}
  \DataTypeTok{effsize.type =} \StringTok{"unbiased"}\NormalTok{,}
  \DataTypeTok{results.subtitle =} \OtherTok{FALSE}\NormalTok{,}
  \DataTypeTok{xlab =} \StringTok{"Calibration"}\NormalTok{,}
  \DataTypeTok{ylab =} \StringTok{"Actual Value"}\NormalTok{,}
  \DataTypeTok{sample.size.label =} \OtherTok{FALSE}\NormalTok{,}
  \DataTypeTok{var.equal =} \OtherTok{TRUE}\NormalTok{,}
  \DataTypeTok{mean.plotting =} \OtherTok{FALSE}\NormalTok{,}
  \DataTypeTok{mean.ci =} \OtherTok{TRUE}\NormalTok{,}
  \DataTypeTok{paired =} \OtherTok{TRUE}\NormalTok{,}
  \DataTypeTok{title.text =} \StringTok{"Actual Value Per Calibration Stage Box-Violin Plots"}\NormalTok{,}
  \DataTypeTok{title.color =} \StringTok{"black"}\NormalTok{,}
  \DataTypeTok{caption.color =} \StringTok{"black"}
\NormalTok{  )}
\NormalTok{p11}
\end{Highlighting}
\end{Shaded}

\includegraphics{Report_files/figure-latex/unnamed-chunk-14-3.pdf}

Checking if the results on the performance may be due to ineffective
initial calibration and not due to familiriazation with electrotactile
feedback.

\begin{Shaded}
\begin{Highlighting}[]
\KeywordTok{require}\NormalTok{(data.table)}
\KeywordTok{setDT}\NormalTok{(intensities)}

\NormalTok{intensities_wide <-}\StringTok{ }\KeywordTok{dcast}\NormalTok{(intensities,ParticipantID }\OperatorTok{~}\StringTok{ }\NormalTok{Calibration, }\DataTypeTok{value.var=}\KeywordTok{c}\NormalTok{(}\StringTok{"Sensation"}\NormalTok{, }\StringTok{"Pain"}\NormalTok{,}\StringTok{"ActualValue"}\NormalTok{))}

\NormalTok{intensities_wide}\OperatorTok{$}\NormalTok{DiffSensation <-}\StringTok{ }\NormalTok{intensities_wide}\OperatorTok{$}\NormalTok{Sensation_middle }\OperatorTok{-}\StringTok{ }\NormalTok{intensities_wide}\OperatorTok{$}\NormalTok{Sensation_initial}

\NormalTok{intensities_wide}\OperatorTok{$}\NormalTok{DiffPain <-}\StringTok{ }\NormalTok{intensities_wide}\OperatorTok{$}\NormalTok{Pain_middle }\OperatorTok{-}\StringTok{ }\NormalTok{intensities_wide}\OperatorTok{$}\NormalTok{Pain_initial}

\NormalTok{intensities_wide}\OperatorTok{$}\NormalTok{DiffActual <-}\StringTok{ }\NormalTok{intensities_wide}\OperatorTok{$}\NormalTok{ActualValue_middle }\OperatorTok{-}\StringTok{ }\NormalTok{intensities_wide}\OperatorTok{$}\NormalTok{ActualValue_initial}



\KeywordTok{identify_outliers}\NormalTok{(intensities_wide, }\DataTypeTok{variable =} \StringTok{"DiffSensation"}\NormalTok{, }\DataTypeTok{coef =} \FloatTok{1.5}\NormalTok{)}
\end{Highlighting}
\end{Shaded}

\begin{verbatim}
##    ParticipantID Sensation_initial Sensation_middle Sensation_final
## 1:            14               2.2              3.5             1.7
##    Pain_initial Pain_middle Pain_final ActualValue_initial ActualValue_middle
## 1:          4.1         4.6        3.8                 3.3                4.1
##    ActualValue_final DiffSensation DiffPain DiffActual is.outlier is.extreme
## 1:                 3           1.3      0.5        0.8       TRUE      FALSE
\end{verbatim}

\begin{Shaded}
\begin{Highlighting}[]
\KeywordTok{identify_outliers}\NormalTok{(intensities_wide, }\DataTypeTok{variable =} \StringTok{"DiffPain"}\NormalTok{, }\DataTypeTok{coef =} \FloatTok{1.5}\NormalTok{)}
\end{Highlighting}
\end{Shaded}

\begin{verbatim}
##    ParticipantID Sensation_initial Sensation_middle Sensation_final
## 1:            12               3.1              3.7             5.1
## 2:            19               2.1              3.2             2.7
##    Pain_initial Pain_middle Pain_final ActualValue_initial ActualValue_middle
## 1:          4.5         7.3        9.0                 3.9                5.9
## 2:          4.3         9.0        5.1                 3.4                6.7
##    ActualValue_final DiffSensation DiffPain DiffActual is.outlier is.extreme
## 1:               7.4           0.6      2.8        2.0       TRUE      FALSE
## 2:               4.1           1.1      4.7        3.3       TRUE       TRUE
\end{verbatim}

\begin{Shaded}
\begin{Highlighting}[]
\KeywordTok{identify_outliers}\NormalTok{(intensities_wide, }\DataTypeTok{variable =} \StringTok{"DiffActual"}\NormalTok{, }\DataTypeTok{coef =} \FloatTok{1.5}\NormalTok{)}
\end{Highlighting}
\end{Shaded}

\begin{verbatim}
##    ParticipantID Sensation_initial Sensation_middle Sensation_final
## 1:            12               3.1              3.7             5.1
## 2:            19               2.1              3.2             2.7
##    Pain_initial Pain_middle Pain_final ActualValue_initial ActualValue_middle
## 1:          4.5         7.3        9.0                 3.9                5.9
## 2:          4.3         9.0        5.1                 3.4                6.7
##    ActualValue_final DiffSensation DiffPain DiffActual is.outlier is.extreme
## 1:               7.4           0.6      2.8        2.0       TRUE      FALSE
## 2:               4.1           1.1      4.7        3.3       TRUE       TRUE
\end{verbatim}

\begin{Shaded}
\begin{Highlighting}[]
\KeywordTok{require}\NormalTok{(psych)}
\KeywordTok{describe.by}\NormalTok{(intensities_wide}\OperatorTok{$}\NormalTok{DiffSensation)}
\end{Highlighting}
\end{Shaded}

\begin{verbatim}
## Warning: describe.by is deprecated. Please use the describeBy function
\end{verbatim}

\begin{verbatim}
## Warning in describeBy(x = x, group = group, mat = mat, type = type, ...): no
## grouping variable requested
\end{verbatim}

\begin{verbatim}
##    vars  n mean   sd median trimmed mad  min max range skew kurtosis   se
## X1    1 24 0.56 0.33    0.5    0.54 0.3 -0.1 1.3   1.4 0.55    -0.05 0.07
\end{verbatim}

\begin{Shaded}
\begin{Highlighting}[]
\KeywordTok{describe.by}\NormalTok{(intensities_wide}\OperatorTok{$}\NormalTok{DiffPain)}
\end{Highlighting}
\end{Shaded}

\begin{verbatim}
## Warning: describe.by is deprecated. Please use the describeBy function

## Warning: no grouping variable requested
\end{verbatim}

\begin{verbatim}
##    vars  n mean   sd median trimmed  mad  min max range skew kurtosis   se
## X1    1 24 1.07 1.08    0.9    0.92 0.59 -0.2 4.7   4.9 1.61     2.92 0.22
\end{verbatim}

\begin{Shaded}
\begin{Highlighting}[]
\KeywordTok{describe.by}\NormalTok{(intensities_wide}\OperatorTok{$}\NormalTok{DiffActual)}
\end{Highlighting}
\end{Shaded}

\begin{verbatim}
## Warning: describe.by is deprecated. Please use the describeBy function

## Warning: no grouping variable requested
\end{verbatim}

\begin{verbatim}
##    vars  n mean   sd median trimmed  mad  min max range skew kurtosis   se
## X1    1 24 0.86 0.72    0.8    0.78 0.59 -0.2 3.3   3.5 1.55     3.26 0.15
\end{verbatim}

\begin{Shaded}
\begin{Highlighting}[]
\CommentTok{#IDs 12 and 19 seem suspicious, however the rest are seem ok}
\CommentTok{#lets exclude them and rerun the analyses}

\NormalTok{intensities}\OperatorTok{$}\NormalTok{ParticipantID[intensities}\OperatorTok{$}\NormalTok{ParticipantID }\OperatorTok{==}\StringTok{ }\DecValTok{19}\NormalTok{] <-}\StringTok{ }\OtherTok{NA}
\NormalTok{intensities}\OperatorTok{$}\NormalTok{ParticipantID[intensities}\OperatorTok{$}\NormalTok{ParticipantID }\OperatorTok{==}\StringTok{ }\DecValTok{12}\NormalTok{] <-}\StringTok{ }\OtherTok{NA}

\NormalTok{intensities <-}\StringTok{ }\KeywordTok{na.omit}\NormalTok{(intensities)}

\KeywordTok{pairwise_t_test}\NormalTok{(}\DataTypeTok{data =}\NormalTok{ intensities, Sensation }\OperatorTok{~}\StringTok{ }\NormalTok{Calibration, }\DataTypeTok{p.adjust.method =} \StringTok{"bonferroni"}\NormalTok{, }\DataTypeTok{paired =} \OtherTok{TRUE}\NormalTok{, }\DataTypeTok{alternative =} \StringTok{"two.sided"}\NormalTok{, }\DataTypeTok{detailed =} \OtherTok{TRUE}\NormalTok{)}
\end{Highlighting}
\end{Shaded}

\begin{verbatim}
## # A tibble: 3 x 15
##   estimate .y.   group1 group2    n1    n2 statistic       p    df conf.low
## *    <dbl> <chr> <chr>  <chr>  <int> <int>     <dbl>   <dbl> <dbl>    <dbl>
## 1  -0.536  Sens~ initi~ middle    22    22    -7.90  1.01e-7    21   -0.678
## 2  -0.555  Sens~ initi~ final     22    22    -4.12  4.89e-4    21   -0.835
## 3  -0.0182 Sens~ middle final     22    22    -0.145 8.86e-1    21   -0.280
## # ... with 5 more variables: conf.high <dbl>, method <chr>, alternative <chr>,
## #   p.adj <dbl>, p.adj.signif <chr>
\end{verbatim}

\begin{Shaded}
\begin{Highlighting}[]
\KeywordTok{pairwise_t_test}\NormalTok{(}\DataTypeTok{data =}\NormalTok{ intensities, Pain }\OperatorTok{~}\StringTok{ }\NormalTok{Calibration, }\DataTypeTok{p.adjust.method =} \StringTok{"bonferroni"}\NormalTok{, }\DataTypeTok{paired =} \OtherTok{TRUE}\NormalTok{, }\DataTypeTok{alternative =} \StringTok{"two.sided"}\NormalTok{, }\DataTypeTok{detailed =} \OtherTok{TRUE}\NormalTok{)}
\end{Highlighting}
\end{Shaded}

\begin{verbatim}
## # A tibble: 3 x 15
##   estimate .y.   group1 group2    n1    n2 statistic       p    df conf.low
## *    <dbl> <chr> <chr>  <chr>  <int> <int>     <dbl>   <dbl> <dbl>    <dbl>
## 1  -0.823  Pain  initi~ middle    22    22    -5.77  1.00e-5    21   -1.12 
## 2  -0.782  Pain  initi~ final     22    22    -5.49  1.91e-5    21   -1.08 
## 3   0.0409 Pain  middle final     22    22     0.424 6.76e-1    21   -0.160
## # ... with 5 more variables: conf.high <dbl>, method <chr>, alternative <chr>,
## #   p.adj <dbl>, p.adj.signif <chr>
\end{verbatim}

\begin{Shaded}
\begin{Highlighting}[]
\KeywordTok{pairwise_t_test}\NormalTok{(}\DataTypeTok{data =}\NormalTok{ intensities, ActualValue }\OperatorTok{~}\StringTok{ }\NormalTok{Calibration, }\DataTypeTok{p.adjust.method =} \StringTok{"bonferroni"}\NormalTok{, }\DataTypeTok{paired =} \OtherTok{TRUE}\NormalTok{, }\DataTypeTok{alternative =} \StringTok{"two.sided"}\NormalTok{, }\DataTypeTok{detailed =} \OtherTok{TRUE}\NormalTok{)}
\end{Highlighting}
\end{Shaded}

\begin{verbatim}
## # A tibble: 3 x 15
##    estimate .y.   group1 group2    n1    n2 statistic       p    df conf.low
## *     <dbl> <chr> <chr>  <chr>  <int> <int>     <dbl>   <dbl> <dbl>    <dbl>
## 1 -7.00e- 1 Actu~ initi~ middle    22    22 -7.52e+ 0 2.18e-7    21   -0.894
## 2 -7.00e- 1 Actu~ initi~ final     22    22 -6.08e+ 0 4.91e-6    21   -0.939
## 3 -2.02e-17 Actu~ middle final     22    22 -2.19e-16 1.00e+0    21   -0.192
## # ... with 5 more variables: conf.high <dbl>, method <chr>, alternative <chr>,
## #   p.adj <dbl>, p.adj.signif <chr>
\end{verbatim}

\begin{Shaded}
\begin{Highlighting}[]
\CommentTok{#OK, we have similar results, lets go a wee bit farther}
\CommentTok{# I will also exclude the IDs which were not included in the performance analyses. }

\NormalTok{intensities}\OperatorTok{$}\NormalTok{ParticipantID[intensities}\OperatorTok{$}\NormalTok{ParticipantID }\OperatorTok{==}\StringTok{ }\DecValTok{9}\NormalTok{] <-}\StringTok{ }\OtherTok{NA}
\NormalTok{intensities}\OperatorTok{$}\NormalTok{ParticipantID[intensities}\OperatorTok{$}\NormalTok{ParticipantID }\OperatorTok{==}\StringTok{ }\DecValTok{17}\NormalTok{] <-}\StringTok{ }\OtherTok{NA}
\NormalTok{intensities}\OperatorTok{$}\NormalTok{ParticipantID[intensities}\OperatorTok{$}\NormalTok{ParticipantID }\OperatorTok{==}\StringTok{ }\DecValTok{20}\NormalTok{] <-}\StringTok{ }\OtherTok{NA}
\NormalTok{intensities <-}\StringTok{ }\KeywordTok{na.omit}\NormalTok{(intensities)}

\KeywordTok{pairwise_t_test}\NormalTok{(}\DataTypeTok{data =}\NormalTok{ intensities, Sensation }\OperatorTok{~}\StringTok{ }\NormalTok{Calibration, }\DataTypeTok{p.adjust.method =} \StringTok{"bonferroni"}\NormalTok{, }\DataTypeTok{paired =} \OtherTok{TRUE}\NormalTok{, }\DataTypeTok{alternative =} \StringTok{"two.sided"}\NormalTok{, }\DataTypeTok{detailed =} \OtherTok{TRUE}\NormalTok{)}
\end{Highlighting}
\end{Shaded}

\begin{verbatim}
## # A tibble: 3 x 15
##   estimate .y.   group1 group2    n1    n2 statistic       p    df conf.low
## *    <dbl> <chr> <chr>  <chr>  <int> <int>     <dbl>   <dbl> <dbl>    <dbl>
## 1  -0.532  Sens~ initi~ middle    19    19    -6.95  1.71e-6    18   -0.692
## 2  -0.484  Sens~ initi~ final     19    19    -3.23  5.00e-3    18   -0.799
## 3   0.0474 Sens~ middle final     19    19     0.339 7.38e-1    18   -0.246
## # ... with 5 more variables: conf.high <dbl>, method <chr>, alternative <chr>,
## #   p.adj <dbl>, p.adj.signif <chr>
\end{verbatim}

\begin{Shaded}
\begin{Highlighting}[]
\KeywordTok{pairwise_t_test}\NormalTok{(}\DataTypeTok{data =}\NormalTok{ intensities, Pain }\OperatorTok{~}\StringTok{ }\NormalTok{Calibration, }\DataTypeTok{p.adjust.method =} \StringTok{"bonferroni"}\NormalTok{, }\DataTypeTok{paired =} \OtherTok{TRUE}\NormalTok{, }\DataTypeTok{alternative =} \StringTok{"two.sided"}\NormalTok{, }\DataTypeTok{detailed =} \OtherTok{TRUE}\NormalTok{)}
\end{Highlighting}
\end{Shaded}

\begin{verbatim}
## # A tibble: 3 x 15
##   estimate .y.   group1 group2    n1    n2 statistic       p    df conf.low
## *    <dbl> <chr> <chr>  <chr>  <int> <int>     <dbl>   <dbl> <dbl>    <dbl>
## 1  -0.884  Pain  initi~ middle    19    19    -5.70  2.09e-5    18   -1.21 
## 2  -0.811  Pain  initi~ final     19    19    -5.03  8.69e-5    18   -1.15 
## 3   0.0737 Pain  middle final     19    19     0.673 5.09e-1    18   -0.156
## # ... with 5 more variables: conf.high <dbl>, method <chr>, alternative <chr>,
## #   p.adj <dbl>, p.adj.signif <chr>
\end{verbatim}

\begin{Shaded}
\begin{Highlighting}[]
\KeywordTok{pairwise_t_test}\NormalTok{(}\DataTypeTok{data =}\NormalTok{ intensities, ActualValue }\OperatorTok{~}\StringTok{ }\NormalTok{Calibration, }\DataTypeTok{p.adjust.method =} \StringTok{"bonferroni"}\NormalTok{, }\DataTypeTok{paired =} \OtherTok{TRUE}\NormalTok{, }\DataTypeTok{alternative =} \StringTok{"two.sided"}\NormalTok{, }\DataTypeTok{detailed =} \OtherTok{TRUE}\NormalTok{)}
\end{Highlighting}
\end{Shaded}

\begin{verbatim}
## # A tibble: 3 x 15
##   estimate .y.   group1 group2    n1    n2 statistic       p    df conf.low
## *    <dbl> <chr> <chr>  <chr>  <int> <int>     <dbl>   <dbl> <dbl>    <dbl>
## 1  -0.732  Actu~ initi~ middle    19    19    -7.25  9.64e-7    18   -0.944
## 2  -0.689  Actu~ initi~ final     19    19    -5.25  5.42e-5    18   -0.965
## 3   0.0421 Actu~ middle final     19    19     0.408 6.88e-1    18   -0.175
## # ... with 5 more variables: conf.high <dbl>, method <chr>, alternative <chr>,
## #   p.adj <dbl>, p.adj.signif <chr>
\end{verbatim}

\#OK, we have similar results again. So, it doesn't seem that for the
lower performance in the 1st part the reason was an inappropriate
calibration. On the other hand, a familiarization with the
electrotactile feedback seems to explain better the difference in the
performance between part 1 and part 2. To clarify, by familiarization I
mean the acceptance of the electrotactile feedback (e.g., it doesnt
startle or frighten the user) as well as the cognitive association (in
psychological terms: conditioning, or in game terms: game mechanics)
between an X event (e.g., when I feel that) and Y action (e.g., then I
stop or I adjust the position of my hand).

Questionnaires

\#Now let's check the questionnaires' results

\#For Questionnaires is better to compare the medians (i.e.,
non-parametric tests) because the responses are not real numbers, the
responses are ordinal data which may be better interpreted as ranks.

\begin{Shaded}
\begin{Highlighting}[]
\NormalTok{quest <-}\StringTok{ }\KeywordTok{read_csv}\NormalTok{(}\StringTok{"C:/repos/contactExperimentRNotebook/Questionnaires.csv"}\NormalTok{)}
\end{Highlighting}
\end{Shaded}

\begin{verbatim}
## 
## -- Column specification --------------------------------------------------------
## cols(
##   .default = col_double(),
##   Gender = col_character(),
##   Handedness = col_character(),
##   `Important Comments` = col_character()
## )
## i Use `spec()` for the full column specifications.
\end{verbatim}

\begin{Shaded}
\begin{Highlighting}[]
\CommentTok{#Lets check the median and the mode of the responses (These should be reported in a table)}
\CommentTok{#TableQuestionnaire <- }

\KeywordTok{median}\NormalTok{(quest}\OperatorTok{$}\NormalTok{VisualUseful)}
\end{Highlighting}
\end{Shaded}

\begin{verbatim}
## [1] 5.5
\end{verbatim}

\begin{Shaded}
\begin{Highlighting}[]
\KeywordTok{Mode}\NormalTok{(quest}\OperatorTok{$}\NormalTok{VisualUseful)}
\end{Highlighting}
\end{Shaded}

\begin{verbatim}
## [1] 5
## attr(,"freq")
## [1] 9
\end{verbatim}

\begin{Shaded}
\begin{Highlighting}[]
\KeywordTok{median}\NormalTok{(quest}\OperatorTok{$}\NormalTok{ElectrotactileUseful)}
\end{Highlighting}
\end{Shaded}

\begin{verbatim}
## [1] 6
\end{verbatim}

\begin{Shaded}
\begin{Highlighting}[]
\KeywordTok{Mode}\NormalTok{(quest}\OperatorTok{$}\NormalTok{ElectrotactileUseful)}
\end{Highlighting}
\end{Shaded}

\begin{verbatim}
## [1] 5
## attr(,"freq")
## [1] 8
\end{verbatim}

\begin{Shaded}
\begin{Highlighting}[]
\KeywordTok{median}\NormalTok{(quest}\OperatorTok{$}\NormalTok{VisualCoherent)}
\end{Highlighting}
\end{Shaded}

\begin{verbatim}
## [1] 5
\end{verbatim}

\begin{Shaded}
\begin{Highlighting}[]
\KeywordTok{Mode}\NormalTok{(quest}\OperatorTok{$}\NormalTok{VisualCoherent)}
\end{Highlighting}
\end{Shaded}

\begin{verbatim}
## [1] 5
## attr(,"freq")
## [1] 10
\end{verbatim}

\begin{Shaded}
\begin{Highlighting}[]
\KeywordTok{median}\NormalTok{(quest}\OperatorTok{$}\NormalTok{RelyingMoreOn)}
\end{Highlighting}
\end{Shaded}

\begin{verbatim}
## [1] 4.5
\end{verbatim}

\begin{Shaded}
\begin{Highlighting}[]
\KeywordTok{Mode}\NormalTok{(quest}\OperatorTok{$}\NormalTok{RelyingMoreOn)}
\end{Highlighting}
\end{Shaded}

\begin{verbatim}
## [1] 3
## attr(,"freq")
## [1] 9
\end{verbatim}

\begin{Shaded}
\begin{Highlighting}[]
\KeywordTok{median}\NormalTok{(quest}\OperatorTok{$}\NormalTok{ElectricalCoherent)}
\end{Highlighting}
\end{Shaded}

\begin{verbatim}
## [1] 5
\end{verbatim}

\begin{Shaded}
\begin{Highlighting}[]
\KeywordTok{Mode}\NormalTok{(quest}\OperatorTok{$}\NormalTok{ElectricalCoherent)}
\end{Highlighting}
\end{Shaded}

\begin{verbatim}
## [1] 5
## attr(,"freq")
## [1] 12
\end{verbatim}

\begin{Shaded}
\begin{Highlighting}[]
\KeywordTok{median}\NormalTok{(quest}\OperatorTok{$}\NormalTok{VisualResembling)}
\end{Highlighting}
\end{Shaded}

\begin{verbatim}
## [1] 3
\end{verbatim}

\begin{Shaded}
\begin{Highlighting}[]
\KeywordTok{Mode}\NormalTok{(quest}\OperatorTok{$}\NormalTok{VisualResembling)}
\end{Highlighting}
\end{Shaded}

\begin{verbatim}
## [1] 3
## attr(,"freq")
## [1] 7
\end{verbatim}

\begin{Shaded}
\begin{Highlighting}[]
\KeywordTok{median}\NormalTok{(quest}\OperatorTok{$}\NormalTok{ModalitiesSynchronized)}
\end{Highlighting}
\end{Shaded}

\begin{verbatim}
## [1] 6
\end{verbatim}

\begin{Shaded}
\begin{Highlighting}[]
\KeywordTok{Mode}\NormalTok{(quest}\OperatorTok{$}\NormalTok{ModalitiesSynchronized)}
\end{Highlighting}
\end{Shaded}

\begin{verbatim}
## [1] 7
## attr(,"freq")
## [1] 8
\end{verbatim}

\begin{Shaded}
\begin{Highlighting}[]
\KeywordTok{median}\NormalTok{(quest}\OperatorTok{$}\NormalTok{ElectrotactileResembling)}
\end{Highlighting}
\end{Shaded}

\begin{verbatim}
## [1] 2.5
\end{verbatim}

\begin{Shaded}
\begin{Highlighting}[]
\KeywordTok{Mode}\NormalTok{(quest}\OperatorTok{$}\NormalTok{ElectrotactileResembling)}
\end{Highlighting}
\end{Shaded}

\begin{verbatim}
## [1] 1
## attr(,"freq")
## [1] 8
\end{verbatim}

\begin{Shaded}
\begin{Highlighting}[]
\KeywordTok{median}\NormalTok{(quest}\OperatorTok{$}\NormalTok{CombinedResembling)}
\end{Highlighting}
\end{Shaded}

\begin{verbatim}
## [1] 4
\end{verbatim}

\begin{Shaded}
\begin{Highlighting}[]
\KeywordTok{Mode}\NormalTok{(quest}\OperatorTok{$}\NormalTok{CombinedResembling)}
\end{Highlighting}
\end{Shaded}

\begin{verbatim}
## [1] 5
## attr(,"freq")
## [1] 7
\end{verbatim}

\begin{Shaded}
\begin{Highlighting}[]
\CommentTok{#Usefulness}
\KeywordTok{wilcox.test}\NormalTok{(quest}\OperatorTok{$}\NormalTok{VisualUseful, quest}\OperatorTok{$}\NormalTok{ElectrotactileUseful, }
                           \DataTypeTok{alternative =} \StringTok{"two.sided"}\NormalTok{, }
                           \DataTypeTok{paired =} \OtherTok{TRUE}\NormalTok{,}
                           \DataTypeTok{exact =} \OtherTok{FALSE}\NormalTok{, }
                           \DataTypeTok{correct =} \OtherTok{FALSE}\NormalTok{, }
                           \DataTypeTok{conf.int =} \OtherTok{TRUE}\NormalTok{, }
                           \DataTypeTok{data =}\NormalTok{ quest) }
\end{Highlighting}
\end{Shaded}

\begin{verbatim}
## 
##  Wilcoxon signed rank test
## 
## data:  quest$VisualUseful and quest$ElectrotactileUseful
## V = 57, p-value = 0.5436
## alternative hypothesis: true location shift is not equal to 0
## 95 percent confidence interval:
##  -1.0000388  0.5000392
## sample estimates:
## (pseudo)median 
##   -4.04176e-05
\end{verbatim}

\begin{Shaded}
\begin{Highlighting}[]
\CommentTok{#Coherence }
\KeywordTok{wilcox.test}\NormalTok{(quest}\OperatorTok{$}\NormalTok{VisualCoherent, quest}\OperatorTok{$}\NormalTok{ElectricalCoherent, }
                           \DataTypeTok{alternative =} \StringTok{"two.sided"}\NormalTok{, }
                           \DataTypeTok{paired =} \OtherTok{TRUE}\NormalTok{,}
                           \DataTypeTok{exact =} \OtherTok{FALSE}\NormalTok{, }
                           \DataTypeTok{correct =} \OtherTok{FALSE}\NormalTok{, }
                           \DataTypeTok{conf.int =} \OtherTok{TRUE}\NormalTok{, }
                           \DataTypeTok{data =}\NormalTok{ quest) }
\end{Highlighting}
\end{Shaded}

\begin{verbatim}
## 
##  Wilcoxon signed rank test
## 
## data:  quest$VisualCoherent and quest$ElectricalCoherent
## V = 18, p-value = 1
## alternative hypothesis: true location shift is not equal to 0
## 95 percent confidence interval:
##  -1.000049  1.000049
## sample estimates:
## (pseudo)median 
##              0
\end{verbatim}

\begin{Shaded}
\begin{Highlighting}[]
\CommentTok{#Resemblance}
\KeywordTok{wilcox.test}\NormalTok{(quest}\OperatorTok{$}\NormalTok{VisualResembling, quest}\OperatorTok{$}\NormalTok{ElectrotactileResembling, }
                           \DataTypeTok{alternative =} \StringTok{"two.sided"}\NormalTok{, }
                           \DataTypeTok{paired =} \OtherTok{TRUE}\NormalTok{,}
                           \DataTypeTok{exact =} \OtherTok{FALSE}\NormalTok{, }
                           \DataTypeTok{correct =} \OtherTok{FALSE}\NormalTok{, }
                           \DataTypeTok{conf.int =} \OtherTok{TRUE}\NormalTok{, }
                           \DataTypeTok{data =}\NormalTok{ quest) }
\end{Highlighting}
\end{Shaded}

\begin{verbatim}
## 
##  Wilcoxon signed rank test
## 
## data:  quest$VisualResembling and quest$ElectrotactileResembling
## V = 130, p-value = 0.1472
## alternative hypothesis: true location shift is not equal to 0
## 95 percent confidence interval:
##  -2.426288e-05  1.500011e+00
## sample estimates:
## (pseudo)median 
##      0.5000355
\end{verbatim}

\begin{Shaded}
\begin{Highlighting}[]
\CommentTok{#No differences!}
\CommentTok{###### Let's check on which type of feedback the users relied upon more}

\CommentTok{#Includes Visual, Both, Electrotactile. Values: 0 or 1}
\NormalTok{Reliance <-}\StringTok{ }\KeywordTok{read_csv}\NormalTok{(}\StringTok{"C:/repos/contactExperimentRNotebook/reliance.csv"}\NormalTok{)}
\end{Highlighting}
\end{Shaded}

\begin{verbatim}
## 
## -- Column specification --------------------------------------------------------
## cols(
##   ID = col_double(),
##   Type = col_character(),
##   Reliance = col_double()
## )
\end{verbatim}

\begin{Shaded}
\begin{Highlighting}[]
\NormalTok{Reliance}\OperatorTok{$}\NormalTok{Type <-}\StringTok{ }\KeywordTok{factor}\NormalTok{(Reliance}\OperatorTok{$}\NormalTok{Type, }\DataTypeTok{levels =} \KeywordTok{c}\NormalTok{(}\StringTok{"Both"}\NormalTok{, }\StringTok{"Visual"}\NormalTok{, }\StringTok{"Electrotactile"}\NormalTok{))}
\KeywordTok{pairwise.wilcox.test}\NormalTok{(Reliance}\OperatorTok{$}\NormalTok{Reliance, Reliance}\OperatorTok{$}\NormalTok{Type, }\DataTypeTok{alternative =} \StringTok{"greater"}\NormalTok{,}\DataTypeTok{p.adjust.method =} \StringTok{"bonferroni"}\NormalTok{)}
\end{Highlighting}
\end{Shaded}

\begin{verbatim}
## Warning in wilcox.test.default(xi, xj, paired = paired, ...): cannot compute
## exact p-value with ties

## Warning in wilcox.test.default(xi, xj, paired = paired, ...): cannot compute
## exact p-value with ties

## Warning in wilcox.test.default(xi, xj, paired = paired, ...): cannot compute
## exact p-value with ties
\end{verbatim}

\begin{verbatim}
## 
##  Pairwise comparisons using Wilcoxon rank sum test with continuity correction 
## 
## data:  Reliance$Reliance and Reliance$Type 
## 
##                Both  Visual
## Visual         0.288 -     
## Electrotactile 0.024 0.377 
## 
## P value adjustment method: bonferroni
\end{verbatim}

\begin{Shaded}
\begin{Highlighting}[]
\CommentTok{#Includes Visual and Electrotactile. Values: 0,1,2,3}
\NormalTok{RelianceMore <-}\StringTok{ }\KeywordTok{read_csv}\NormalTok{(}\StringTok{"C:/repos/contactExperimentRNotebook/relianceMore.csv"}\NormalTok{)}
\end{Highlighting}
\end{Shaded}

\begin{verbatim}
## 
## -- Column specification --------------------------------------------------------
## cols(
##   ID = col_double(),
##   TypeMore = col_character(),
##   RelianceMore = col_double()
## )
\end{verbatim}

\begin{Shaded}
\begin{Highlighting}[]
\NormalTok{RelianceMore<-}\StringTok{ }\KeywordTok{dcast}\NormalTok{(RelianceMore,ID }\OperatorTok{~}\StringTok{ }\NormalTok{TypeMore, }\DataTypeTok{value.var=} \StringTok{"RelianceMore"}\NormalTok{)}
\end{Highlighting}
\end{Shaded}

\begin{verbatim}
## Warning in dcast(RelianceMore, ID ~ TypeMore, value.var = "RelianceMore"): The
## dcast generic in data.table has been passed a spec_tbl_df and will attempt to
## redirect to the reshape2::dcast; please note that reshape2 is deprecated, and
## this redirection is now deprecated as well. Please do this redirection yourself
## like reshape2::dcast(RelianceMore). In the next version, this warning will
## become an error.
\end{verbatim}

\begin{Shaded}
\begin{Highlighting}[]
\KeywordTok{wilcox.test}\NormalTok{(RelianceMore}\OperatorTok{$}\NormalTok{Electrotactile, RelianceMore}\OperatorTok{$}\NormalTok{Visual,}
                           \DataTypeTok{alternative =} \StringTok{"greater"}\NormalTok{, }
                           \DataTypeTok{paired =} \OtherTok{TRUE}\NormalTok{,}
                           \DataTypeTok{exact =} \OtherTok{FALSE}\NormalTok{, }
                           \DataTypeTok{correct =} \OtherTok{FALSE}\NormalTok{, }
                           \DataTypeTok{conf.int =} \OtherTok{TRUE}\NormalTok{, }
                           \DataTypeTok{data =}\NormalTok{ RelianceMore) }
\end{Highlighting}
\end{Shaded}

\begin{verbatim}
## 
##  Wilcoxon signed rank test
## 
## data:  RelianceMore$Electrotactile and RelianceMore$Visual
## V = 142, p-value = 0.07042
## alternative hypothesis: true location shift is greater than 0
## 95 percent confidence interval:
##  -1.206806e-05           Inf
## sample estimates:
## (pseudo)median 
##      0.4999568
\end{verbatim}

\begin{Shaded}
\begin{Highlighting}[]
\CommentTok{###### Let's check which resembled better touching a surface}
\NormalTok{Resemble <-}\StringTok{ }\KeywordTok{read_csv}\NormalTok{(}\StringTok{"C:/repos/contactExperimentRNotebook/resemblance3factors.csv"}\NormalTok{)}
\end{Highlighting}
\end{Shaded}

\begin{verbatim}
## 
## -- Column specification --------------------------------------------------------
## cols(
##   ID = col_double(),
##   Type = col_character(),
##   Resemblance = col_double()
## )
\end{verbatim}

\begin{Shaded}
\begin{Highlighting}[]
\KeywordTok{pairwise.wilcox.test}\NormalTok{(Resemble}\OperatorTok{$}\NormalTok{Resemblance, Resemble}\OperatorTok{$}\NormalTok{Type, }\DataTypeTok{p.adjust.method =} \StringTok{"bonferroni"}\NormalTok{)}
\end{Highlighting}
\end{Shaded}

\begin{verbatim}
## Warning in wilcox.test.default(xi, xj, paired = paired, ...): cannot compute
## exact p-value with ties

## Warning in wilcox.test.default(xi, xj, paired = paired, ...): cannot compute
## exact p-value with ties

## Warning in wilcox.test.default(xi, xj, paired = paired, ...): cannot compute
## exact p-value with ties
\end{verbatim}

\begin{verbatim}
## 
##  Pairwise comparisons using Wilcoxon rank sum test with continuity correction 
## 
## data:  Resemble$Resemblance and Resemble$Type 
## 
##                Combined Electrotactile
## Electrotactile 0.074    -             
## Visual         0.596    0.655         
## 
## P value adjustment method: bonferroni
\end{verbatim}

\#Note for the interpertation:

\#Median: The median value is the number that is in the middle, with the
same amount of numbers below and above.

\#Mode: The mode is the most commonly/frequently observed value in a set
of data (i.e., the most frequent response)

Visual

\#Usefulness

\#Median:5.5

\#Mode: 5

\#Interpertation: Useful

\#Coherence

\#Median:5

\#Mode: 5

\#Interpertation: Coherent

\#Resemblance

\#Median:3

\#Mode: 3

\#Interpertation: Different

Electrotactile

\#Usefulness

\#Median:6

\#Mode: 5 (8 Response)

\#Interpertation: Useful to Very Useful

\#Coherence

\#Median:5

\#Mode: 5

\#Interpertation: Coherent

\#Resemblance

\#Median: 2.5

\#Mode: 1 (8 responses)

\#Interpertation: Different to Extremely Different

Combined

\#Usefulness (Rely more on visual or electrotactile)

\#Median:4.5 (i.e., half of the responses indicated \textgreater{} 4
which means 12 preferred the electrotactile since 5-7 corresponds to
electrotactile. Note that \textless{} 4.5, means that the other 12
responses were for both visual and combined feedback. combined = 4,
while visual = 1 - 3)

\#Mode: 3 (9 Responses)

\#Interpertation: Ballanced, however it leans towards Electrotactile

\#Modalities synchronized

\#Median:6

\#Mode: 7

\#Interpertation: Very Coherent to Completely Coherent

\#Resemblance

\#Median:4

\#Mode: 5

\#Interpertation: Moderately Similar to Similar

Comparisons

\#1) In Terms of Usefulness, Coherence, and resemblance

\#There are not significant differences between Combined, Visual, and
Electrotatcile feedback, which replicates our findings on the users'
performance in part 2, where we did not find any significant differences
as well. In combination with the medians and modes for each question
(see above), we may infer that electrotactile feedback was equivalent to
visual feedback, as well as it received positive evaluations by the
users especially in terms of usefulness and coherence.

\#2) Reliance

\#Again there were no significant differences amongst Combined, Visual,
and Electrotatcile feedback. The only significant difference that was
detected was between Electrotactile and Combined feedback (in favor of
electrotactile). In general, considering also the medians and modes of
the responses on this question (see also the bar chart below), we may
infer that the users relied more on a single type of feedback in the
combined feedback condition, and this single type of feedback was most
of the times the electrotactile feedback (12 users), followed by visual
feedback (8 users) and both (4 users). These findings again align with
the the results on the performance in part 2.

\hypertarget{resemblance-part2}{%
\section{Resemblance Part2}\label{resemblance-part2}}

\#Regarding the electrotactile feedback, our findings replicate the
findings of previous studies which found that the sensation of touching
provided by electrotactile feedback substantially deviates from the
sensation of touching in real world.

\#Now, Let's visualize the data

\begin{Shaded}
\begin{Highlighting}[]
\CommentTok{#Visualization of the responses in terms of Usefulness, Coherence, and Resemblance}

\NormalTok{visualVSelectro <-}\StringTok{ }\KeywordTok{read_csv}\NormalTok{(}\StringTok{"C:/repos/contactExperimentRNotebook/visualVSelectro.csv"}\NormalTok{)}
\end{Highlighting}
\end{Shaded}

\begin{verbatim}
## 
## -- Column specification --------------------------------------------------------
## cols(
##   ID = col_double(),
##   Type = col_character(),
##   Usefulnes = col_double(),
##   Resemblance = col_double(),
##   Coherence = col_double()
## )
\end{verbatim}

\begin{Shaded}
\begin{Highlighting}[]
\NormalTok{?}\KeywordTok{ggbetweenstats}\NormalTok{()}
\end{Highlighting}
\end{Shaded}

\begin{verbatim}
## starting httpd help server ...
\end{verbatim}

\begin{verbatim}
##  done
\end{verbatim}

\begin{Shaded}
\begin{Highlighting}[]
\NormalTok{p12 <-}\StringTok{ }\NormalTok{ggstatsplot}\OperatorTok{::}\KeywordTok{ggbetweenstats}\NormalTok{(}
  \DataTypeTok{data =}\NormalTok{ visualVSelectro,}
  \DataTypeTok{x =} \StringTok{"Type"}\NormalTok{,}
  \DataTypeTok{y =} \StringTok{"Usefulnes"}\NormalTok{,}
  \DataTypeTok{type =} \StringTok{"np"}\NormalTok{,}
  \DataTypeTok{pairwise.comparisons =} \OtherTok{FALSE}\NormalTok{,}
  \DataTypeTok{pairwise.display =} \StringTok{"significant"}\NormalTok{,}
  \DataTypeTok{p.adjust.method =} \StringTok{"bonferroni"}\NormalTok{,}
  \DataTypeTok{effsize.type =} \StringTok{"unbiased"}\NormalTok{,}
  \DataTypeTok{results.subtitle =} \OtherTok{FALSE}\NormalTok{,}
  \DataTypeTok{xlab =} \StringTok{"Type of Feedback"}\NormalTok{,}
  \DataTypeTok{ylab =} \StringTok{"Usefulness"}\NormalTok{,}
  \DataTypeTok{sample.size.label =} \OtherTok{FALSE}\NormalTok{,}
  \DataTypeTok{var.equal =} \OtherTok{FALSE}\NormalTok{,}
  \DataTypeTok{mean.plotting =} \OtherTok{FALSE}\NormalTok{,}
  \DataTypeTok{mean.ci =} \OtherTok{TRUE}\NormalTok{,}
  \DataTypeTok{paired =} \OtherTok{TRUE}\NormalTok{,}
  \DataTypeTok{title.text =} \StringTok{"Usefulness Box-Violin Plots"}\NormalTok{,}
  \DataTypeTok{title.color =} \StringTok{"black"}\NormalTok{,}
  \DataTypeTok{caption.color =} \StringTok{"black"}
\NormalTok{  )}
\NormalTok{p12}
\end{Highlighting}
\end{Shaded}

\includegraphics{Report_files/figure-latex/unnamed-chunk-17-1.pdf}

\begin{Shaded}
\begin{Highlighting}[]
\NormalTok{p13 <-}\StringTok{ }\NormalTok{ggstatsplot}\OperatorTok{::}\KeywordTok{ggbetweenstats}\NormalTok{(}
  \DataTypeTok{data =}\NormalTok{ visualVSelectro,}
  \DataTypeTok{x =} \StringTok{"Type"}\NormalTok{,}
  \DataTypeTok{y =} \StringTok{"Resemblance"}\NormalTok{,}
  \DataTypeTok{type =} \StringTok{"np"}\NormalTok{,}
  \DataTypeTok{pairwise.comparisons =} \OtherTok{FALSE}\NormalTok{,}
  \DataTypeTok{pairwise.display =} \StringTok{"significant"}\NormalTok{,}
  \DataTypeTok{p.adjust.method =} \StringTok{"bonferroni"}\NormalTok{,}
  \DataTypeTok{effsize.type =} \StringTok{"unbiased"}\NormalTok{,}
  \DataTypeTok{results.subtitle =} \OtherTok{FALSE}\NormalTok{,}
  \DataTypeTok{xlab =} \StringTok{"Type of Feedback"}\NormalTok{,}
  \DataTypeTok{ylab =} \StringTok{"Resemblance"}\NormalTok{,}
  \DataTypeTok{sample.size.label =} \OtherTok{FALSE}\NormalTok{,}
  \DataTypeTok{var.equal =} \OtherTok{FALSE}\NormalTok{,}
  \DataTypeTok{mean.plotting =} \OtherTok{FALSE}\NormalTok{,}
  \DataTypeTok{mean.ci =} \OtherTok{TRUE}\NormalTok{,}
  \DataTypeTok{paired =} \OtherTok{TRUE}\NormalTok{,}
  \DataTypeTok{title.text =} \StringTok{"Resemblance Box-Violin Plots"}\NormalTok{,}
  \DataTypeTok{title.color =} \StringTok{"black"}\NormalTok{,}
  \DataTypeTok{caption.color =} \StringTok{"black"}
\NormalTok{  )}
\NormalTok{p13}
\end{Highlighting}
\end{Shaded}

\includegraphics{Report_files/figure-latex/unnamed-chunk-17-2.pdf}

\begin{Shaded}
\begin{Highlighting}[]
\NormalTok{p14 <-}\StringTok{ }\NormalTok{ggstatsplot}\OperatorTok{::}\KeywordTok{ggbetweenstats}\NormalTok{(}
  \DataTypeTok{data =}\NormalTok{ visualVSelectro,}
  \DataTypeTok{x =} \StringTok{"Type"}\NormalTok{,}
  \DataTypeTok{y =} \StringTok{"Coherence"}\NormalTok{,}
  \DataTypeTok{type =} \StringTok{"np"}\NormalTok{,}
  \DataTypeTok{pairwise.comparisons =} \OtherTok{FALSE}\NormalTok{,}
  \DataTypeTok{pairwise.display =} \StringTok{"significant"}\NormalTok{,}
  \DataTypeTok{p.adjust.method =} \StringTok{"bonferroni"}\NormalTok{,}
  \DataTypeTok{effsize.type =} \StringTok{"unbiased"}\NormalTok{,}
  \DataTypeTok{results.subtitle =} \OtherTok{FALSE}\NormalTok{,}
  \DataTypeTok{xlab =} \StringTok{"Type of Feedback"}\NormalTok{,}
  \DataTypeTok{ylab =} \StringTok{"Coherence"}\NormalTok{,}
  \DataTypeTok{sample.size.label =} \OtherTok{FALSE}\NormalTok{,}
  \DataTypeTok{var.equal =} \OtherTok{FALSE}\NormalTok{,}
  \DataTypeTok{mean.plotting =} \OtherTok{FALSE}\NormalTok{,}
  \DataTypeTok{mean.ci =} \OtherTok{TRUE}\NormalTok{,}
  \DataTypeTok{paired =} \OtherTok{TRUE}\NormalTok{,}
  \DataTypeTok{title.text =} \StringTok{"Coherence Box-Violin Plots"}\NormalTok{,}
  \DataTypeTok{title.color =} \StringTok{"black"}\NormalTok{,}
  \DataTypeTok{caption.color =} \StringTok{"black"}
\NormalTok{  )}
\NormalTok{p14}
\end{Highlighting}
\end{Shaded}

\includegraphics{Report_files/figure-latex/unnamed-chunk-17-3.pdf}

\begin{Shaded}
\begin{Highlighting}[]
\NormalTok{Preferences <-}\StringTok{ }\KeywordTok{read_csv}\NormalTok{(}\StringTok{"C:/repos/contactExperimentRNotebook/prefFeed.csv"}\NormalTok{)}
\end{Highlighting}
\end{Shaded}

\begin{verbatim}
## 
## -- Column specification --------------------------------------------------------
## cols(
##   Reliance = col_double(),
##   Type = col_character()
## )
\end{verbatim}

\begin{Shaded}
\begin{Highlighting}[]
\CommentTok{#This will show on which type of feedback the users relied upon more}
\KeywordTok{ggplot}\NormalTok{(Preferences) }\OperatorTok{+}\StringTok{ }\KeywordTok{geom_bar}\NormalTok{(}\KeywordTok{aes}\NormalTok{(}\DataTypeTok{x =}\NormalTok{ Preferences}\OperatorTok{$}\NormalTok{Type))}
\end{Highlighting}
\end{Shaded}

\includegraphics{Report_files/figure-latex/unnamed-chunk-17-4.pdf}

\begin{Shaded}
\begin{Highlighting}[]
\CommentTok{#This will show on which type of feedback the users relied upon more}
\KeywordTok{hist}\NormalTok{(quest}\OperatorTok{$}\NormalTok{RelyingMoreOn)}
\end{Highlighting}
\end{Shaded}

\includegraphics{Report_files/figure-latex/unnamed-chunk-17-5.pdf}

\begin{Shaded}
\begin{Highlighting}[]
\NormalTok{p15 <-}\StringTok{ }\NormalTok{ggstatsplot}\OperatorTok{::}\KeywordTok{ggbetweenstats}\NormalTok{(}
  \DataTypeTok{data =}\NormalTok{ Reliance,}
  \DataTypeTok{x =} \StringTok{"Type"}\NormalTok{,}
  \DataTypeTok{y =} \StringTok{"Reliance"}\NormalTok{,}
  \DataTypeTok{type =} \StringTok{"np"}\NormalTok{,}
  \DataTypeTok{pairwise.comparisons =} \OtherTok{FALSE}\NormalTok{,}
  \DataTypeTok{pairwise.display =} \StringTok{"significant"}\NormalTok{,}
  \DataTypeTok{p.adjust.method =} \StringTok{"bonferroni"}\NormalTok{,}
  \DataTypeTok{effsize.type =} \StringTok{"unbiased"}\NormalTok{,}
  \DataTypeTok{results.subtitle =} \OtherTok{FALSE}\NormalTok{,}
  \DataTypeTok{xlab =} \StringTok{"Type of Feedback"}\NormalTok{,}
  \DataTypeTok{ylab =} \StringTok{"Reliance"}\NormalTok{,}
  \DataTypeTok{sample.size.label =} \OtherTok{FALSE}\NormalTok{,}
  \DataTypeTok{var.equal =} \OtherTok{FALSE}\NormalTok{,}
  \DataTypeTok{mean.plotting =} \OtherTok{FALSE}\NormalTok{,}
  \DataTypeTok{mean.ci =} \OtherTok{TRUE}\NormalTok{,}
  \DataTypeTok{paired =} \OtherTok{TRUE}\NormalTok{,}
  \DataTypeTok{title.text =} \StringTok{"Reliance Box-Violin Plots"}\NormalTok{,}
  \DataTypeTok{title.color =} \StringTok{"black"}\NormalTok{,}
  \DataTypeTok{caption.color =} \StringTok{"black"}
\NormalTok{  )}
\NormalTok{p15}
\end{Highlighting}
\end{Shaded}

\includegraphics{Report_files/figure-latex/unnamed-chunk-17-6.pdf}

\begin{Shaded}
\begin{Highlighting}[]
\CommentTok{#This will show which type of feedback resemble better touching a surface}
\NormalTok{p16 <-}\StringTok{ }\NormalTok{ggstatsplot}\OperatorTok{::}\KeywordTok{ggbetweenstats}\NormalTok{(}
  \DataTypeTok{data =}\NormalTok{ Resemble,}
  \DataTypeTok{x =} \StringTok{"Type"}\NormalTok{,}
  \DataTypeTok{y =} \StringTok{"Resemblance"}\NormalTok{,}
  \DataTypeTok{type =} \StringTok{"np"}\NormalTok{,}
  \DataTypeTok{pairwise.comparisons =} \OtherTok{FALSE}\NormalTok{,}
  \DataTypeTok{pairwise.display =} \StringTok{"significant"}\NormalTok{,}
  \DataTypeTok{p.adjust.method =} \StringTok{"bonferroni"}\NormalTok{,}
  \DataTypeTok{effsize.type =} \StringTok{"unbiased"}\NormalTok{,}
  \DataTypeTok{results.subtitle =} \OtherTok{FALSE}\NormalTok{,}
  \DataTypeTok{xlab =} \StringTok{"Type of Feedback"}\NormalTok{,}
  \DataTypeTok{ylab =} \StringTok{"Resemblance"}\NormalTok{,}
  \DataTypeTok{sample.size.label =} \OtherTok{FALSE}\NormalTok{,}
  \DataTypeTok{var.equal =} \OtherTok{FALSE}\NormalTok{,}
  \DataTypeTok{mean.plotting =} \OtherTok{FALSE}\NormalTok{,}
  \DataTypeTok{mean.ci =} \OtherTok{TRUE}\NormalTok{,}
  \DataTypeTok{paired =} \OtherTok{TRUE}\NormalTok{,}
  \DataTypeTok{title.text =} \StringTok{"Resemblance Box-Violin Plots"}\NormalTok{,}
  \DataTypeTok{title.color =} \StringTok{"black"}\NormalTok{,}
  \DataTypeTok{caption.color =} \StringTok{"black"}
\NormalTok{  )}
\NormalTok{p16}
\end{Highlighting}
\end{Shaded}

\includegraphics{Report_files/figure-latex/unnamed-chunk-17-7.pdf}

\#The rest of info on the electrotactile feedback from the
questionnaires

\begin{Shaded}
\begin{Highlighting}[]
\KeywordTok{median}\NormalTok{(quest}\OperatorTok{$}\NormalTok{ElectricalSensation)}
\end{Highlighting}
\end{Shaded}

\begin{verbatim}
## [1] 4
\end{verbatim}

\begin{Shaded}
\begin{Highlighting}[]
\KeywordTok{Mode}\NormalTok{(quest}\OperatorTok{$}\NormalTok{ElectricalSensation)}
\end{Highlighting}
\end{Shaded}

\begin{verbatim}
## [1] 4
## attr(,"freq")
## [1] 10
\end{verbatim}

\begin{Shaded}
\begin{Highlighting}[]
\CommentTok{#Moderate i.e., neither pleasant nor annoying}
\KeywordTok{hist}\NormalTok{(quest}\OperatorTok{$}\NormalTok{ElectricalSensation)}
\end{Highlighting}
\end{Shaded}

\includegraphics{Report_files/figure-latex/unnamed-chunk-18-1.pdf}

\begin{Shaded}
\begin{Highlighting}[]
\KeywordTok{median}\NormalTok{(quest}\OperatorTok{$}\NormalTok{ElectricalUpdatePerception)}
\end{Highlighting}
\end{Shaded}

\begin{verbatim}
## [1] 5.5
\end{verbatim}

\begin{Shaded}
\begin{Highlighting}[]
\KeywordTok{Mode}\NormalTok{(quest}\OperatorTok{$}\NormalTok{ElectricalUpdatePerception)}
\end{Highlighting}
\end{Shaded}

\begin{verbatim}
## [1] 5 6 7
## attr(,"freq")
## [1] 6
\end{verbatim}

\begin{Shaded}
\begin{Highlighting}[]
\CommentTok{# Often, Very often, All the time! }
\KeywordTok{hist}\NormalTok{(quest}\OperatorTok{$}\NormalTok{ElectricalUpdatePerception)}
\end{Highlighting}
\end{Shaded}

\includegraphics{Report_files/figure-latex/unnamed-chunk-18-2.pdf}

\#OK so we have a neutral feedback regarding the pleasantness or
discomfort of the sensation provided by the electrotactile feedback.
which I interpreted as a positive result since we did not recei,
especially considering the previous literature.

\#Regarding the perceptual update that the user presses harder the
surface, the results are really positive, and postulate that the
electrotactile feedback may contribute with strengthening the
plausibility illusion (i.e., the illusion that the virtual environment
responds to your action). However, this should be meticuously
investigated in a future study.

\end{document}
